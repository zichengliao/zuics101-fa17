\documentclass[12pt]{article}

\setlength{\topmargin}{-1in} \addtolength{\textheight}{4.00in}
\setlength{\oddsidemargin}{.00in} \addtolength{\textwidth}{1in}

\usepackage{amsmath,color,graphicx}

\usepackage{listings}      
\lstset{language=Python}   

\usepackage{amsfonts}
\usepackage{amsmath}
\usepackage{blindtext}
\usepackage{enumitem}


\nofiles

\pagestyle{empty}

\setlength{\parindent}{0in}


\begin{document}

\noindent {\sc {\bf {\Large Quiz 2}}
            \hfill ZJUI CS 101, Fall 2016}
%\bigskip

\noindent {\sc  {\large Student ID:}
            \hfill {\large Name:}
             \hfill}
%\bigskip

{\bf Question 1.}  What is the value of \texttt{x} in the end of the program below?
\begin{lstlisting}[frame=single]
	s = '%' + 'f'
	i = 4 / 8
	x = float(s % i) * 3
\end{lstlisting}
\vspace{-0.5cm}
\begin{enumerate}[label=\Alph*]
	\item  \texttt{'0.50.50.5'}
	\item  \texttt{'\%f\%f\%f'}
	\item  \texttt{1.5}
	\item  \texttt{'1.5'}
\end{enumerate}
%Answer: C

{\bf Question 2.}    What is the value of \texttt{x} in the end of the program below?
\begin{lstlisting}[frame=single]
  	s = "WATER MAIN"[1:5]
  	t = int(3.6)
  	x = s[-1] + s[t-2]
  \end{lstlisting}
  \vspace{-0.5cm}
  \begin{enumerate}[label=\Alph*]
  	\item  \texttt{"WA"}
  	\item  \texttt{"RA"}
  	\item  \texttt{" T"}
  	\item  \texttt{"RT"}
  \end{enumerate} 

%Answer: D

{\bf Question 3.} What does the following program print?
\begin{lstlisting}[frame=single]
    x = "5"
    y = "%i"
    print( (x+y) % 3)
\end{lstlisting}
\vspace{-0.5cm}
\begin{enumerate}[label=\Alph*]
	\item  \texttt{535353}
	\item  \texttt{555}
	\item  \texttt{53}
	\item  None of the above
\end{enumerate}
%Answer: C

{\bf Question 4.} What is the value of \texttt{x} in the end of the following program?
\begin{lstlisting}[frame=single]
  	i = len("WATER MAIN")
  	c = (1.0 + 2.0j) * (-i)
  	x = abs(min(c.real, -5))
  \end{lstlisting}
  \vspace{-0.5cm}
  \begin{enumerate}[label=\Alph*]
  	\item  \texttt{10.0}
  	\item  \texttt{11.0}
  	\item  \texttt{12.0}
  	\item  \texttt{13.0}
  \end{enumerate}
%Answer: A

\end{document}

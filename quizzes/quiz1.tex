\documentclass[12pt]{article}

\setlength{\topmargin}{-.9in} \addtolength{\textheight}{3.00in}
\setlength{\oddsidemargin}{.00in} \addtolength{\textwidth}{1in}

\usepackage{amsmath,color,graphicx}

\usepackage{listings}      
\lstset{language=Python}   

\usepackage{amsfonts}
\usepackage{amsmath}
\usepackage{blindtext}
\usepackage{enumitem}


\nofiles

\pagestyle{empty}

\setlength{\parindent}{0in}


\begin{document}

\noindent {\sc {\bf {\Large Quiz 1}}
            \hfill ZJUI CS 101, Fall 2016}
%\bigskip

\noindent {\sc  {\large Student ID:}
            \hfill {\large Name:}
             \hfill}
\bigskip

{\bf Question 1.} What does the following program print?
\begin{lstlisting}[frame=single]
   x = "6"
   y = 10 % 3
   print(x * y)
\end{lstlisting}

\begin{enumerate}[label=\Alph*]
	\item  6
	\item  1
	\item  66
	\item  61
\end{enumerate}

%Answer: A

{\bf Question 2.}  What are the type and value of \texttt{r} in the program below, respectively?
\begin{lstlisting}[frame=single]
  	c = (20 + 5j)
  	i = 25
  	r = c.real + i
\end{lstlisting}
  
   \begin{enumerate}[label=\Alph*]
   	\item  \texttt{int}, \texttt{45}
   	\item  \texttt{complex}, \texttt{45 + 5j}
   	\item  \texttt{float}, \texttt{45.0}
   	\item  \texttt{complex}, \texttt{45 + 0j}
   \end{enumerate}

%Answer: C

{\bf Question 3.} Which of the following expressions is most likely to cause an \textbf{overflow}?
 \begin{enumerate}[label=\Alph*]
% 	\item  \texttt{10 ** 100000}
 	\item  \texttt{"10" * 100000}
 	\item  \texttt{10.0 ** 100000}
 	\item  \texttt{"10" ** 100000}
% 	\item  None of the above
 \end{enumerate}
%Answer: B

{\bf Question 4.} What does the following program print?
\begin{lstlisting}[frame=single]
	x = "45"
	y = "%i"
	print( (x+y) % 2)
\end{lstlisting}
\begin{enumerate}[label=\Alph*]
	\item  \texttt{452}
	\item  \texttt{4444}
	\item  \texttt{4545}
	\item  None of the above
\end{enumerate}
%Answer: A

\end{document}

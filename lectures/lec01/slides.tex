%!TEX program = xelatex
\documentclass[11pt]{beamer}

\usepackage{amsfonts}
\usepackage{amsmath}
\usepackage{blindtext}
\usepackage{enumitem}
\usepackage{hyperref}
\usepackage{colortbl}
\usepackage{fancyvrb}
\usepackage{booktabs}

\hypersetup{pdfborder = {0 0 0}}

%\usetheme{SaoPaulo}  %commented by Tao to avoid error; replace it with the line below
\usetheme{Warsaw}




\title{Welcome to CS \emph{101}!}
\subtitle{Introduction to Programming}
\author{CS101 Lecture \#1}
\date{2016-09-26}

%\setcounter{showSlideNumbers}{1} %commented by Tao to avoid error

\begin{document}
%  \setcounter{showProgressBar}{0}  %commented by Tao to avoid error
%  \setcounter{showSlideNumbers}{0}  %commented by Tao to avoid error

%Tao added below to avoid errors
\newcommand{\Enlarge}{\large}
\newcommand{\CSBase}{blue}
\newcommand{\CSGradBot}{orange}
\newcommand{\CSAltDark}{black}
\newcommand{\CSPureBase}{blue}

\newcommand{\myitem}{\item}
\newcommand{\mysubitem}{\item}


%%%%%%%%%%%%%%%%%%%%%%%%%%%%%%%%%%%%%%%%%%%%%%%%%%%%%%%%%%%%%%%%%%%%%%%%%%%%%%%%
\frame{\titlepage}

%%%%%%%%%%%%%%%%%%%%%%%%%%%%%%%%%%%%%%%%%%%%%%%%%%%%%%%%%%%%%%%%%%%%%%%%%%%%%%%%
\setcounter{framenumber}{0}
%\setcounter{showProgressBar}{1}  %commented by Tao to avoid error
%\setcounter{showSlideNumbers}{1}  %commented by Tao to avoid error

%%%%%%%%%%%%%%%%%%%%%%%%%%%%%%%%%%%%%%%%%%%%%%%%%%%%%%%%%%%%%%%%%%%%%%%%%%%%%%%%
\section{Class Structure}

%%%%%%%%%%%%%%%%%%%%%%%%%%%%%%%%%%%%%%%%%%%%%%%%%%%%%%%%%%%%%%%%%%%%%%%%%%%%%%%%
\begin{frame}[plain,c]
  \frametitle{Class Website}
  \Enlarge 

  \begin{center}
    %\textcolor{CS101Base}{\Huge \texttt{go.illinois.edu/cs101}}%commented by Tao to avoid error; replace it with the line below
    \textcolor{\CSBase}{\small \texttt{\url{https://relate.cs.illinois.edu/course/zuics101fa16/}}}
  \end{center}
     Steps for enrolling in the course web:
     \begin{itemize}  
     	\myitem Step 1. Click the ``Sign in $>>$'' button near the top of the course web.
     	\myitem Step 2. Click the second button ``Sign in using your email $>>$''.   	
     	\myitem Step 3. Enter your \textbf{Zhejiang University email address} in the Email input box, and then click the ``Send sign-in email''.
     	\myitem Step 4. Click the URL included in the email titled ``Your RELATE sign-in link'' (sent to you) to sign in.
     	\myitem Step 5. Change the browser's URL to be \small{\url{https://relate.cs.illinois.edu/course/zuics101fa16/}}
     	\myitem \large{Step 6. Click the ``Enroll'' button near the top.}
     \end{itemize}
\end{frame}

%%%%%%%%%%%%%%%%%%%%%%%%%%%%%%%%%%%%%%%%%%%%%%%%%%%%%%%%%%%%%%%%%%%%%%%%%%%%%%%%
\begin{frame}
  \frametitle{Grading}
  \begin{tabular}{*{27}{ll}}
    \toprule
    20\% & Homework \\
    25\% & Labs \\
    10\% & Lecture Participation \\
    20\% & Midterms \\
    25\% & Final Exam \\
    \bottomrule
  \end{tabular}
\end{frame}

%%%%%%%%%%%%%%%%%%%%%%%%%%%%%%%%%%%%%%%%%%%%%%%%%%%%%%%%%%%%%%%%%%%%%%%%%%%%%%%%
\begin{frame}
  \frametitle{Required Supplies}
  %\Enlarge %commented by Tao to avoid error

  \begin{itemize}
    %\myitem i>clicker \\ \textcolor{\CSGradBot}{\footnotesize\hspace{1em} Grades count starting Wed 08-31} \pause  %commented by Tao to avoid error
    %\myitem CodeLab account \\ \textcolor{\CSGradBot}{\footnotesize\hspace{1em} Instructions in \texttt{hw01}} \pause %commented by Tao to avoid error
    %\myitem No textbook! %commented by Tao to avoid error
    \item CodeLab account \\ \textcolor{\CSGradBot}{\footnotesize\hspace{1em} Instructions in \texttt{hw01}} %\pause %commented by Tao to avoid error
     %\myitem No textbook! %commented by Tao to avoid error
  \end{itemize}
\end{frame}

%%%%%%%%%%%%%%%%%%%%%%%%%%%%%%%%%%%%%%%%%%%%%%%%%%%%%%%%%%%%%%%%%%%%%%%%%%%%%%%%
\begin{frame}
  \frametitle{Homework Policies}
  \Enlarge

  \begin{itemize}
    \myitem No late homework submissions. \pause
    \myitem All machine-generated grades are final. \pause
    \myitem Late registrants should keep up with work. \\ \textcolor{\CSGradBot}{\footnotesize\hspace{1em} Corollary:  No extensions or exceptions for late registration.} \pause
    \myitem Get help at Blackboard forum. \\ \textcolor{\CSGradBot}{\footnotesize\hspace{1em} Be civil to staff and peers. \\ All posts containing solutions should be marked as private.}
  \end{itemize}
\end{frame}

%%%%%%%%%%%%%%%%%%%%%%%%%%%%%%%%%%%%%%%%%%%%%%%%%%%%%%%%%%%%%%%%%%%%%%%%%%%%%%%%
\begin{frame}[plain,c]
  \frametitle{Class Website}
  \Enlarge

  \begin{center}
    \textcolor{\CSBase}{\Huge Lab \#1 this Friday!}
  \end{center}
\end{frame}

%%%%%%%%%%%%%%%%%%%%%%%%%%%%%%%%%%%%%%%%%%%%%%%%%%%%%%%%%%%%%%%%%%%%%%%%%%%%%%%%
\section{Programming}

%%%%%%%%%%%%%%%%%%%%%%%%%%%%%%%%%%%%%%%%%%%%%%%%%%%%%%%%%%%%%%%%%%%%%%%%%%%%%%%%
%\begin{frame}[fragile]
%  \frametitle{Early mathematics}

%  \begin{tabular}{cc}
%  \includegraphics[height=0.75\textheight]{./img/ybc7289.jpg}
%  \includegraphics[height=0.375\textheight]{./img/ybc7289-schematic.jpg}
%  \end{tabular}
%\end{frame}

%%%%%%%%%%%%%%%%%%%%%%%%%%%%%%%%%%%%%%%%%%%%%%%%%%%%%%%%%%%%%%%%%%%%%%%%%%%%%%%%
%\begin{frame}[fragile]
%  \frametitle{Early mathematics}

%  \begin{tabular}{cc}
%  \includegraphics[height=0.75\textheight]{./img/pentagon.png}
%  \end{tabular}
%\end{frame}

%%%%%%%%%%%%%%%%%%%%%%%%%%%%%%%%%%%%%%%%%%%%%%%%%%%%%%%%%%%%%%%%%%%%%%%%%%%%%%%%
%\begin{frame}[fragile]
%  \frametitle{Early calculation}

%  \includegraphics[height=0.25\textheight]{./img/abacus.jpg} \\
%  \includegraphics[width=0.75\textwidth]{./img/pascal.jpg}
%\end{frame}

%%%%%%%%%%%%%%%%%%%%%%%%%%%%%%%%%%%%%%%%%%%%%%%%%%%%%%%%%%%%%%%%%%%%%%%%%%%%%%%%
%\begin{frame}[fragile]
%  \frametitle{Early calculation}

%  \includegraphics[height=0.75\textheight]{./img/al-khwarizmi.png}
%\end{frame}

%%%%%%%%%%%%%%%%%%%%%%%%%%%%%%%%%%%%%%%%%%%%%%%%%%%%%%%%%%%%%%%%%%%%%%%%%%%%%%%%
%\begin{frame}[fragile]
%  \frametitle{Characteristica universalis}

%  \begin{tabular}{cc}
%  \includegraphics[height=0.75\textheight]{./img/wilkins.jpg}
%  \includegraphics[width=0.5\textwidth]{./img/llull.png}
%  \end{tabular}
%\end{frame}

%%%%%%%%%%%%%%%%%%%%%%%%%%%%%%%%%%%%%%%%%%%%%%%%%%%%%%%%%%%%%%%%%%%%%%%%%%%%%%%%
%\begin{frame}[fragile]
%  \frametitle{Modern calculation}

%  \includegraphics[height=0.75\textheight]{./img/babbage-1.jpg}
%\end{frame}

%%%%%%%%%%%%%%%%%%%%%%%%%%%%%%%%%%%%%%%%%%%%%%%%%%%%%%%%%%%%%%%%%%%%%%%%%%%%%%%%
%\begin{frame}[fragile]
%  \frametitle{Modern calculation}

%  \includegraphics[height=0.75\textheight]{./img/jacquard.jpg}
%\end{frame}

%%%%%%%%%%%%%%%%%%%%%%%%%%%%%%%%%%%%%%%%%%%%%%%%%%%%%%%%%%%%%%%%%%%%%%%%%%%%%%%%
%\begin{frame}[fragile]
%  \frametitle{Modern calculation}

%  \includegraphics[height=0.75\textheight]{./img/babbage-2.png}
%\end{frame}

%%%%%%%%%%%%%%%%%%%%%%%%%%%%%%%%%%%%%%%%%%%%%%%%%%%%%%%%%%%%%%%%%%%%%%%%%%%%%%%%
%\begin{frame}[fragile]
%  \frametitle{Modern calculation}

%  \includegraphics[height=0.75\textheight]{./img/babbage-3.jpg}
%\end{frame}

%%%%%%%%%%%%%%%%%%%%%%%%%%%%%%%%%%%%%%%%%%%%%%%%%%%%%%%%%%%%%%%%%%%%%%%%%%%%%%%%
%\begin{frame}[fragile]
%  \frametitle{Modern calculation}

%  \includegraphics[height=0.75\textheight]{./img/ibm.jpg}
  
  
%\end{frame}

%%%%%%%%%%%%%%%%%%%%%%%%%%%%%%%%%%%%%%%%%%%%%%%%%%%%%%%%%%%%%%%%%%%%%%%%%%%%%%%%
\begin{frame}[fragile]
  \frametitle{Modern calculation}
 \centering
  \includegraphics[width=0.75\textwidth]{./img/turing-et-al.jpg}
  
   \begin{itemize} 
   	\myitem \textcolor{\CSBase}{\tiny \texttt{\url{https://en.wikipedia.org/wiki/Church\%E2\%80\%93Turing_thesis}}}
   	
   \myitem \textcolor{\CSBase}{\tiny \texttt{\url{https://www.bigquestionsonline.com/2013/04/30/what-did-turing-establish-about-limits-computers-nature-mathematics/}}}
   
    \myitem \textcolor{\CSBase}{\tiny \texttt{\url{http://www.alanturing.net/turing_archive/pages/reference\%20articles/Bio\%20of\%20Alan\%20Turing.html}}}
   
   %	\myitem \textcolor{\CSBase}{\tiny \texttt{\url{http://taoxie.cs.illinois.edu/sefamily.htm}}}
   \end{itemize}
\end{frame}

%%%%%%%%%%%%%%%%%%%%%%%%%%%%%%%%%%%%%%%%%%%%%%%%%%%%%%%%%%%%%%%%%%%%%%%%%%%%%%%%
\begin{frame}[fragile]
  \frametitle{Modern calculation}

  \includegraphics[height=0.75\textheight]{./img/enigma.jpg}\\
     \textcolor{\CSBase}{\small \texttt{\url{https://en.wikipedia.org/wiki/Enigma\_machine}}}
     \textcolor{\CSBase}{\small \texttt{\url{https://en.wikipedia.org/wiki/Cryptanalysis\_of\_the\_Enigma}}}
     
\end{frame}

%%%%%%%%%%%%%%%%%%%%%%%%%%%%%%%%%%%%%%%%%%%%%%%%%%%%%%%%%%%%%%%%%%%%%%%%%%%%%%%%
\begin{frame}[fragile]
  \frametitle{Modern calculation}

  \includegraphics[height=0.75\textheight]{./img/eniac.jpg}\\
   \textcolor{\CSBase}{\small \texttt{\url{https://en.wikipedia.org/wiki/ENIAC}}}
\end{frame}

%%%%%%%%%%%%%%%%%%%%%%%%%%%%%%%%%%%%%%%%%%%%%%%%%%%%%%%%%%%%%%%%%%%%%%%%%%%%%%%%
\begin{frame}[fragile]
	\frametitle{Modern calculation}
	
	\includegraphics[height=0.75\textheight]{./img/illiac.jpg}\\
	\textcolor{\CSBase}{\small \texttt{\url{https://en.wikipedia.org/wiki/ILLIAC}}}
\end{frame}


%%%%%%%%%%%%%%%%%%%%%%%%%%%%%%%%%%%%%%%%%%%%%%%%%%%%%%%%%%%%%%%%%%%%%%%%%%%%%%%%
\begin{frame}[fragile]
  \frametitle{Algorithms}
\end{frame}

%%%%%%%%%%%%%%%%%%%%%%%%%%%%%%%%%%%%%%%%%%%%%%%%%%%%%%%%%%%%%%%%%%%%%%%%%%%%%%%%
\begin{frame}[fragile]
  \frametitle{Algorithms}

  \includegraphics[height=0.75\textheight]{./img/hedge-maze.jpg}
\end{frame}

%%%%%%%%%%%%%%%%%%%%%%%%%%%%%%%%%%%%%%%%%%%%%%%%%%%%%%%%%%%%%%%%%%%%%%%%%%%%%%%%
\begin{frame}[fragile]
  \frametitle{Computing}

  \includegraphics[width=\textwidth]{./img/assembler-1.png}
\end{frame}

%%%%%%%%%%%%%%%%%%%%%%%%%%%%%%%%%%%%%%%%%%%%%%%%%%%%%%%%%%%%%%%%%%%%%%%%%%%%%%%%
\begin{frame}[fragile]
  \frametitle{Computing}

  \includegraphics[width=\textwidth]{./img/assembler-2.png}
\end{frame}

%%%%%%%%%%%%%%%%%%%%%%%%%%%%%%%%%%%%%%%%%%%%%%%%%%%%%%%%%%%%%%%%%%%%%%%%%%%%%%%%
\begin{frame}[fragile]
  \frametitle{Computing}

  \includegraphics[width=\textwidth]{./img/assembler-3.png}
\end{frame}

%%%%%%%%%%%%%%%%%%%%%%%%%%%%%%%%%%%%%%%%%%%%%%%%%%%%%%%%%%%%%%%%%%%%%%%%%%%%%%%%
\begin{frame}[fragile]
  \frametitle{Computing}

  \includegraphics[width=\textwidth]{./img/assembler-4.png}
\end{frame}

%%%%%%%%%%%%%%%%%%%%%%%%%%%%%%%%%%%%%%%%%%%%%%%%%%%%%%%%%%%%%%%%%%%%%%%%%%%%%%%%
\begin{frame}[fragile]
  \frametitle{Algorithms}

  \begin{centering}
  \includegraphics[width=0.5\textwidth]{./img/python-logo.png}
  \end{centering}
\end{frame}

%%%%%%%%%%%%%%%%%%%%%%%%%%%%%%%%%%%%%%%%%%%%%%%%%%%%%%%%%%%%%%%%%%%%%%%%%%%%%%%%
\begin{frame}[fragile]
  \frametitle{Algorithms}
  \Enlarge
%TODO
  \begin{Verbatim}[commandchars=\\\{\}]
\textcolor{\CSAltDark}{
depth * area = volume} \pause

\textcolor{\CSAltDark}{
volume of rain / volume per raindrop}
\textcolor{\CSAltDark}{
\hspace{2cm}= number of raindrops} \pause

\textcolor{\CSPureBase}{
volume_rain = area * depth} \pause

\textcolor{\CSPureBase}{
n_raindrops = volume_rain / volume_raindrop}
  \end{Verbatim}
\end{frame}

%%%%%%%%%%%%%%%%%%%%%%%%%%%%%%%%%%%%%%%%%%%%%%%%%%%%%%%%%%%%%%%%%%%%%%%%%%%%%%%%
\begin{frame}
  \frametitle{What is a program?}
  \Enlarge

  \begin{itemize} \pause
    \myitem A set of instructions a computer executes to achieve a goal.
  \end{itemize}
\end{frame}

%%%%%%%%%%%%%%%%%%%%%%%%%%%%%%%%%%%%%%%%%%%%%%%%%%%%%%%%%%%%%%%%%%%%%%%%%%%%%%%%
\begin{frame}
  \frametitle{What is data?}
  \Enlarge

  \begin{itemize} \pause
    \myitem Information stored in a computer. \pause
    \myitem All data is stored in binary.
  \end{itemize}
  \includegraphics[width=\textwidth]{./img/assembler-2.png}
\end{frame}

%%%%%%%%%%%%%%%%%%%%%%%%%%%%%%%%%%%%%%%%%%%%%%%%%%%%%%%%%%%%%%%%%%%%%%%%%%%%%%%%
\begin{frame}
  \frametitle{What is data?}
  \Enlarge

  \begin{itemize}
    \myitem Binary data must be interpreted:
	\begin{itemize}
	  \mysubitem instruction \pause
	  \mysubitem value (number, character) \pause
	  \mysubitem memory location \pause
	\end{itemize}
  \end{itemize}
  \begin{centering}
  \includegraphics[height=0.5\textheight]{./img/dali.png}
  \end{centering}
\end{frame}

%%%%%%%%%%%%%%%%%%%%%%%%%%%%%%%%%%%%%%%%%%%%%%%%%%%%%%%%%%%%%%%%%%%%%%%%%%%%%%%%
\begin{frame}
  \frametitle{What is a program?}
  \Enlarge

  \begin{itemize} \pause
    \myitem Programs are data! \pause
	\myitem Instructions are encoded in binary.
  \end{itemize}
  \includegraphics[width=\textwidth]{./img/assembler-4.png}
\end{frame}

%%%%%%%%%%%%%%%%%%%%%%%%%%%%%%%%%%%%%%%%%%%%%%%%%%%%%%%%%%%%%%%%%%%%%%%%%%%%%%%%

%%%%%%%%%%%%%%%%%%%%%%%%%%%%%%%%%%%%%%%%%%%%%%%%%%%%%%%%%%%%%%%%%%%%%%%%%%%%%%%%
\begin{frame}
	\frametitle{Computational Thinking}
	\Enlarge
	
	\includegraphics[width=\textwidth]{./img/wing.jpg}
\end{frame}

%%%%%%%%%%%%%%%%%%%%%%%%%%%%%%%%%%%%%%%%%%%%%%%%%%%%%%%%%%%%%%%%%%%%%%%%%%%%%%%%

%%%%%%%%%%%%%%%%%%%%%%%%%%%%%%%%%%%%%%%%%%%%%%%%%%%%%%%%%%%%%%%%%%%%%%%%%%%%%%%%
\begin{frame}
	\frametitle{Engineer Joke: Engineering Thinking}
	\Enlarge
	\begin{itemize} 
		\myitem Four engineers traveling in a car an the car breaks down ...
		\myitem \textbf{Mechanical engineer}: ``Sounds to me as if the pistons have seized. We'll have to strip down the engine before we can get the car working again"
		\myitem \textbf{Chemical engineer}: ``it sounded to me as if the fuel might be contaminated. I think we should clear out the fuel system."
		\myitem \textbf{Electrical engineer}: ``I thought it might be an grounding problem or maybe a faulty plug lead."
		\myitem \textbf{Software/computer engineer}: ``Ummm perhaps if we all get out of the car and get back in again?"
	\end{itemize}
\end{frame}

%%%%%%%%%%%%%%%%%%%%%%%%%%%%%%%%%%%%%%%%%%%%%%%%%%%%%%%%%%%%%%%%%%%%%%%%%%%%%%%%


%%%%%%%%%%%%%%%%%%%%%%%%%%%%%%%%%%%%%%%%%%%%%%%%%%%%%%%%%%%%%%%%%%%%%%%%%%%%%%%%
\begin{frame}
	\frametitle{Reality in Industry: Engineering  Thinking}
	\Enlarge
	\begin{itemize} 
		\myitem Researchers working on a robot arm for assembling pens.
		\myitem They face challenges, e.g., lacking sufficient accuracy.
		\myitem Any directions for solving the problem?
	
	\end{itemize}
	 \includegraphics[width=\textwidth]{./img/robotarm.jpg}
\end{frame}

%%%%%%%%%%%%%%%%%%%%%%%%%%%%%%%%%%%%%%%%%%%%%%%%%%%%%%%%%%%%%%%%%%%%%%%%%%%%%%%%

\section{Reminders}

%%%%%%%%%%%%%%%%%%%%%%%%%%%%%%%%%%%%%%%%%%%%%%%%%%%%%%%%%%%%%%%%%%%%%%%%%%%%%%%%
\begin{frame}
  \frametitle{Reminders}
  \Enlarge

  %\begin{itemize}
 % \myitem i>clicker \\ \textcolor{\CSGradBot}{\footnotesize\hspace{1em} Grades count starting Wed 08-31}
  %\end{itemize}
  \begin{center}
    \textcolor{\CSBase}{\small \texttt{\url{https://relate.cs.illinois.edu/course/zuics101fa16/}}}
  \end{center}
\end{frame}

\end{document}

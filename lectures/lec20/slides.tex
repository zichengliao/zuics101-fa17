%!TEX program = xelatex
\documentclass[11pt]{beamer}

\usepackage{amsfonts}
\usepackage{amsmath}
\usepackage{blindtext}
\usepackage{enumitem}
\usepackage{fancyvrb}
\usepackage{tikz}

\usetheme{SaoPaulo}

\title{Numerical Python}
\subtitle{optimization}
\author{CS101 Lecture \#20}
\date{2016-11-07}

\setcounter{showSlideNumbers}{1}

\newcommand{\correctstar}{{\Large\textcolor{red}{$\star$}}}

\begin{document}
  \setcounter{showProgressBar}{0}
  \setcounter{showSlideNumbers}{0}

%%%%%%%%%%%%%%%%%%%%%%%%%%%%%%%%%%%%%%%%%%%%%%%%%%%%%%%%%%%%%%%%%%%%%%%%%%%%%%%%
\frame{\titlepage}

%%%%%%%%%%%%%%%%%%%%%%%%%%%%%%%%%%%%%%%%%%%%%%%%%%%%%%%%%%%%%%%%%%%%%%%%%%%%%%%%
\setcounter{framenumber}{0}
\setcounter{showProgressBar}{1}
\setcounter{showSlideNumbers}{1}

%%%%%%%%%%%%%%%%%%%%%%%%%%%%%%%%%%%%%%%%%%%%%%%%%%%%%%%%%%%%%%%%%%%%%%%%%%%%%%%%
\section{Administrivia}

%%%%%%%%%%%%%%%%%%%%%%%%%%%%%%%%%%%%%%%%%%%%%%%%%%%%%%%%%%%%%%%%%%%%%%%%%%%%%%%%
\begin{frame}
  \frametitle{Administrivia}
  \Enlarge

  \begin{itemize}
  \myitem  Homework \#10 is due Friday, Nov.\ 11.
  \myitem  Midterm \#2 is Monday, Nov. 14 from 7--8 p.m.
  \mysubitem \textcolor{CS101PureBase}{Practice midterm will be posted today.}
  \end{itemize}
\end{frame}

%%%%%%%%%%%%%%%%%%%%%%%%%%%%%%%%%%%%%%%%%%%%%%%%%%%%%%%%%%%%%%%%%%%%%%%%%%%%%%%%
\section{Warmup Quiz}

%%%%%%%%%%%%%%%%%%%%%%%%%%%%%%%%%%%%%%%%%%%%%%%%%%%%%%%%%%%%%%%%%%%%%%%%%%%%%%%%
\begin{frame}[fragile]
  \frametitle{Question \#1}

  \begin{Verbatim}
x = 'ABCD'
z = 'XYZ'

for a in itertools.product( x,y ):
    print( ' '.join( a ) )
  \end{Verbatim}

Which of the following is \emph{not} printed?

  \begin{enumerate}[label=\Alph*]
    \item  \texttt{'A X'}
    \item  \texttt{'B D'}
    \item  \texttt{'C X'}
    \item  \texttt{'D Z'}
  \end{enumerate}
\end{frame}

%%%%%%%%%%%%%%%%%%%%%%%%%%%%%%%%%%%%%%%%%%%%%%%%%%%%%%%%%%%%%%%%%%%%%%%%%%%%%%%%
\begin{frame}[fragile]
  \frametitle{Question \#1}

  \begin{Verbatim}
x = 'ABCD'
z = 'XYZ'

for a in itertools.product( x,y ):
    print( ' '.join( a ) )
  \end{Verbatim}

Which of the following is \emph{not} printed?

  \begin{enumerate}[label=\Alph*]
    \item  \texttt{'A X'}
    \item  \texttt{'B D'}  \correctstar
    \item  \texttt{'C X'}
    \item  \texttt{'D Z'}
  \end{enumerate}
\end{frame}

%Which of the following blocks will implement a brute-force search over the entire parameter space?

%%%%%%%%%%%%%%%%%%%%%%%%%%%%%%%%%%%%%%%%%%%%%%%%%%%%%%%%%%%%%%%%%%%%%%%%%%%%%%%%
\section{Brute-Force Search}

%%%%%%%%%%%%%%%%%%%%%%%%%%%%%%%%%%%%%%%%%%%%%%%%%%%%%%%%%%%%%%%%%%%%%%%%%%%%%%%
\begin{frame}[fragile]
  \frametitle{Brute-force search}
  \Enlarge

  \begin{enumerate}
  \myitem  Assume that a password can contain characters from the alphabet (upper- and lower-case); digits; and a selection of special characters (ampersand, dash):  86 characters.  \pause
  \end{enumerate}
  \begin{center}
  \begin{tabular}{rr}
    Characters & Search Space \\ \hline
    1 & 86 \\
    2 & $86^{2} = 7\,396$ \\ \pause
    3 & $86^{3} = 636\,056$ \\
    4 & $86^{3} = 54\,700\,816$ \\
    5 & $86^{3} = 4\,704\,270\,176$ \\ \pause
    10 & $86^{10} = 2.2 \times 10^{19}$ \\
    20 & $86^{10} = 4.9 \times 10^{38}$ \\
  \end{tabular}
  \end{center}
\end{frame}

%%%%%%%%%%%%%%%%%%%%%%%%%%%%%%%%%%%%%%%%%%%%%%%%%%%%%%%%%%%%%%%%%%%%%%%%%%%%%%%
\begin{frame}[fragile]
  \frametitle{Brute-force search}
  \Enlarge

  \begin{enumerate}
  \myitem  If Python can try a password attempt every $1 \times 10^{-7}$ s, how long does it take to crack a password of length $n$?
  \end{enumerate}
  \begin{center}
  \begin{tabular}{rrl}
    Characters & Search Space & Time\\ \hline
    1 & 86                    & $8.6 \times 10^{-6} \,\text{s}$ \\
    2 & $7\,396$              & $7.4 \times 10^{-4} \,\text{s}$ \\
    3 & $636\,056$            & $6.4 \times 10^{-2} \,\text{s}$ \\
    4 & $54\,700\,816$        & $5.4 \,\text{s}$ \\
    5 & $4\,704\,270\,176$    & $470.4 \,\text{s}$ \\ \pause
    10 & $2.2 \times 10^{19}$ & $1.9 \times 10^{14} \,\text{s} = 6 \times 10^{6} \,\text{a}$ \\ \pause
    20 & $4.9 \times 10^{38}$ & $4.9 \times 10^{31} \,\text{s}$ \\
  \end{tabular}
  \end{center}
\end{frame}

%%%%%%%%%%%%%%%%%%%%%%%%%%%%%%%%%%%%%%%%%%%%%%%%%%%%%%%%%%%%%%%%%%%%%%%%%%%%%%%%
\section{Heuristic Optimization}

%%%%%%%%%%%%%%%%%%%%%%%%%%%%%%%%%%%%%%%%%%%%%%%%%%%%%%%%%%%%%%%%%%%%%%%%%%%%%%%%
\begin{frame}[fragile]
  \frametitle{Heuristic optimization}
  \Enlarge

  \begin{enumerate}
  \myitem  In many cases, a ``good-enough'' solution is fine.  \pause
  \myitem  If we have a figure of \emph{relative} merit, we can classify candidate solutions by how good they are.  \pause
  \myitem  Heuristic algorithms don't guarantee the `best' solution, but are often adequate (and the only choice!).
  \end{enumerate}
\end{frame}

%%%%%%%%%%%%%%%%%%%%%%%%%%%%%%%%%%%%%%%%%%%%%%%%%%%%%%%%%%%%%%%%%%%%%%%%%%%%%%%%
\begin{frame}[fragile]
  \frametitle{Hill-climbing algorithm}
  \Enlarge

  \begin{enumerate}
  \myitem  {\textbf Strategy}:  Always selecting neighboring candidate solution which improves on this one. \pause
  \myitem  {\textbf Analogy}:  Trying to find the highest hill by only taking a step uphill from where you are. \pause
  \myitem  {\textbf Pitfall}:  Finding a \emph{local} optimum instead of the global optimum.
  \end{enumerate}
\end{frame}

%%%%%%%%%%%%%%%%%%%%%%%%%%%%%%%%%%%%%%%%%%%%%%%%%%%%%%%%%%%%%%%%%%%%%%%%%%%%%%%%
\begin{frame}[fragile]
  \frametitle{Steepest ascent algorithm}
  \Enlarge

  \begin{enumerate}
  \myitem  {\textbf Strategy}:  Tweaking our current solution by changing all elements to improve the result.  Picking the candidate solution with the greatest improvement. \pause
  \myitem  {\textbf Analogy}:  Trying to find the highest hill by always taking the \emph{steepest} step uphill from where you are. \pause
  \myitem  {\textbf Pitfall}:  Finding a \emph{local} optimum instead of the global optimum.
  \end{enumerate}
\end{frame}

%%%%%%%%%%%%%%%%%%%%%%%%%%%%%%%%%%%%%%%%%%%%%%%%%%%%%%%%%%%%%%%%%%%%%%%%%%%%%%%%
\begin{frame}[fragile]
  \frametitle{Random sampling}
  \Enlarge

  \begin{enumerate}
  \myitem  {\textbf Strategy}:  Choosing at random a candidate solution (sometimes within a constrained space). \pause
  \myitem  {\textbf Analogy}:  Picking random heights in the region of a hill, accepting the tallest as the highest. \pause
  \myitem  {\textbf Pitfall}:  Without good constraints, missing the optimum value.
  \end{enumerate}
\end{frame}

%%%%%%%%%%%%%%%%%%%%%%%%%%%%%%%%%%%%%%%%%%%%%%%%%%%%%%%%%%%%%%%%%%%%%%%%%%%%%%%%
\begin{frame}[fragile]
  \frametitle{Random walk}
  \Enlarge

  \begin{enumerate}
  \myitem  {\textbf Strategy}:  Tweaking the current candidate solution at random, and possibly rejecting the solution if worse. \pause
  \myitem  {\textbf Analogy}:  Taking random steps near a hill, but maybe not taking the step if it's worse. \pause
  \myitem  {\textbf Pitfall}:  Converging slowly, can still miss best candidate solution.  BUT:  has a way from getting stuck in local optima.
  \end{enumerate}
\end{frame}

%%%%%%%%%%%%%%%%%%%%%%%%%%%%%%%%%%%%%%%%%%%%%%%%%%%%%%%%%%%%%%%%%%%%%%%%%%%%%%%%
\begin{frame}[fragile]
  \frametitle{Example}
  \Enlarge

  \begin{enumerate}
  \myitem  We require:
    \begin{enumerate}
    \mysubitem  A problem with relative solution assessment
    \mysubitem  An algorithm to assess solutions
    \end{enumerate}
  \myitem  The password cracking didn't have the former.
  \myitem  Let's revisit the bag-packing algorithm.
  \end{enumerate}
\end{frame}

%%%%%%%%%%%%%%%%%%%%%%%%%%%%%%%%%%%%%%%%%%%%%%%%%%%%%%%%%%%%%%%%%%%%%%%%%%%%%%%%
\begin{frame}[fragile]
  \frametitle{Example}
  \Enlarge

  \begin{enumerate}
  \myitem  Our comparative strategies:
    \begin{enumerate}
    \mysubitem  Brute-force (last lecture)
    \mysubitem  Hill-climbing \pause
      \begin{enumerate}
      \mysubitem  Select heaviest item, then add next heaviest, etc. \pause
      \mysubitem  Select most valuable item, then add next most valuable item, etc. \pause
      \end{enumerate}
    \mysubitem  Random sampling \pause
    \mysubitem  Random walk:  sample randomly, then iteratively allow change
    \end{enumerate}
  \end{enumerate}
\end{frame}

%%%%%%%%%%%%%%%%%%%%%%%%%%%%%%%%%%%%%%%%%%%%%%%%%%%%%%%%%%%%%%%%%%%%%%%%%%%%%%%%
\begin{frame}[fragile]
  \frametitle{Setup}
  \Enlarge

  \begin{Verbatim}
import numpy as np
import matplotlib.pyplot as plt
import itertools

n = 10
items   = list( range( n ) )
weights = np.random.uniform( size=(n,) ) * 50
values  = np.random.uniform( size=(n,) ) * 100
  \end{Verbatim}
\end{frame}

%%%%%%%%%%%%%%%%%%%%%%%%%%%%%%%%%%%%%%%%%%%%%%%%%%%%%%%%%%%%%%%%%%%%%%%%%%%%%%%%
\begin{frame}[fragile]
  \frametitle{Setup}

  \begin{Verbatim}
def f( wts, vals ):
    total_weight = 0
    total_value = 0

    for i in range( len( wts ) ):
        total_weight += wts[ i ]
        total_value  += vals[ i ]

    if total_weight >= 50:
        return 0
    else:
        return total_value
  \end{Verbatim}
\end{frame}

%%%%%%%%%%%%%%%%%%%%%%%%%%%%%%%%%%%%%%%%%%%%%%%%%%%%%%%%%%%%%%%%%%%%%%%%%%%%%%%%
\begin{frame}[fragile]
  \frametitle{Tracking cases}

  \begin{Verbatim}
max_value = 0.0
max_set = None
lists = []
for i in range(n):
    for set in itertools.combinations( items,i ):
        wts  = []
        vals = []
        for item in set:
            wts.append( weights[ item ] )
            vals.append( values[ item ] )
        value = f( wts,vals )
        lists.append( ( wts, value ) )
        if value > 0:
            print( value, wts )
        if value > max_value:
            max_value = value
            max_set = set
  \end{Verbatim}
\end{frame}

%%%%%%%%%%%%%%%%%%%%%%%%%%%%%%%%%%%%%%%%%%%%%%%%%%%%%%%%%%%%%%%%%%%%%%%%%%%%%%%%
\begin{frame}[fragile]
  \frametitle{Tracking cases}

  \begin{Verbatim}
array = np.array( lists )
plt.plot( array[:,1], 'b.' )
plt.xlim( ( 0, len(lists) ) )
plt.show()
  \end{Verbatim}
\end{frame}

%%%%%%%%%%%%%%%%%%%%%%%%%%%%%%%%%%%%%%%%%%%%%%%%%%%%%%%%%%%%%%%%%%%%%%%%%%%%%%%%
\begin{frame}[fragile]
  \frametitle{Brute-force search}

  \begin{Verbatim}
import itertools

max_value = 0.0
max_set = None
for i in range(n):
    for set in itertools.combinations( items,i ):
        wts  = []
        vals = []
        for item in set:
            wts.append( weights[ item ] )
            vals.append( values[ item ] )
        value = f( wts,vals )
        if value > max_value:
            max_value = value
            max_set = set
  \end{Verbatim}
\end{frame}

%%%%%%%%%%%%%%%%%%%%%%%%%%%%%%%%%%%%%%%%%%%%%%%%%%%%%%%%%%%%%%%%%%%%%%%%%%%%%%%%
\begin{frame}[fragile]
  \frametitle{Hill-climbing search}

  \begin{Verbatim}
max_wt = 50.0

wts_orig  = wts[ : ]
vals_orig = vals[ : ]

best_vals = [ ]
best_wts  = [ ]
best_vals.append( max( vals ) )
best_wts.append( wts[ vals.index( max( vals ) ) ] )
wts.remove( wts[ vals.index( max( vals ) ) ] )
vals.remove( max( vals ) )
  \end{Verbatim}
\end{frame}

%%%%%%%%%%%%%%%%%%%%%%%%%%%%%%%%%%%%%%%%%%%%%%%%%%%%%%%%%%%%%%%%%%%%%%%%%%%%%%%%
\begin{frame}[fragile]
  \frametitle{Hill-climbing search}

  \begin{Verbatim}
while sum( best_wts ) + wts[ vals.index( max( vals ) ) ] \
        < max_wt:
    best_vals.append( max( vals ) )
    best_wts.append( wts[ vals.index( max( vals ) ) ] )
    wts.remove( wts[ vals.index( max( vals ) ) ] )
    vals.remove( max( vals ) )

wts  = wts_orig[ : ]
vals = vals_orig[ : ]
  \end{Verbatim}
\end{frame}

%%%%%%%%%%%%%%%%%%%%%%%%%%%%%%%%%%%%%%%%%%%%%%%%%%%%%%%%%%%%%%%%%%%%%%%%%%%%%%%%
\begin{frame}[fragile]
  \frametitle{Random walk}

  \begin{Verbatim}
# try a configuration at random
# alter it at random with small likelihood of getting worse
for t in range( 1000 ):
    # two possible moves:  adding or removing
    if f( next_wts,next_vals ) > f( trial_wts,trial_vals ):
        # if improvement, accept the change
    else:
        # if no improvement, *maybe* accept the change
    # if all-time best, track it
# (see random-walk.py)
  \end{Verbatim}
\end{frame}

%%%%%%%%%%%%%%%%%%%%%%%%%%%%%%%%%%%%%%%%%%%%%%%%%%%%%%%%%%%%%%%%%%%%%%%%%%%%%%%%
\section{Code Performance}

%%%%%%%%%%%%%%%%%%%%%%%%%%%%%%%%%%%%%%%%%%%%%%%%%%%%%%%%%%%%%%%%%%%%%%%%%%%%%%%%
\begin{frame}[fragile]
  \frametitle{Code performance}
  \Enlarge

  \begin{enumerate}
  \myitem  In order to compare algorithms, we need a way to measure code run time (called ``wallclock time''). \pause
  \myitem  The \texttt{timeit} module provides three ways to time your code: \pause
    \begin{enumerate}
    \mysubitem  Interpreter:  \texttt{timeit.timeit('func( n )', number=10000)} \pause
    \mysubitem  Command line:  \texttt{python3 -m timeit 'code'} \pause
    \mysubitem  Notebook:  \texttt{\%timeit func( n )}  (this is easiest) \pause
    \end{enumerate}
  \myitem  These run your code many times and return an average time to completion.
  \end{enumerate}
\end{frame}

%%%%%%%%%%%%%%%%%%%%%%%%%%%%%%%%%%%%%%%%%%%%%%%%%%%%%%%%%%%%%%%%%%%%%%%%%%%%%%%%
\begin{frame}[fragile]
  \frametitle{Fibonacci sequence}
  \Enlarge

  $$
  F_{n} = F_{n-1} + F_{n-2} \hspace{2cm} F_{1} = F_{2} = 1
  $$

  $$
  1\,1\,2\,3\,5\,8\,13\,21\,34\,55\,...
  $$

  \pause

  $$
  F_{n} = \frac{\left( \frac{1+\sqrt{5}}{2} \right)^{n} + \left( \frac{2}{1+\sqrt{5}} \right)^{n}}{\sqrt{5} + \frac{1}{2}}
  $$
\end{frame}

%%%%%%%%%%%%%%%%%%%%%%%%%%%%%%%%%%%%%%%%%%%%%%%%%%%%%%%%%%%%%%%%%%%%%%%%%%%%%%%%
\begin{frame}[fragile]
  \frametitle{Analytical Fibonacci}
  \Enlarge

  \begin{Verbatim}
def fib_a( n ):
    sqrt_5 = 5**0.5;
    p = ( 1 + sqrt_5 ) / 2;
    q = 1 / p;
    return int( (p**n + q**n) / sqrt_5 + 0.5 )
  \end{Verbatim}
\end{frame}

%%%%%%%%%%%%%%%%%%%%%%%%%%%%%%%%%%%%%%%%%%%%%%%%%%%%%%%%%%%%%%%%%%%%%%%%%%%%%%%%
\begin{frame}[fragile]
  \frametitle{Recursive Fibonacci}
  \Enlarge

  \begin{Verbatim}
def fib_r( n ):
    if n == 1 or n == 2:
        return 1
    else:
        return fib_r( n-1 ) + fib_r( n-2 )
  \end{Verbatim}
\end{frame}

%%%%%%%%%%%%%%%%%%%%%%%%%%%%%%%%%%%%%%%%%%%%%%%%%%%%%%%%%%%%%%%%%%%%%%%%%%%%%%%%
\begin{frame}[fragile]
  \frametitle{Comparison}
  \Enlarge

  \begin{Verbatim}
%timeit fib_a( 12 )
%timeit fib_r( 12 )
  \end{Verbatim}
  \pause
  \begin{enumerate}
  \myitem  On my machine, \texttt{fib\_a} is 55 $\times$ faster than \texttt{fib\_r} for \texttt{n = 12}.  (Will this performance get better or worse for larger \texttt{n}?)
  \end{enumerate}
\end{frame}

%%%%%%%%%%%%%%%%%%%%%%%%%%%%%%%%%%%%%%%%%%%%%%%%%%%%%%%%%%%%%%%%%%%%%%%%%%%%%%%%
\section{Comparing Results}

%%%%%%%%%%%%%%%%%%%%%%%%%%%%%%%%%%%%%%%%%%%%%%%%%%%%%%%%%%%%%%%%%%%%%%%%%%%%%%%%
\begin{frame}[fragile]
  \frametitle{Comparing results}
  \Enlarge

  \begin{enumerate}
  \myitem  \texttt{array}s don't play nicely with comparisons:
  \begin{Verbatim}
one = np.ones( ( 5, ) )
if one == 1:
    print( 'setup correct' )
  \end{Verbatim}
  \pause
  \begin{Verbatim}[commandchars=\\\{\},commentchar=\%]
\textcolor{red}{ValueError}: The truth value of an array
    with more than one element is ambiguous.
  \end{Verbatim}
  \pause
  \myitem  Which element is compared?  It's ambiguous.
  \end{enumerate}
\end{frame}

%%%%%%%%%%%%%%%%%%%%%%%%%%%%%%%%%%%%%%%%%%%%%%%%%%%%%%%%%%%%%%%%%%%%%%%%%%%%%%%%
\begin{frame}[fragile]
  \frametitle{Comparing results}
  \Enlarge

  \begin{enumerate}
  \myitem  \texttt{array}s have the built-in methods \texttt{any} and \texttt{all}:
  \begin{Verbatim}
one = np.ones( ( 5, ) )
if one.all() == 1:
    print( 'setup correct' )
  \end{Verbatim}
  \pause
  \begin{Verbatim}[commandchars=\\\{\},commentchar=\%]
domain = np.linspace( 0,10,11 )
if domain.any() == 1:
    print( 'setup contains one' )
  \end{Verbatim}
  \end{enumerate}
\end{frame}

\end{document}

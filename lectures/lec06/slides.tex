%!TEX program = xelatex
\documentclass[11pt]{beamer}

\usepackage{amsfonts}
\usepackage{amsmath}
\usepackage{blindtext}
\usepackage{enumitem}
\usepackage{fancyvrb}

\usetheme{SaoPaulo}

\title{Python Basics!}
\subtitle{scripting, logic, control}
\author{CS101 Lecture \#6}
\date{2016-10-17}

\setcounter{showSlideNumbers}{1}

\begin{document}
  \setcounter{showProgressBar}{0}
  \setcounter{showSlideNumbers}{0}

%%%%%%%%%%%%%%%%%%%%%%%%%%%%%%%%%%%%%%%%%%%%%%%%%%%%%%%%%%%%%%%%%%%%%%%%%%%%%%%%
\frame{\titlepage}

%%%%%%%%%%%%%%%%%%%%%%%%%%%%%%%%%%%%%%%%%%%%%%%%%%%%%%%%%%%%%%%%%%%%%%%%%%%%%%%%
\setcounter{framenumber}{0}
\setcounter{showProgressBar}{1}
\setcounter{showSlideNumbers}{1}

%%%%%%%%%%%%%%%%%%%%%%%%%%%%%%%%%%%%%%%%%%%%%%%%%%%%%%%%%%%%%%%%%%%%%%%%%%%%%%%%
\section{Administrivia}

%%%%%%%%%%%%%%%%%%%%%%%%%%%%%%%%%%%%%%%%%%%%%%%%%%%%%%%%%%%%%%%%%%%%%%%%%%%%%%%%
\begin{frame}
  \frametitle{Administrivia}
  \Enlarge
  \begin{itemize}
  \myitem  Homework \#2 is due Wed Oct.\ 19.
  \myitem  Homework \#3 is due Wed Oct.\ 26.
  \end{itemize}
\end{frame}

%%%%%%%%%%%%%%%%%%%%%%%%%%%%%%%%%%%%%%%%%%%%%%%%%%%%%%%%%%%%%%%%%%%%%%%%%%%%%%%%
\section{Warmup Question (No Quiz this lecture!)}

%%%%%%%%%%%%%%%%%%%%%%%%%%%%%%%%%%%%%%%%%%%%%%%%%%%%%%%%%%%%%%%%%%%%%%%%%%%%%%%%
\begin{frame}[fragile]
	\frametitle{Question \#1}
	\Enlarge
	
	\begin{Verbatim}[commandchars=\\\{\},commentchar=\%]
    s = "74.125.21.147"
    i = s.find( "." )
    x = s[i+1:i+3]
    x = x * 2
	\end{Verbatim}
	What is the value of \texttt{x}?
	\begin{enumerate}[label=\Alph*]
		\item  \texttt{"125125"}
		\item  \texttt{250}
		\item  \texttt{"1212"} % $\star$
		\item  \texttt{24}
	\end{enumerate}
\end{frame}


%%%%%%%%%%%%%%%%%%%%%%%%%%%%%%%%%%%%%%%%%%%%%%%%%%%%%%%%%%%%%%%%%%%%%%%%%%%%%%%%
\section{Composing Functions}

%%%%%%%%%%%%%%%%%%%%%%%%%%%%%%%%%%%%%%%%%%%%%%%%%%%%%%%%%%%%%%%%%%%%%%%%%%%%%%%%
\begin{frame}[fragile]
	\frametitle{Example:  Defining functions}
	\Enlarge
	
 \begin{Verbatim}[commandchars=\\\{\},commentchar=\%]
def pow(a, b):
    y = a ** b
    return y
\end{Verbatim}
\end{frame}

%%%%%%%%%%%%%%%%%%%%%%%%%%%%%%%%%%%%%%%%%%%%%%%%%%%%%%%%%%%%%%%%%%%%%%%%%%%%%%%%
\begin{frame}[fragile]
	\frametitle{Defining functions}
	\Enlarge
	
	\begin{itemize}
		\myitem  We define a function with the following:
		\begin{itemize}
			\mysubitem  the keyword \texttt{def}
			\mysubitem  the name of the function
			\mysubitem  a pair of parentheses
			\mysubitem  a \textbf{block} of code
		\end{itemize}
	\end{itemize}
\end{frame}

%%%%%%%%%%%%%%%%%%%%%%%%%%%%%%%%%%%%%%%%%%%%%%%%%%%%%%%%%%%%%%%%%%%%%%%%%%%%%%%%
\begin{frame}[fragile]
	\frametitle{Example:  Defining functions}
	\Enlarge
	
	\begin{columns}
		\begin{column}{0.75\textwidth}
			\begin{Verbatim}[commandchars=\\\{\},commentchar=\%]
def greetings():
   print("Bom dia!")
   print("Bonjour!")
   print("Ni hao!")
   print("Hello!")
   print("Shalom!")
   print("Guten tag!")
   print("Konichiwa!")
   print("As-salamu alaykum!")
			\end{Verbatim}
		\end{column}
		\begin{column}{0.25\textwidth}
			header \\ ~ \\ ~ \\ ~ \\ ~ body \\ ~ \\ ~ \\ ~ \\ ~ \\
		\end{column}
	\end{columns}
\end{frame}

%%%%%%%%%%%%%%%%%%%%%%%%%%%%%%%%%%%%%%%%%%%%%%%%%%%%%%%%%%%%%%%%%%%%%%%%%%%%%%%%
\begin{frame}[fragile]
  \frametitle{Block}
  \Enlarge

  \begin{itemize}
  \myitem  A section of code grouped together. %\pause
  \myitem  Begins with a \texttt{:}. %\pause
  \myitem  Contents of the block are indented:
  \end{itemize}
  \begin{Verbatim}[commandchars=\\\{\},commentchar=\%]
def hello():
    print('hello')
  \end{Verbatim}
\end{frame}

%%%%%%%%%%%%%%%%%%%%%%%%%%%%%%%%%%%%%%%%%%%%%%%%%%%%%%%%%%%%%%%%%%%%%%%%%%%%%%%%
\begin{frame}
  \frametitle{Scope}
  \Enlarge

  \begin{itemize}
  \myitem  Variables defined inside of a block are \emph{independent} of variables outside of the block. %\pause
  \myitem  Variables inside a block \emph{do not exist} outside of the block. %\pause
  \myitem  Blocks are isolated from the rest of the code!
  \end{itemize}
\end{frame}

%%%%%%%%%%%%%%%%%%%%%%%%%%%%%%%%%%%%%%%%%%%%%%%%%%%%%%%%%%%%%%%%%%%%%%%%%%%%%%%%
\begin{frame}[fragile]
  \frametitle{Example:  Defining functions}
  \Enlarge

  \begin{Verbatim}[commandchars=\\\{\},commentchar=\%]
\textcolor{CS101AltDark}{
a = 5}
\textcolor{CS101AltDark}{
def fun():}
\textcolor{CS101PureBase}{
    a = 3}
\textcolor{CS101PureBase}{
    b = 4}
\textcolor{CS101PureBase}{
    a = a + b } 
\textcolor{CS101AltDark}{fun()}
\textcolor{CS101AltDark}{print(a)}
  \end{Verbatim}
\end{frame}

%%%%%%%%%%%%%%%%%%%%%%%%%%%%%%%%%%%%%%%%%%%%%%%%%%%%%%%%%%%%%%%%%%%%%%%%%%%%%%%%
\begin{frame}[fragile]
  \frametitle{\texttt{return}}
  \Enlarge

  \begin{itemize}
  \myitem  Functions can return values with the keyword \texttt{return}. %\pause
    \begin{Verbatim}[commandchars=\\\{\},commentchar=\%]
def three():
    return 3
    \end{Verbatim}
  \myitem  %\pause \texttt{return} immediately exits a function. %\pause
    \begin{Verbatim}[commandchars=\\\{\},commentchar=\%]
def zero():
  return 0
  print('0')
    \end{Verbatim}
   \myitem Does the above code face an error?\\
   \myitme Does the print statement take effect if invoking the function? 
  \end{itemize}
\end{frame}

%%%%%%%%%%%%%%%%%%%%%%%%%%%%%%%%%%%%%%%%%%%%%%%%%%%%%%%%%%%%%%%%%%%%%%%%%%%%%%%%
\begin{frame}[fragile]
  \frametitle{Parameters}
  \Enlarge

  \begin{itemize}
  \myitem  Functions can accept values as parameters (input, arguments). %\pause
  \myitem  These variables are declared in the function header. %\pause
  \myitem  Multiple parameters are separated by commas. %\pause
  \end{itemize}
  \begin{Verbatim}[commandchars=\\\{\},commentchar=\%]
def print_message( msg ):
    print( msg )
  \end{Verbatim}
\end{frame}

%%%%%%%%%%%%%%%%%%%%%%%%%%%%%%%%%%%%%%%%%%%%%%%%%%%%%%%%%%%%%%%%%%%%%%%%%%%%%%%%
\begin{frame}[fragile]
  \frametitle{Example}
  \Enlarge

  \begin{Verbatim}[commandchars=\\\{\},commentchar=\%]
def fun(a):
    return a+2

x = fun(2) * fun(3)
  \end{Verbatim}
  What is the value of \texttt{x}?
  \begin{enumerate}[label=\Alph*]
  \item  \texttt{6}
  \item  \texttt{8}
  \item  \texttt{24}
  \item  None of the above.   % $\star$
  \end{enumerate}
\end{frame}

%%%%%%%%%%%%%%%%%%%%%%%%%%%%%%%%%%%%%%%%%%%%%%%%%%%%%%%%%%%%%%%%%%%%%%%%%%%%%%%%
\begin{frame}[fragile]
  \frametitle{Example}
  \Enlarge

  \begin{Verbatim}[commandchars=\\\{\},commentchar=\%]
def fun(m):
    return m.title().swapcase()

x = fun( "abb") + fun( "acab" )
  \end{Verbatim}
  What is the value of \texttt{x}?
  \begin{enumerate}[label=\Alph*]
  \item  \texttt{'AbbAcab'}
  \item  \texttt{'aBBaCAB'} % $\star$  
  \item  \texttt{'abbacab'}
  \item  \texttt{'ABBACAB'}
  \end{enumerate}
\end{frame}

%%%%%%%%%%%%%%%%%%%%%%%%%%%%%%%%%%%%%%%%%%%%%%%%%%%%%%%%%%%%%%%%%%%%%%%%%%%%%%%%
\begin{frame}[fragile]
  \frametitle{Example}
  \Enlarge

  \begin{Verbatim}[commandchars=\\\{\},commentchar=\%]
def fun(a,b):
    c = ((a + ' ') * len(b)).title()

x = fun( "ab", "caa" )
  \end{Verbatim}
  What is the value of \texttt{x}?
  \begin{enumerate}[label=\Alph*]
  \item  \texttt{'ab ab ab '}
  \item  \texttt{'Ab Ab Ab '}
  \item  \texttt{'AB AB AB '}
  \item  None of the above.   % $\star$
  \end{enumerate}
\end{frame}

%%%%%%%%%%%%%%%%%%%%%%%%%%%%%%%%%%%%%%%%%%%%%%%%%%%%%%%%%%%%%%%%%%%%%%%%%%%%%%%%
\begin{frame}[fragile]
  \frametitle{Example}
  \Enlarge

  \begin{Verbatim}[commandchars=\\\{\},commentchar=\%]
def fun(a,b):
    c = ((a + ' ') * len(b)).title()
    return c

x = fun( "ab", "caa" )
  \end{Verbatim}
  What is the value of \texttt{x}?
  \begin{enumerate}[label=\Alph*]
  \item  \texttt{'ab ab ab '}
  \item  \texttt{'Ab Ab Ab '} % $\star$
  \item  \texttt{'AB AB AB '}
  \end{enumerate}
\end{frame}

%%%%%%%%%%%%%%%%%%%%%%%%%%%%%%%%%%%%%%%%%%%%%%%%%%%%%%%%%%%%%%%%%%%%%%%%%%%%%%%%
\section{Boolean Logic}

%%%%%%%%%%%%%%%%%%%%%%%%%%%%%%%%%%%%%%%%%%%%%%%%%%%%%%%%%%%%%%%%%%%%%%%%%%%%%%%%
\begin{frame}[fragile]
  \frametitle{Boolean}
  \Enlarge

  \begin{itemize}
  \myitem  \texttt{bool} is a type with two possible values:
    \begin{itemize}
    \mysubitem  \texttt{True}
    \mysubitem  \texttt{False}
    \end{itemize} %\pause
  \myitem  We use these to make decisions.
  \myitem  Their logic is based on Boolean algebra. %\pause
  \myitem  Operators:
    \begin{itemize}
    \mysubitem  \texttt{and}
    \mysubitem  \texttt{or}
    \mysubitem  \texttt{not}
    \end{itemize}
  \end{itemize}
\end{frame}

%%%%%%%%%%%%%%%%%%%%%%%%%%%%%%%%%%%%%%%%%%%%%%%%%%%%%%%%%%%%%%%%%%%%%%%%%%%%%%%%
\begin{frame}[fragile]
  \frametitle{Example:  Boolean logic}
  \Enlarge

  \begin{center}
  $$
  0 < x \leq 10
  $$
  \\
  \texttt{(x > 0) and (x <= 10)}
  \end{center}
\end{frame}

%%%%%%%%%%%%%%%%%%%%%%%%%%%%%%%%%%%%%%%%%%%%%%%%%%%%%%%%%%%%%%%%%%%%%%%%%%%%%%%%
\begin{frame}[fragile]
  \frametitle{Boolean operators}
  \Enlarge

  \begin{columns}
  \begin{column}{0.6\textwidth}
  \begin{tabular}{lll}
  \texttt{and}   & \texttt{True}  & \texttt{False} \\
  \texttt{True}  & \texttt{True}  & \texttt{False} \\
  \texttt{False} & \texttt{False} & \texttt{False} \\
  \end{tabular}
  \newline \newline \newline
  \textcolor{CS101GradBot}{
  \begin{tabular}{lll}
  \texttt{or}    & \texttt{True}  & \texttt{False} \\
  \texttt{True}  & \texttt{True}  & \texttt{True}  \\
  \texttt{False} & \texttt{True}  & \texttt{False} \\
  \end{tabular}
  }
  \end{column}
  \begin{column}{0.4\textwidth}
    True when BOTH inputs are true
    \newline \newline \newline
    \textcolor{CS101GradBot}{
    True when EITHER input is true
    }
  \end{column}
  \end{columns}
\end{frame}

%%%%%%%%%%%%%%%%%%%%%%%%%%%%%%%%%%%%%%%%%%%%%%%%%%%%%%%%%%%%%%%%%%%%%%%%%%%%%%%%
\begin{frame}[fragile]
  \frametitle{Boolean operators}
  \Enlarge

  \begin{columns}
  \begin{column}{0.6\textwidth}
  \begin{tabular}{lll}
  \texttt{not}   & Result         \\
  \texttt{True}  & \texttt{False} \\
  \texttt{False} & \texttt{True}  \\
  \end{tabular}
  \end{column}
  \begin{column}{0.4\textwidth}
    Inverts truth-value
  \end{column}
  \end{columns}
\end{frame}

%%%%%%%%%%%%%%%%%%%%%%%%%%%%%%%%%%%%%%%%%%%%%%%%%%%%%%%%%%%%%%%%%%%%%%%%%%%%%%%%
\begin{frame}[fragile]
  \frametitle{Example}
  \Enlarge

  \begin{Verbatim}[commandchars=\\\{\},commentchar=\%]
def fun():
    return True and False

x = fun() and not (True or False)
  \end{Verbatim}
  What is the value of \texttt{x}?
  \begin{enumerate}[label=\Alph*]
  \item  \texttt{True}
  \item  \texttt{False}  %$\star$
  \end{enumerate}
\end{frame}

%%%%%%%%%%%%%%%%%%%%%%%%%%%%%%%%%%%%%%%%%%%%%%%%%%%%%%%%%%%%%%%%%%%%%%%%%%%%%%%%
\begin{frame}[fragile]
  \frametitle{Comparison operators}
  \Enlarge

  \begin{itemize}
  \myitem  These produce Boolean output.
    \begin{itemize}
    \mysubitem  less than, \texttt{<}
    \mysubitem  greater than, \texttt{>}
    \mysubitem  less than or equal to, \texttt{<=}
    \mysubitem  greater than or equal to, \texttt{>=}
    \mysubitem  \textcolor{red}{equal to, \texttt{==}}
    \mysubitem  not equal to, \texttt{!=}
    \end{itemize}
  \end{itemize}
\end{frame}

%%%%%%%%%%%%%%%%%%%%%%%%%%%%%%%%%%%%%%%%%%%%%%%%%%%%%%%%%%%%%%%%%%%%%%%%%%%%%%%%
\begin{frame}[fragile]
  \frametitle{Example}
  \Enlarge

  \begin{Verbatim}[commandchars=\\\{\},commentchar=\%]
a = 5
b = 3

x = (a < 5) and ((b <= 5) or (a != b))
  \end{Verbatim}
  What is the value of \texttt{x}?
  \begin{enumerate}[label=\Alph*]
  \item  \texttt{True}
  \item  \texttt{False} % $\star$
  \end{enumerate}
\end{frame}

%%%%%%%%%%%%%%%%%%%%%%%%%%%%%%%%%%%%%%%%%%%%%%%%%%%%%%%%%%%%%%%%%%%%%%%%%%%%%%%%
\begin{frame}[fragile]
  \frametitle{Example}
  \Enlarge

  \begin{Verbatim}[commandchars=\\\{\},commentchar=\%]
a = 'URSA MAJOR'
b = 'GEMINI'

x = a < b and a[1] != b[-2]
  \end{Verbatim}
  What is the value of \texttt{x}?
  \begin{enumerate}[label=\Alph*]
  \item  \texttt{True}
  \item  \texttt{False}  % $\star$
  \end{enumerate}
\end{frame}

%%%%%%%%%%%%%%%%%%%%%%%%%%%%%%%%%%%%%%%%%%%%%%%%%%%%%%%%%%%%%%%%%%%%%%%%%%%%%%%%
\begin{frame}[fragile]
  \frametitle{Example}
  \Enlarge

  \begin{Verbatim}[commandchars=\\\{\},commentchar=\%]
def fun(a,b):
    return a<b
a = 3
b = 4
x = fun(b,a)
  \end{Verbatim}
  What is the value of \texttt{x}?
  \begin{enumerate}[label=\Alph*]
  \item  \texttt{True}
  \item  \texttt{False} % $\star$
  \end{enumerate}
\end{frame}

%%%%%%%%%%%%%%%%%%%%%%%%%%%%%%%%%%%%%%%%%%%%%%%%%%%%%%%%%%%%%%%%%%%%%%%%%%%%%%%%
\section{Conditional Execution}

%%%%%%%%%%%%%%%%%%%%%%%%%%%%%%%%%%%%%%%%%%%%%%%%%%%%%%%%%%%%%%%%%%%%%%%%%%%%%%%%
\begin{frame}[fragile]
  \frametitle{Control flow}
  \Enlarge

  \begin{itemize}
  \myitem  \emph{Control flow} represents actual sequence of lines executed by processor. %\pause
  \myitem  \emph{Conditional execution} lets you execute (or not) a block of code based on logical comparison.
  \end{itemize}
\end{frame}

%%%%%%%%%%%%%%%%%%%%%%%%%%%%%%%%%%%%%%%%%%%%%%%%%%%%%%%%%%%%%%%%%%%%%%%%%%%%%%%%
\begin{frame}[fragile]
  \frametitle{Example:  \texttt{if} statement}
  \Enlarge

  \begin{Verbatim}[commandchars=\\\{\},commentchar=\%]
ans = input( "Enter a number:" )
if float(ans) < 0:
    print( "The number was negative." )
  \end{Verbatim}
\end{frame}

%%%%%%%%%%%%%%%%%%%%%%%%%%%%%%%%%%%%%%%%%%%%%%%%%%%%%%%%%%%%%%%%%%%%%%%%%%%%%%%%
\begin{frame}[fragile]
  \frametitle{\texttt{if} statement}
  \Enlarge

  \begin{itemize}
  \myitem  We create an \texttt{if} statement as follows:
    \begin{itemize}
    \mysubitem  the keyword \texttt{if}
    \mysubitem  a logical comparison (results in \texttt{bool})
    \mysubitem  a \textbf{block} of code
    \end{itemize}
  \end{itemize}
\end{frame}

%%%%%%%%%%%%%%%%%%%%%%%%%%%%%%%%%%%%%%%%%%%%%%%%%%%%%%%%%%%%%%%%%%%%%%%%%%%%%%%%
\begin{frame}[fragile]
  \frametitle{Alternative execution}
  \Enlarge

  \begin{itemize}
  \myitem  This lets us make decisions in the program! %\pause
  \myitem  We can change program behavior as it executes.
  \end{itemize}
\end{frame}

%%%%%%%%%%%%%%%%%%%%%%%%%%%%%%%%%%%%%%%%%%%%%%%%%%%%%%%%%%%%%%%%%%%%%%%%%%%%%%%%
\begin{frame}[fragile]
  \frametitle{Example:  \texttt{if} statements}
  \Enlarge

  \begin{Verbatim}[commandchars=\\\{\},commentchar=\%]
ans = input( "Enter a number:" )
if float(ans) < 0:
    print( "The number was negative." )
if float(ans) > 0:
    print( "The number was positive." )
if float(ans) == 0:
    print( "The number was zero." )
  \end{Verbatim}
\end{frame}

%%%%%%%%%%%%%%%%%%%%%%%%%%%%%%%%%%%%%%%%%%%%%%%%%%%%%%%%%%%%%%%%%%%%%%%%%%%%%%%%
\section{Reminders}

%%%%%%%%%%%%%%%%%%%%%%%%%%%%%%%%%%%%%%%%%%%%%%%%%%%%%%%%%%%%%%%%%%%%%%%%%%%%%%%%
\begin{frame}
  \frametitle{Reminders}
  \Enlarge

  \begin{itemize}
  \myitem  Homework \#2 is due Wed Oct.\ 19.
  \myitem  Homework \#3 is due Wed Oct.\ 26.
  \end{itemize}
\end{frame}

\end{document}

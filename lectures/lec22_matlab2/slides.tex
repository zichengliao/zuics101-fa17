%!TEX program = xelatex
\documentclass[11pt]{beamer}

\usepackage{amsfonts}
\usepackage{amsmath}
\usepackage{blindtext}
\usepackage{enumitem}
\usepackage{fancyvrb}
\usepackage{tikz}

%\usetheme{Recife}  % can use SaoPaulo as well
\usetheme{SaoPaulo}  % can use SaoPaulo as well

\title{MATLAB}
\subtitle{Introduction, Part II}
\author{CS101 Lecture \#23}
\date{2016-12-21}

\setcounter{showSlideNumbers}{1}

\newcommand{\correctstar}{{\Large\textcolor{red}{$\star$}}}

\begin{document}
  \setcounter{showProgressBar}{0}
  \setcounter{showSlideNumbers}{0}

%%%%%%%%%%%%%%%%%%%%%%%%%%%%%%%%%%%%%%%%%%%%%%%%%%%%%%%%%%%%%%%%%%%%%%%%%%%%%%%%
\frame{\titlepage}

%%%%%%%%%%%%%%%%%%%%%%%%%%%%%%%%%%%%%%%%%%%%%%%%%%%%%%%%%%%%%%%%%%%%%%%%%%%%%%%%
\setcounter{framenumber}{0}
\setcounter{showProgressBar}{1}
\setcounter{showSlideNumbers}{1}

%%%%%%%%%%%%%%%%%%%%%%%%%%%%%%%%%%%%%%%%%%%%%%%%%%%%%%%%%%%%%%%%%%%%%%%%%%%%%%%%
\section{Administrivia}

%%%%%%%%%%%%%%%%%%%%%%%%%%%%%%%%%%%%%%%%%%%%%%%%%%%%%%%%%%%%%%%%%%%%%%%%%%%%%%%%
\begin{frame}
  \frametitle{Administrivia}
  \Enlarge

  \begin{itemize}
  \myitem  Midterm \#2 graded
  \myitem  Homework \#11 will be  due Wed Jan. 4.  \pause
  \myitem  Homework \#12 will be released over the break, due Friday, Jan 13. 
  %\myitem  Grade check period coming up:  Dec.\ 9--14.  \pause
  \end{itemize}
\end{frame}

%%%%%%%%%%%%%%%%%%%%%%%%%%%%%%%%%%%%%%%%%%%%%%%%%%%%%%%%%%%%%%%%%%%%%%%%%%%%%%%%
\section{Warmup Questions}

%%%%%%%%%%%%%%%%%%%%%%%%%%%%%%%%%%%%%%%%%%%%%%%%%%%%%%%%%%%%%%%%%%%%%%%%%%%%%%%%
\begin{frame}[fragile]
  \frametitle{Question \#1}
  \Enlarge
$$
\left(
\begin{array}{ccc}
1 & 2 & 2 \\
2 & 1 & 2 \\
2 & 2 & 1
\end{array}
\right)
$$

How can we produce this array?

  \begin{enumerate}[label=\Alph*]
    \item  \texttt{ones(3,3) - 2*eye(3,3)}
    \item  \texttt{ones(3,3) + 2*eye(3,3)}
    \item  \texttt{2*ones(3,3) + eye(3,3)}
    \item  \texttt{2*ones(3,3) - eye(3,3)}
  \end{enumerate}
\end{frame}

%%%%%%%%%%%%%%%%%%%%%%%%%%%%%%%%%%%%%%%%%%%%%%%%%%%%%%%%%%%%%%%%%%%%%%%%%%%%%%%%
\begin{frame}[fragile]
  \frametitle{Question \#1}
  \Enlarge
$$
\left(
\begin{array}{ccc}
1 & 2 & 2 \\
2 & 1 & 2 \\
2 & 2 & 1
\end{array}
\right)
$$

How can we produce this array?

  \begin{enumerate}[label=\Alph*]
    \item  \texttt{ones(3,3) - 2*eye(3,3)}
    \item  \texttt{ones(3,3) + 2*eye(3,3)}
    \item  \texttt{2*ones(3,3) + eye(3,3)}
    \item  \texttt{2*ones(3,3) - eye(3,3)}  \correctstar
  \end{enumerate}
\end{frame}

%%%%%%%%%%%%%%%%%%%%%%%%%%%%%%%%%%%%%%%%%%%%%%%%%%%%%%%%%%%%%%%%%%%%%%%%%%%%%%%%
\begin{frame}[fragile]
  \frametitle{Question \#2}
  \Enlarge
$$
\left(
\begin{array}{cc}
1 & 2 \\
3 & 4 \\
5 & 6
\end{array}
\right)
$$

How do we access \texttt{6} in this array?

  \begin{enumerate}[label=\Alph*]
    \item  \texttt{A(2,1)}
    \item  \texttt{A(1,2)}
    \item  \texttt{A(3,2)}
    \item  \texttt{A(2,3)}
  \end{enumerate}
\end{frame}

%%%%%%%%%%%%%%%%%%%%%%%%%%%%%%%%%%%%%%%%%%%%%%%%%%%%%%%%%%%%%%%%%%%%%%%%%%%%%%%%
\begin{frame}[fragile]
  \frametitle{Question \#2}
  \Enlarge
$$
\left(
\begin{array}{cc}
1 & 2 \\
3 & 4 \\
5 & 6
\end{array}
\right)
$$

How do we access \texttt{6} in this array?

  \begin{enumerate}[label=\Alph*]
    \item  \texttt{A(2,1)}
    \item  \texttt{A(1,2)}
    \item  \texttt{A(3,2)}  \correctstar
    \item  \texttt{A(2,3)}
  \end{enumerate}
\end{frame}

%%%%%%%%%%%%%%%%%%%%%%%%%%%%%%%%%%%%%%%%%%%%%%%%%%%%%%%%%%%%%%%%%%%%%%%%%%%%%%%%
\section{MATLAB}

%%%%%%%%%%%%%%%%%%%%%%%%%%%%%%%%%%%%%%%%%%%%%%%%%%%%%%%%%%%%%%%%%%%%%%%%%%%%%%%
\begin{frame}[fragile]
  \frametitle{Basics}
  \Enlarge

  \begin{enumerate}
  \myitem  \texttt{a = [ 1 2 3 ]; \%row vector}
  \myitem  \texttt{b = [ 1 2 3 ]'; \%column vector}
  \myitem  \texttt{A = [ 1 2 3 ; 4 5 6 ]; \%matrix}
  \myitem  \texttt{B = [ a ; b ];  \% matrix composition}
  \end{enumerate}
\end{frame}

% Today will feel like a grab bag.
% Many things will be useful.  I will try to identify those that are not expected to be on an exam, however.
%TODO: check all code, add finite difference equations?


%%%%%%%%%%%%%%%%%%%%%%%%%%%%%%%%%%%%%%%%%%%%%%%%%%%%%%%%%%%%%%%%%%%%%%%%%%%%%%%
\begin{frame}[fragile]
	\frametitle{Matrix--Vector Operations}
	\Enlarge
	
	\begin{enumerate}
		\myitem If A is an m × n matrix (i.e., with n columns), then the product A x is defined for n × 1 column vectors x . If we let A x = b , then b is an m × 1 column vector. In other words, the number of rows in A (which can be anything) determines the number of rows in the product b. \url{http://mathinsight.org/matrix_vector_multiplication}
	\end{enumerate}
\end{frame}


%%%%%%%%%%%%%%%%%%%%%%%%%%%%%%%%%%%%%%%%%%%%%%%%%%%%%%%%%%%%%%%%%%%%%%%%%%%%%%%
\begin{frame}[fragile]
  \frametitle{Array operations}
  \Enlarge

  \begin{enumerate}
  \myitem  Matrix v. elementwise operations:
    \begin{enumerate}
    \mysubitem  Matrix operations are matrix--vector operations:
    \end{enumerate}
  \end{enumerate}
  $$
\left( \begin{array}{cc}
  1 & 0 \\
  0 & 1
\end{array} \right)
\left( \begin{array}{c}
  2 \\
  3
\end{array} \right)
=
\left( \begin{array}{c}
  2 \\
  3
\end{array} \right)
  $$
  \pause
  \begin{Verbatim}
[ 1 0 ; 0 1 ] * [ 2 3 ]'
  \end{Verbatim}
  \pause
  $$
\left( \begin{array}{cc}
  1 & 2 \\
  1 & 1
\end{array} \right)
\left( \begin{array}{c}
  2 \\
  3
\end{array} \right)
=
\left( \begin{array}{c}
  2+6 \\
  3+2
\end{array} \right)
=
\left( \begin{array}{c}
  8 \\
  5
\end{array} \right)
  $$
  \pause
  \begin{Verbatim}
[ 1 2 ; 1 1 ] * [ 2 3 ]'
  \end{Verbatim}
\end{frame}

%%%%%%%%%%%%%%%%%%%%%%%%%%%%%%%%%%%%%%%%%%%%%%%%%%%%%%%%%%%%%%%%%%%%%%%%%%%%%%%
\begin{frame}[fragile]
  \frametitle{Array operations}
  \Enlarge

  \begin{enumerate}
  \myitem  Matrix v. elementwise operations:
    \begin{enumerate}
    \mysubitem  Elementwise operations are spreadsheet-like operations:
    \end{enumerate}
  \end{enumerate}
  $$
\left( \begin{array}{cc}
  1 & 0 \\
  0 & 1
\end{array} \right)
\times
\left( \begin{array}{cc}
  2 & 4 \\
  3 & 5
\end{array} \right)
=
\left( \begin{array}{cc}
  2 & 0 \\
  0 & 5
\end{array} \right)
  $$
  \pause
  \begin{Verbatim}
[ 1 0 ; 0 1 ] .* [ 2 4 ; 3 5 ]
  \end{Verbatim}
\end{frame}

%%%%%%%%%%%%%%%%%%%%%%%%%%%%%%%%%%%%%%%%%%%%%%%%%%%%%%%%%%%%%%%%%%%%%%%%%%%%%%%%
\begin{frame}[fragile]
  \frametitle{Indexing arrays}
  \Enlarge

  \begin{enumerate}
  \myitem  We can index arrays with arrays.
  \end{enumerate}
  \begin{semiverbatim}
A = 0:10:100;
B = A( [ 5,9,2,2 ] );
  \end{semiverbatim}
  \pause
  \begin{enumerate}
  \myitem  This permits slicing.
  \end{enumerate}
  \begin{semiverbatim}
A = 0:10:100;
B = A( 4:7 );
  \end{semiverbatim}
\end{frame}

%%%%%%%%%%%%%%%%%%%%%%%%%%%%%%%%%%%%%%%%%%%%%%%%%%%%%%%%%%%%%%%%%%%%%%%%%%%%%%%%
\begin{frame}[fragile]
  \frametitle{Indexing arrays}
  \Enlarge

  \begin{enumerate}
  \myitem  In more dimensions:
  \end{enumerate}
  \begin{semiverbatim}
A = [ 1,2,3 ; 4,5,6 ; 7,8,9 ];
B = A( 1:2,1:2 );
C = A( :,1:2 );
  \end{semiverbatim}
\end{frame}

%%%%%%%%%%%%%%%%%%%%%%%%%%%%%%%%%%%%%%%%%%%%%%%%%%%%%%%%%%%%%%%%%%%%%%%%%%%%%%%
\begin{frame}[fragile]
  \frametitle{Multiple returns}
  \Enlarge

  \begin{enumerate}
  \myitem  Functions can return several values.  \pause
  \end{enumerate}
  \begin{Verbatim}
function [ a,b ] = nonsense( x,y )
    a = x ^ 2;
    b = y ^ 3;
end

[ q r ] = nonsense( 3,4 )
  \end{Verbatim}
\end{frame}

%%%%%%%%%%%%%%%%%%%%%%%%%%%%%%%%%%%%%%%%%%%%%%%%%%%%%%%%%%%%%%%%%%%%%%%%%%%%%%%
%\begin{frame}[fragile]
%  \frametitle{Multiple returns}
%  \Enlarge

%  \begin{enumerate}
%  \myitem  But be careful---sizes cause surprises.
%  \end{enumerate}
%  %TODO
%  \begin{Verbatim}
%A = [ 'HELLO';'WORLD' ];
%A( 2,3 )
%C = [ 'HELLO';'WORLD!' ];
 % \end{Verbatim}
 % \pause
 % \begin{enumerate}
 % \myitem  How could this affect function return %values?  \pause
 % \myitem  The solution is called a \emph{cell} (but we won't cover those in 101).
 % \end{enumerate}
%\end{frame}

%%%%%%%%%%%%%%%%%%%%%%%%%%%%%%%%%%%%%%%%%%%%%%%%%%%%%%%%%%%%%%%%%%%%%%%%%%%%%%%
\begin{frame}[fragile]
  \frametitle{Plotting}
  \Enlarge

  \begin{enumerate}
  \myitem  \texttt{plot} works identically to \texttt{plt.plot}.  \pause
  \myitem  \texttt{figure} creates a new figure (window for plots).  \pause
  \end{enumerate}
  \begin{Verbatim}
x = 0:.1:2*pi;
y = sin( x );
figure
plot( x,y,'o' );
title( 'sin(x)' );
xlabel( 'x values' );
ylabel( 'y values' );
  \end{Verbatim}
  \pause
  \begin{enumerate}
  \myitem  MATLAB also supplies an excellent plot editor.
  \end{enumerate}
\end{frame}

%%%%%%%%%%%%%%%%%%%%%%%%%%%%%%%%%%%%%%%%%%%%%%%%%%%%%%%%%%%%%%%%%%%%%%%%%%%%%%%
\begin{frame}[fragile]
  \frametitle{Plotting}
  \Enlarge

  \begin{enumerate}
  \myitem  Here's what we have now:
    \begin{enumerate}
    \mysubitem  \texttt{function}s
    \mysubitem  array definitions, operations, slicing
    \mysubitem  plotting
    \end{enumerate}
  \pause
  \myitem  We've seen these parts---what about the rest of our ``control structures''?
  \end{enumerate}
\end{frame}

%%%%%%%%%%%%%%%%%%%%%%%%%%%%%%%%%%%%%%%%%%%%%%%%%%%%%%%%%%%%%%%%%%%%%%%%%%%%%%%
\begin{frame}[fragile]
  \frametitle{Finite difference}

  \begin{Verbatim}
%% set parameters
alpha = 0.1;
tmax = 0.5;     % maximum time (s)
length = 3.0;   % length of material
dx = 0.2;       % mesh spacing
dt = 0.01;      % time step (s)

%% data storage initialization
t = 0:dt:tmax;                  % (s)
x = 0:dx:length;                % (m)
u = zeros(numel(t), numel(x));  % Kelvin
  \end{Verbatim}
\end{frame}

%%%%%%%%%%%%%%%%%%%%%%%%%%%%%%%%%%%%%%%%%%%%%%%%%%%%%%%%%%%%%%%%%%%%%%%%%%%%%%%
\begin{frame}[fragile]
  \frametitle{Finite difference}

  \begin{Verbatim}
%% set initial condition
u(1,x>=1&x<=2) = 353.15;         % Kelvin (= 80 deg C)
r = alpha * dt / (dx^2);
s = 1 - 2*r;

%% loop through time steps
for i = 2:1:numel(t)
    for j = 2:1:(numel(x)-1)
       u(i,j) = r*u(i-1,j-1) + s*u(i-1,j) + r*u(i-1,j+1);
    end
end
  \end{Verbatim}
\end{frame}

%%%%%%%%%%%%%%%%%%%%%%%%%%%%%%%%%%%%%%%%%%%%%%%%%%%%%%%%%%%%%%%%%%%%%%%%%%%%%%%%
\begin{frame}[fragile]
  \frametitle{\texttt{for} statement}
  \Enlarge

  \begin{itemize}
  \myitem  The \texttt{for} loop ranges over a set of possible values. \pause
  \myitem  This is \emph{not} as flexible as Python's \texttt{in} syntax---think of always having to loop over the \emph{index} rather than the item.
  \end{itemize}
\end{frame}

%%%%%%%%%%%%%%%%%%%%%%%%%%%%%%%%%%%%%%%%%%%%%%%%%%%%%%%%%%%%%%%%%%%%%%%%%%%%%%%%
\begin{frame}[fragile]
  \frametitle{\texttt{for} statement}
  \Enlarge

  \begin{itemize}
  \myitem  We create a \texttt{for} loop as follows:
    \begin{itemize}
    \mysubitem  statement \texttt{for var in range}, where you create \texttt{var} and provide \texttt{range}
    \mysubitem  one or more statements
    \mysubitem  closing statement \texttt{end}
    \end{itemize}
  \pause
  \myitem  Also have \texttt{continue} and \texttt{break} available.
  \end{itemize}
\end{frame}

%%%%%%%%%%%%%%%%%%%%%%%%%%%%%%%%%%%%%%%%%%%%%%%%%%%%%%%%%%%%%%%%%%%%%%%%%%%%%%%%
\begin{frame}[fragile]
  \frametitle{Example:  \texttt{absolute.m}}
  \Enlarge

  \begin{semiverbatim}
function [ y ] = absolute( x )
    y = 0;
    if x >= 0
        y = x;
    else
        y = -x;
end
  \end{semiverbatim}
\end{frame}

%%%%%%%%%%%%%%%%%%%%%%%%%%%%%%%%%%%%%%%%%%%%%%%%%%%%%%%%%%%%%%%%%%%%%%%%%%%%%%%%
\begin{frame}[fragile]
  \frametitle{\texttt{if}/\texttt{else} statement}
  \Enlarge

  \begin{itemize}
  \myitem  We create an \texttt{if}/\text{else} statement as follows:
    \begin{itemize}
    \mysubitem  the keyword \texttt{if}
    \mysubitem  a logical comparison \textcolor{red}{(more on these!)}
    \mysubitem  a \textbf{block} of code  \pause
    \mysubitem  the keyword \texttt{elseif} \textcolor{red}{(note this!)}
    \mysubitem  a new logical comparison
    \mysubitem  a different \textbf{block} of code  \pause
    \mysubitem  the keyword \texttt{else}
    \mysubitem  a different \textbf{block} of code  \pause
    \mysubitem  the keyword \texttt{end}
    \end{itemize}
  \end{itemize}
\end{frame}

%%%%%%%%%%%%%%%%%%%%%%%%%%%%%%%%%%%%%%%%%%%%%%%%%%%%%%%%%%%%%%%%%%%%%%%%%%%%%%%%
\begin{frame}[fragile]
  \frametitle{Logical statements}
  \Enlarge

  \begin{itemize}
  \myitem  MATLAB does \emph{not} have a \texttt{bool} data type. \pause
  \myitem  Instead of \texttt{True}/\texttt{False}, MATLAB uses integers:
    \begin{itemize}
    \mysubitem  \texttt{0} means \texttt{False}
    \mysubitem  \texttt{1} means \texttt{True}
    \end{itemize}
  \pause
  \myitem  Available logical operators include:
    \begin{itemize}
    \mysubitem  \texttt{<}, \texttt{>}, \texttt{<=}, \texttt{>=}, \texttt{==}, \texttt{\~=}
    \mysubitem  \texttt{\&\&} for `and', \texttt{||} for `or'
    \mysubitem  \texttt{ismember} checks equality of elements in arrays.
    \mysubitem  Also, logical operators as indices! \pause
    \mysubitem  \texttt{A( A<0 )}
    \end{itemize}
  \end{itemize}
\end{frame}

%%%%%%%%%%%%%%%%%%%%%%%%%%%%%%%%%%%%%%%%%%%%%%%%%%%%%%%%%%%%%%%%%%%%%%%%%%%%%%%
\begin{frame}[fragile]
  \frametitle{File I/O}
  \Enlarge

  \begin{enumerate}
  \myitem  Saving data uses \texttt{save}:
  \end{enumerate}
  \begin{Verbatim}
A = [ 1 2 3 ; 4 5 6 ];
save( 'test', 'A' );
  \end{Verbatim}
  \pause
  \begin{enumerate}
  \myitem  Note that the \emph{string} version of the variable name is required!
  \myitem  \texttt{load} also useful:
  \end{enumerate}
  \begin{Verbatim}
A = load( 'test', 'A' );
  \end{Verbatim}
\end{frame}

%%%%%%%%%%%%%%%%%%%%%%%%%%%%%%%%%%%%%%%%%%%%%%%%%%%%%%%%%%%%%%%%%%%%%%%%%%%%%%%
\begin{frame}[fragile]
  \frametitle{File I/O}
  \Enlarge

  \begin{enumerate}
  \myitem  A more advanced tool:  \texttt{importdata}
  \end{enumerate}
  \begin{Verbatim}
data = importdata( 'rainfall.txt' );
  \end{Verbatim}
  \pause
  \begin{enumerate}
  \myitem  Can be used to process CSVs.  \pause
  \myitem  Old process using \texttt{fopen}, \texttt{fscanf}, \texttt{fclose}, \texttt{fprintf} also common.
  \end{enumerate}
\end{frame}

%%%%%%%%%%%%%%%%%%%%%%%%%%%%%%%%%%%%%%%%%%%%%%%%%%%%%%%%%%%%%%%%%%%%%%%%%%%%%%%
\begin{frame}[fragile]
  \frametitle{Images}
  \Enlarge

  \begin{enumerate}
  \myitem  Images can also be opened as files.
  \end{enumerate}
  \begin{Verbatim}
A = importdata( 'rabbit-bw.jpg' );
image( A );
  \end{Verbatim}
  \pause
  \begin{enumerate}
  \myitem  Black and white images are arrays of \texttt{0}s and \texttt{1}s.
  \myitem  Greyscale images are values from \texttt{0} and \texttt{1}.
  \myitem  Color images are three-dimensional arrays.  (Why?)
  \myitem  Variations exist depending on the underlying data.
  \end{enumerate}
\end{frame}

% - PDE toolbox example?
% - find factorial using function browser

\end{document}

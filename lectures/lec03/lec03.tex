%!TEX program = xelatex
\documentclass[11pt]{beamer}

\usepackage{amsfonts}
\usepackage{amsmath}
\usepackage{amssymb}
\usepackage{blindtext}
\usepackage{enumitem}
\usepackage{fancyvrb}

\usetheme{SaoPaulo}

\title{Python Basics!}
\subtitle{data types, strings, indexing}
\author{CS101 Lecture \#3}
\date{2017-9-26}

\setcounter{showSlideNumbers}{1}

\begin{document}
  \setcounter{showProgressBar}{0}
  \setcounter{showSlideNumbers}{0}

%%%%%%%%%%%%%%%%%%%%%%%%%%%%%%%%%%%%%%%%%%%%%%%%%%%%%%%%%%%%%%%%%%%%%%%%%%%%%%%%
\frame{\titlepage}

%%%%%%%%%%%%%%%%%%%%%%%%%%%%%%%%%%%%%%%%%%%%%%%%%%%%%%%%%%%%%%%%%%%%%%%%%%%%%%%%
\setcounter{framenumber}{0}
\setcounter{showProgressBar}{1}
\setcounter{showSlideNumbers}{1}

%%%%%%%%%%%%%%%%%%%%%%%%%%%%%%%%%%%%%%%%%%%%%%%%%%%%%%%%%%%%%%%%%%%%%%%%%%%%%%%%
\section{Administrivia}

%%%%%%%%%%%%%%%%%%%%%%%%%%%%%%%%%%%%%%%%%%%%%%%%%%%%%%%%%%%%%%%%%%%%%%%%%%%%%%%%
\begin{frame}
  \frametitle{Administrivia}
  \Enlarge
  \begin{itemize}
  \mysubitem  Homework \#1 deadline just passed.
  %\mysubitem  Final answer counts.
  \mysubitem  Solutions have been released on CodeLab.
  \end{itemize}
\end{frame}

%%%%%%%%%%%%%%%%%%%%%%%%%%%%%%%%%%%%%%%%%%%%%%%%%%%%%%%%%%%%%%%%%%%%%%%%%%%%%%%%
\begin{frame}
  \frametitle{Administrivia}
  \Enlarge
  \begin{itemize}
  \myitem  Where can you get help in this class?
    \begin{itemize}
    \mysubitem  Blackboard forum 
    \mysubitem  Lab and office hours 
    \mysubitem  Email me
    \mysubitem  \emph{Make use of search engine!}
    \end{itemize} %\pause
   %\myitem  You don't need to install Python---but you're encouraged to have one.
  \end{itemize}
\end{frame}

%%%%%%%%%%%%%%%%%%%%%%%%%%%%%%%%%%%%%%%%%%%%%%%%%%%%%%%%%%%%%%%%%%%%%%%%%%%%%%%%
\begin{frame}
  \frametitle{Administrivia}
  %\Enlarge
  %\begin{itemize}
  Course enrollment and CodeLab registration using zju.edu.cn account (or the intl. account) is OK. \\
  
  \vspace{2mm}
  But {\bf Lab submissions} (to: \textcolor{blue}{cs101homework@intl.zju.edu.cn}) have to come from your \textcolor{blue}{intl.zju.edu.cn} account.
  %\end{itemize}
\end{frame}


%%%%%%%%%%%%%%%%%%%%%%%%%%%%%%%%%%%%%%%%%%%%%%%%%%%%%%%%%%%%%%%%%%%%%%%%%%%%%%%%
\begin{frame}
  \frametitle{Administrivia}
  \Enlarge
  \begin{itemize}
  \myitem  {\Large Lab \#2 tomorrow.}
  \end{itemize}
\end{frame}

%%%%%%%%%%%%%%%%%%%%%%%%%%%%%%%%%%%%%%%%%%%%%%%%%%%%%%%%%%%%%%%%%%%%%%%%%%%%%%%%
\section{Quick Review \& A Bit New}

%%%%%%%%%%%%%%%%%%%%%%%%%%%%%%%%%%%%%%%%%%%%%%%%%%%%%%%%%%%%%%%%%%%%%%%%%%%%%%%%
\begin{frame}[fragile]
  \frametitle{How Assignment Works}
  \Enlarge

  \begin{semiverbatim}
x = 10 \pause
y = x * x \pause
x * x = y \pause  \textcolor{red}{# error! assignment is from rhs to lhs} \pause
x,y = y,x   # a neat trick
\end{semiverbatim}

\end{frame}

%%%%%%%%%%%%%%%%%%%%%%%%%%%%%%%%%%%%%%%%%%%%%%%%%%%%%%%%%%%%%%%%%%%%%%%%%%%%%%%%
\section{Warmup Quiz}


%%%%%%%%%%%%%%%%%%%%%%%%%%%%%%%%%%%%%%%%%%%%%%%%%%%%%%%%%%%%%%%%%%%%%%%%%%%%%%%%
\begin{frame}[fragile]
  \frametitle{Question \#1}
  \Enlarge

  \begin{semiverbatim}
x = 10
y = x + 1
y = x * y
  \end{semiverbatim}
  What is the value of \texttt{y}?
  \begin{enumerate}[label=\Alph*]
  \item  11
  \item  100
  \item  110
  \item  None of the above
  \end{enumerate}
\end{frame}

%%%%%%%%%%%%%%%%%%%%%%%%%%%%%%%%%%%%%%%%%%%%%%%%%%%%%%%%%%%%%%%%%%%%%%%%%%%%%%%%
\begin{frame}[fragile]
  \frametitle{Question \#2}
  \Enlarge

  \begin{semiverbatim}
x = 10
y = x + 1
y = x * y
  \end{semiverbatim}
  What do we call \texttt{x}?
  \begin{enumerate}[label=\Alph*]
  \item  a literal
  \item  a variable
  \item  an expression
  \item  a statement
  \end{enumerate}
\end{frame}

%%%%%%%%%%%%%%%%%%%%%%%%%%%%%%%%%%%%%%%%%%%%%%%%%%%%%%%%%%%%%%%%%%%%%%%%%%%%%%%%
\begin{frame}[fragile]
  \frametitle{Question \#3}
  \Enlarge

  \begin{semiverbatim}
x = 10
y = x + 1
y = x * y
  \end{semiverbatim}
  What do we call \texttt{10}?
  \begin{enumerate}[label=\Alph*]
  \item  a literal
  \item  a variable
  \item  an expression
  \item  a statement
  \end{enumerate}
\end{frame}

%%%%%%%%%%%%%%%%%%%%%%%%%%%%%%%%%%%%%%%%%%%%%%%%%%%%%%%%%%%%%%%%%%%%%%%%%%%%%%%%
\begin{frame}[fragile]
  \frametitle{Question \#4}
  \Enlarge

  \begin{semiverbatim}
x = 10
y = x + 1
y = x * y
  \end{semiverbatim}
  What do we call \texttt{y = x * y}?
  \begin{enumerate}[label=\Alph*]
  \item  a literal
  \item  a variable
  \item  an expression
  \item  a statement
  \end{enumerate}
\end{frame}

%%%%%%%%%%%%%%%%%%%%%%%%%%%%%%%%%%%%%%%%%%%%%%%%%%%%%%%%%%%%%%%%%%%%%%%%%%%%%%%%
\begin{frame}[fragile]
  \frametitle{Question \#5}
  \Enlarge

  \begin{semiverbatim}
x = 10
y = x + 1
y = x * y
  \end{semiverbatim}
  What do we call \texttt{x * y}?
  \begin{enumerate}[label=\Alph*]
  \item  a literal
  \item  a variable
  \item  an expression
  \item  a statement
  \end{enumerate}
\end{frame}

%%%%%%%%%%%%%%%%%%%%%%%%%%%%%%%%%%%%%%%%%%%%%%%%%%%%%%%%%%%%%%%%%%%%%%%%%%%%%%%%
\begin{frame}[fragile]
  \frametitle{Question \#6}
  \Enlarge

  \begin{semiverbatim}
x = 10
y = x
x = 5
  \end{semiverbatim}
  What is the value of \texttt{y}?
  \begin{enumerate}[label=\Alph*]
  \item  10
  \item  5
  \end{enumerate}
\end{frame}

\iffalse
%%%%%%%%%%%%%%%%%%%%%%%%%%%%%%%%%%%%%%%%%%%%%%%%%%%%%%%%%%%%%%%%%%%%%%%%%%%%%%%%
\begin{frame}
	\frametitle{Reminder}
	\Enlarge
	\begin{itemize}
		\myitem  You will have graded quiz starting from the upcoming Monday lecture!
	\end{itemize}
\end{frame}
\fi

%%%%%%%%%%%%%%%%%%%%%%%%%%%%%%%%%%%%%%%%%%%%%%%%%%%%%%%%%%%%%%%%%%%%%%%%%%%%%%%%
\section{Data Types}

%%%%%%%%%%%%%%%%%%%%%%%%%%%%%%%%%%%%%%%%%%%%%%%%%%%%%%%%%%%%%%%%%%%%%%%%%%%%%%%%
\begin{frame}
  \frametitle{Why need data type?}
  \Enlarge

  \texttt{01001000 01000101 01001100 01001100} \\
  A computer binary code
  \begin{itemize}
  \myitem  Computer represents different kinds of data (5, 'apple', operator +) in 0s and 1s\pause
  \myitem  Different types of data are {\bf encoded} in binary with different rules \pause
  \myitem What is \emph{encoding}?
  \end{itemize}
\end{frame}


%%%%%%%%%%%%%%%%%%%%%%%%%%%%%%%%%%%%%%%%%%%%%%%%%%%%%%%%%%%%%%%%%%%%%%%%%%%%%%%%
\begin{frame}
  \frametitle{Example}
  \Enlarge
  The same binary data can be interpreted in different ways based on their \emph{data type} \\ \pause
  
  \vspace{4mm}
  \texttt{01100111} can be the number \texttt{103}, hexadecimal \texttt{67}, or a letter \texttt{'g'}, etc. \\ \pause
  
  \vspace{4mm}
  In order to interpret it correctly, we need to know its data type.
\end{frame}

%%%%%%%%%%%%%%%%%%%%%%%%%%%%%%%%%%%%%%%%%%%%%%%%%%%%%%%%%%%%%%%%%%%%%%%%%%%%%%%%
\begin{frame}
  \frametitle{What is a \textbf{data type}?}
  \Enlarge
  \Enlarge

  \begin{itemize}
  \myitem  A \textbf{data type} defines an encoding rule. \pause
  \begin{itemize}
  \mysubitem  i.e. how data is represented in memory by 0s and 1s. \pause
  \end{itemize}
  \myitem  It also defines the allowed operations 
  \begin{itemize}
  	\mysubitem e.g. cannot do arithmetic to characters.
  \end{itemize}
  \end{itemize}
\end{frame}


%%%%%%%%%%%%%%%%%%%%%%%%%%%%%%%%%%%%%%%%%%%%%%%%%%%%%%%%%%%%%%%%%%%%%%%%%%%%%%%%
\section{Numeric Data Types}

%%%%%%%%%%%%%%%%%%%%%%%%%%%%%%%%%%%%%%%%%%%%%%%%%%%%%%%%%%%%%%%%%%%%%%%%%%%%%%%%
\begin{frame}
  \frametitle{Representing numbers in binary}
  \Enlarge

  \begin{itemize}
  \myitem  Binary encoding for numbers:
    \begin{tabular}{*{27}{l}}
      \texttt{00000000} & 0 & \texttt{00000100} & 4 & \texttt{00001000} & 8\\
      \texttt{00000001} & 1 & \texttt{00000101} & 5 & \texttt{00001001} & 9\\
      \texttt{00000010} & 2 & \texttt{00000110} & 6 & ... & \\
      \texttt{00000011} & 3 & \texttt{00000111} & 7 & \texttt{11111111} & ...\\
    \end{tabular}\pause
  
  \myitem example: 01011010
  \end{itemize} \pause
  
  
  %\textcolor{blue}{\small \texttt{\url{https://en.wikipedia.org/wiki/Binary_number}}} \\
  \hspace{7mm} \textcolor{blue}{\small \texttt{\url{https://www.bottomupcs.com/chapter01.xhtml}}} 
  
\end{frame}

%%%%%%%%%%%%%%%%%%%%%%%%%%%%%%%%%%%%%%%%%%%%%%%%%%%%%%%%%%%%%%%%%%%%%%%%%%%%%%%%
\begin{frame}
  \frametitle{Integers ($int$), $\mathbb{Z}$}
  \Enlarge

  \begin{itemize}
  \myitem  How about \emph{Integers}? \\ 
  $$ ..., -3, -2, -1, 0, 1, 2, 3, ... $$\vspace{-2mm}  \pause
  \myitem Negative numbers  \pause
  	\begin{itemize}
		\mysubitem Use the leftmost bit as \textbf{sign bit}. \pause
		\mysubitem 0 for positive; 1 for negative\pause
		\mysubitem the rest of the bits representing magnitude\pause
	\end{itemize}
  \myitem What are the limits of a 8-bit integer representation?\\ \pause
    $$ -128 ... 127 $$
  \end{itemize}
\end{frame}

%%%%%%%%%%%%%%%%%%%%%%%%%%%%%%%%%%%%%%%%%%%%%%%%%%%%%%%%%%%%%%%%%%%%%%%%%%%%%%%%
\begin{frame}
  \frametitle{History}
  \Enlarge

  \begin{itemize}
  \myitem Old version python {\bf int} are 32 bits long (in the range of $-2^{31}$ to $2^{31}-1$) 
  \myitem That's -2147483648 to 2147483647
  \myitem values too big: \emph{overflow}
  \myitem values too small: \emph{underflow}
  \end{itemize}
\end{frame}

%%%%%%%%%%%%%%%%%%%%%%%%%%%%%%%%%%%%%%%%%%%%%%%%%%%%%%%%%%%%%%%%%%%%%%%%%%%%%%%%
\begin{frame}
  \frametitle{History}
  \Enlarge

  \begin{itemize}
  \myitem Python has another integer type: {\bf long} 
  \myitem Represents with no restrictions on size (no overflow/underflow) \pause
  \myitem Since v2.2, python converts int overflow to a {\bf long} \pause
  \myitem newer Python versions promises there is no distinction between {\bf int} and {\bf long} \pause
  %\myitem Arithmetic with integers in Python will always be correct \pause
  \myitem Don't get spoiled by this (many languages still have clear integer types and limits).
  \end{itemize}
  
  \textcolor{blue}{\small \texttt{\url{https://en.wikipedia.org/wiki/Integer_(computer_science)}}} 
  
\end{frame}


%%%%%%%%%%%%%%%%%%%%%%%%%%%%%%%%%%%%%%%%%%%%%%%%%%%%%%%%%%%%%%%%%%%%%%%%%%%%%%%%
\begin{frame}
  \frametitle{Integer operations}
  \Enlarge

  \begin{itemize}
  \myitem  Evaluating an expression of integers will generally result in an integer answer
    \begin{itemize}
    \mysubitem  \texttt{3 + 5} \pause
    \mysubitem  \textcolor{red}{EXCEPTION:  DIVISION!} \pause
    \mysubitem  \texttt{3 / 4 $\rightarrow$ 0.75} \pause
    \mysubitem  \texttt{3 // 4 $\rightarrow$ 0} (floor division)\pause
    \mysubitem  \texttt{4 / 2 $\rightarrow$ ??}
    \end{itemize}
  \end{itemize}
\end{frame}

%%%%%%%%%%%%%%%%%%%%%%%%%%%%%%%%%%%%%%%%%%%%%%%%%%%%%%%%%%%%%%%%%%%%%%%%%%%%%%%%
\begin{frame}
  \frametitle{Floating-point numbers, $\mathbb{R}$}
  \Enlarge

  \begin{itemize}
  \myitem  Floating-point numbers include a fractional part. \\
    \textcolor{CS101GradBot}{(Anything with a decimal point---\texttt{2.4}, \texttt{3.0}.)} \pause
  \myitem  What's different? \pause
    \begin{itemize}
    \mysubitem  represents \emph{Real numbers} (integers, fractional, and $\pi$, $e$, sqrt(2), etc.) \pause
    \mysubitem  up to a precision and limit (maximum and minimum)
    \end{itemize}
  %\myitem Floating point representation in Binary
  \end{itemize}
  %\hspace{7mm}\textcolor{blue}{\small \texttt{\url{https://en.wikipedia.org/wiki/IEEE_754-1985}}} {\small(IEEE 754 standard)}
  %\textcolor{blue}{\small \texttt{\url{http://sites.cs.queensu.ca/courses/cisc121/Record/Week09/Arith.pdf}}}
\end{frame}

%%%%%%%%%%%%%%%%%%%%%%%%%%%%%%%%%%%%%%%%%%%%%%%%%%%%%%%%%%%%%%%%%%%%%%%%%%%%%%%%
\begin{frame}
  \frametitle{Floating-point numbers, $\mathbb{R}$}
  \Enlarge
  \begin{center}
  \includegraphics[width=0.6\textwidth]{./img/Numbersystems.png}\\
  Real numbers (R) include the rational (Q), which include the integers (Z), which include the natural numbers (N).
  \end{center}
\end{frame}

%%%%%%%%%%%%%%%%%%%%%%%%%%%%%%%%%%%%%%%%%%%%%%%%%%%%%%%%%%%%%%%%%%%%%%%%%%%%%%%%
\begin{frame}
  \frametitle{Floating-point operations}
  \Enlarge

  \begin{itemize}
  \myitem  Evaluating an expression of floating-point values will result in a floating-point answer. \pause
    \begin{itemize}
    \mysubitem  \texttt{3.0 + 5.5 $\rightarrow$ 8.5} \pause
    \mysubitem  \texttt{3.0 + 5.0 $\rightarrow$ 8.0} \pause
    \mysubitem  \texttt{3   + 5.5 $\rightarrow$ ?} (what happens here?)
    \end{itemize} \pause
  \myitem  Engineers and scientists need to think carefully about data type, precision, and type \emph{conversion}.
  \end{itemize}
\end{frame}


%%%%%%%%%%%%%%%%%%%%%%%%%%%%%%%%%%%%%%%%%%%%%%%%%%%%%%%%%%%%%%%%%%%%%%%%%%%%%%%%
\begin{frame}
  %\frametitle{Complex numbers, $\mathbb{C}$}
  %\Enlarge
  \begin{center}
  \vspace{3mm}
  \includegraphics[width=0.55\textwidth]{./img/monroeeinstein.png}\\
  Einstein or Monroe?
  \end{center}
\end{frame}


\iffalse
%%%%%%%%%%%%%%%%%%%%%%%%%%%%%%%%%%%%%%%%%%%%%%%%%%%%%%%%%%%%%%%%%%%%%%%%%%%%%%%%
\begin{frame}
  \frametitle{Complex numbers, $\mathbb{C}$}
  \Enlarge

  \begin{itemize}
  \myitem  Represent numbers with an imaginary component. \pause
  \myitem  Use \texttt{j} for $i$: \\
    \textcolor{CS101GradBot}{\texttt{1.0 + 1j}} \\
    \textcolor{CS101GradBot}{\texttt{2 + 0j}} 
  %\myitem  Think of "jmaginary" numbers, I suppose.
  \end{itemize}
\end{frame}

%%%%%%%%%%%%%%%%%%%%%%%%%%%%%%%%%%%%%%%%%%%%%%%%%%%%%%%%%%%%%%%%%%%%%%%%%%%%%%%%
\begin{frame}[fragile]
  \frametitle{Example}
  \Enlarge

  \begin{semiverbatim}
x = 4
y = 3 + 1j
z = 33.3333
print( x + y + z )
  \end{semiverbatim}
  What is printed to the screen?
  \begin{enumerate}[label=\Alph*]
  \item  \texttt{40}
  \item  \texttt{40.3333}
  \item  \texttt{40.3333 + 1j}
  \item  None of the above
  \end{enumerate}
\end{frame}

%%%%%%%%%%%%%%%%%%%%%%%%%%%%%%%%%%%%%%%%%%%%%%%%%%%%%%%%%%%%%%%%%%%%%%%%%%%%%%%%
\begin{frame}
  \frametitle{Attribute operator \textbf{.}}
  \Enlarge

  \begin{itemize}
  \myitem  Reaches inside of a value to access part of its data (called an attribute). \pause
  \myitem  Extracts special variables stored ``inside'' of the type.
    \begin{semiverbatim}
print(x.real)

print(x.imag)
    \end{semiverbatim} \pause
  \myitem  Both of these components are floats.
  \end{itemize}
\end{frame}

%%%%%%%%%%%%%%%%%%%%%%%%%%%%%%%%%%%%%%%%%%%%%%%%%%%%%%%%%%%%%%%%%%%%%%%%%%%%%%%%
\begin{frame}[fragile]
  \frametitle{Example}
  \Enlarge

  \begin{semiverbatim}
x = (3.5 + 1j)
y = 1
z = x + y
  \end{semiverbatim}
  What is the value of \texttt{z.imag}? \pause
  \begin{enumerate}[label=\Alph*]
  \item  \texttt{4.5 + 1j}
  \item  \texttt{4.5}
  \item  \texttt{1j}
  \item  \texttt{1.0}
  \end{enumerate}
\end{frame}

\fi
%%%%%%%%%%%%%%%%%%%%%%%%%%%%%%%%%%%%%%%%%%%%%%%%%%%%%%%%%%%%%%%%%%%%%%%%%%%%%%%%
\section{String Data Type}


%%%%%%%%%%%%%%%%%%%%%%%%%%%%%%%%%%%%%%%%%%%%%%%%%%%%%%%%%%%%%%%%%%%%%%%%%%%%%%%%
\begin{frame}
  \frametitle{ASCII encoding table}
  \Enlarge
  \includegraphics[width=\textwidth]{./img/ascii-table.png} \\ \pause
  
  {\small The table provides an \emph{encoding} scheme from symbols to numbers} \\
  \texttt{72 69 76 76 79} = \texttt{H E L L O} %\pause
  
  %\texttt{'HELLO'}
\end{frame}

%%%%%%%%%%%%%%%%%%%%%%%%%%%%%%%%%%%%%%%%%%%%%%%%%%%%%%%%%%%%%%%%%%%%%%%%%%%%%%%%
\begin{frame}
  \frametitle{How to store text on computer?}
  \Enlarge

  \begin{itemize}
  \myitem \texttt{H E L L O} = \texttt{72 69 76 76 79} \pause
  \myitem  Each symbol is stored individually, one byte long: \\
   \vspace{2mm} \pause
    \begin{tabular}{*{27}{l}}
      72 & \texttt{01001000} \\
      69 & \texttt{01000101} \\
      76 & \texttt{01001100} \\
      76 & \texttt{01001100} \\
      79 & \texttt{01001111} \\
    \end{tabular} \pause
    
    \vspace{2mm}
    {\small \texttt{'HELLO'} : \textcolor{CS101GradBot}{\texttt{01001000 01000101 01001100 01001100 01001111}}}
  \end{itemize}
\end{frame}

%%%%%%%%%%%%%%%%%%%%%%%%%%%%%%%%%%%%%%%%%%%%%%%%%%%%%%%%%%%%%%%%%%%%%%%%%%%%%%%%
\begin{frame}
  \frametitle{How to store text on computer?}
  \Enlarge

  \begin{itemize}
   \myitem What's the size of a plain txt file with 1000 english words? 
  \end{itemize}
\end{frame}

%%%%%%%%%%%%%%%%%%%%%%%%%%%%%%%%%%%%%%%%%%%%%%%%%%%%%%%%%%%%%%%%%%%%%%%%%%%%%%%%
\begin{frame}
  \frametitle{Strings}
  \Enlarge

  \begin{itemize}
  \myitem  As a literal:  text surrounded by quotes.
    \begin{itemize}
    \mysubitem  \texttt{"DEEP"}
    \end{itemize} \pause
  \myitem  Each symbol is a character. \pause
  \myitem  Unlike numeric types, strings vary in length.
  \end{itemize}
\end{frame}

%%%%%%%%%%%%%%%%%%%%%%%%%%%%%%%%%%%%%%%%%%%%%%%%%%%%%%%%%%%%%%%%%%%%%%%%%%%%%%%%
\begin{frame}
  \frametitle{String operations}
  \Enlarge

  \begin{itemize}
  \myitem  \textbf{Concatenation}:  combine two strings
    \begin{itemize}
    \mysubitem  Uses the \texttt{+} symbol
    \mysubitem  \texttt{'RACE' + 'CAR'} \pause
    \mysubitem the ``same'' operator works differently with different types of operand (\emph{operator overload})
    \end{itemize} \pause
  \myitem  \textbf{Repetition}:  repeat a string
    \begin{itemize}
    \mysubitem  Uses the \texttt{*}
    \mysubitem  \texttt{'HELLO '*10}
    \end{itemize} \pause
  \myitem  \textbf{Formatting}:  used to encode other data as string
    \begin{itemize}
    \mysubitem  Uses \texttt{\%} symbol
    \end{itemize}
  \end{itemize}
\end{frame}

%%%%%%%%%%%%%%%%%%%%%%%%%%%%%%%%%%%%%%%%%%%%%%%%%%%%%%%%%%%%%%%%%%%%%%%%%%%%%%%%
\begin{frame}[fragile]
  \frametitle{Formatting operator}
  \Enlarge

  \begin{itemize}
  \myitem  Creates string with value inserted \pause
    \begin{itemize}
    \mysubitem  Formats nicely
    \mysubitem  Requires indicator of type inside of string
    \end{itemize} \pause
  \begin{semiverbatim}
x = 123 * 54
s = "String is: %i" % x
print(s)
  \end{semiverbatim}
  \end{itemize}
\end{frame}

%%%%%%%%%%%%%%%%%%%%%%%%%%%%%%%%%%%%%%%%%%%%%%%%%%%%%%%%%%%%%%%%%%%%%%%%%%%%%%%%
\begin{frame}[fragile]
  \frametitle{Example}
  \Enlarge

  \begin{Verbatim}[commandchars=\\\{\}]
name = "Tao"
grade = 2 / 3
m1 = "Hello, %s!" % name
m2 = "Your grade is:  %f." % grade
print(m1)
print(m2) \pause

\textcolor{CS101PureBase}{
Hello, Tao!}
\textcolor{CS101PureBase}{
Your grade is 0.66667.}
  \end{Verbatim}
\end{frame}

%%%%%%%%%%%%%%%%%%%%%%%%%%%%%%%%%%%%%%%%%%%%%%%%%%%%%%%%%%%%%%%%%%%%%%%%%%%%%%%%
\begin{frame}[fragile]
  \frametitle{Example}
  \Enlarge

  \begin{semiverbatim}
x = 3
s = ("%i" % (x+1)) * x**(5%x)
print(s)
  \end{semiverbatim}

  What does this program print?
  \begin{enumerate}[label=\Alph*]
  \item  \texttt{333333333333}
  \item  \texttt{444444444}
  \item  \texttt{9999}
  \item  \texttt{\%i\%i\%i\%i\%i}
  \end{enumerate}
\end{frame}

%%%%%%%%%%%%%%%%%%%%%%%%%%%%%%%%%%%%%%%%%%%%%%%%%%%%%%%%%%%%%%%%%%%%%%%%%%%%%%%%
\begin{frame}[fragile]
  \frametitle{Indexing operator}
  \Enlarge

  \begin{itemize}
  \myitem  Extracts single character \pause
\begin{semiverbatim}
a = "FIRE"
a[0]
\end{semiverbatim} \pause
  \myitem  The integer is the index. \pause
  \myitem  \textcolor{red}{\bf We count from zero!} (same in C, C++, Java) \pause
  \myitem  If negative, counts down from end.
  \myitem a[-1] refers to the last character
  \end{itemize}
\end{frame}

%%%%%%%%%%%%%%%%%%%%%%%%%%%%%%%%%%%%%%%%%%%%%%%%%%%%%%%%%%%%%%%%%%%%%%%%%%%%%%%%
\begin{frame}[fragile]
  \frametitle{Question}
  \Enlarge

  \begin{semiverbatim}
s = "ABCDE"
i = 3
x = s[i]
  \end{semiverbatim}

  What is the value of \texttt{x}?
  \begin{enumerate}[label=\Alph*]
  \item  \texttt{'A'}
  \item  \texttt{'B'}
  \item  \texttt{'C'}
  \item  \texttt{'D'}
  \item  \texttt{'E'}
  \end{enumerate}
\end{frame}

%%%%%%%%%%%%%%%%%%%%%%%%%%%%%%%%%%%%%%%%%%%%%%%%%%%%%%%%%%%%%%%%%%%%%%%%%%%%%%%%
\begin{frame}[fragile]
  \frametitle{Question}
  \Enlarge

  \begin{semiverbatim}
s = "ABCDE"
i = 25 % 3
y = s[i]
  \end{semiverbatim}

  What is the value of \texttt{y}?
  \begin{enumerate}[label=\Alph*]
  \item  \texttt{'A'}
  \item  \texttt{'B'}
  \item  \texttt{'C'}
  \item  \texttt{'D'}
  \item  \texttt{'E'}
  \end{enumerate}
\end{frame}

%%%%%%%%%%%%%%%%%%%%%%%%%%%%%%%%%%%%%%%%%%%%%%%%%%%%%%%%%%%%%%%%%%%%%%%%%%%%%%%%
\begin{frame}[fragile]
  \frametitle{Question}
  \Enlarge

  \begin{semiverbatim}
s = "ABCDE"
i = (11 % 3) - 7
z = s[i]
  \end{semiverbatim}

  What is the value of \texttt{z}?
  \begin{enumerate}[label=\Alph*]
  \item  \texttt{'A'}
  \item  \texttt{'B'}
  \item  \texttt{'C'}
  \item  \texttt{'D'}
  \item  \texttt{'E'}
  \end{enumerate}
\end{frame}

%%%%%%%%%%%%%%%%%%%%%%%%%%%%%%%%%%%%%%%%%%%%%%%%%%%%%%%%%%%%%%%%%%%%%%%%%%%%%%%%
\begin{frame}[fragile]
  \frametitle{Question}
  \Enlarge

  \begin{semiverbatim}
s = "ABCDE"
i = (11 % 3) + 3
z = s[i]
  \end{semiverbatim}

  What is the value of \texttt{z}?
  %How about \texttt{s[-5]}?
\end{frame}



\end{document}

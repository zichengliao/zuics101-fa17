%!TEX program = xelatex
\documentclass[11pt]{beamer}

\usepackage{amsfonts}
\usepackage{amsmath}
\usepackage{blindtext}
\usepackage{enumitem}
\usepackage{fancyvrb}

\usetheme{SaoPaulo}

\title{Python Applications}
\subtitle{manipulating \texttt{lists}s}
\author{CS101 Lecture \#13}
\date{2016-11-14}

\setcounter{showSlideNumbers}{1}

\begin{document}
  \setcounter{showProgressBar}{0}
  \setcounter{showSlideNumbers}{0}

%%%%%%%%%%%%%%%%%%%%%%%%%%%%%%%%%%%%%%%%%%%%%%%%%%%%%%%%%%%%%%%%%%%%%%%%%%%%%%%%
\frame{\titlepage}

%%%%%%%%%%%%%%%%%%%%%%%%%%%%%%%%%%%%%%%%%%%%%%%%%%%%%%%%%%%%%%%%%%%%%%%%%%%%%%%%
\setcounter{framenumber}{0}
\setcounter{showProgressBar}{1}
\setcounter{showSlideNumbers}{1}

%%%%%%%%%%%%%%%%%%%%%%%%%%%%%%%%%%%%%%%%%%%%%%%%%%%%%%%%%%%%%%%%%%%%%%%%%%%%%%%%
\section{Administrivia}

%%%%%%%%%%%%%%%%%%%%%%%%%%%%%%%%%%%%%%%%%%%%%%%%%%%%%%%%%%%%%%%%%%%%%%%%%%%%%%%%
\begin{frame}
  \frametitle{Administrivia}
  \Enlarge

  \begin{itemize}
  \myitem  Homework \#7 is due Friday, Nov.\ 25.
  \mysubitem  Use the \texttt{split(',')} approach.
  \myitem  Midterm reflection exercise on website for extra credit.
  \end{itemize}
\end{frame}

%%%%%%%%%%%%%%%%%%%%%%%%%%%%%%%%%%%%%%%%%%%%%%%%%%%%%%%%%%%%%%%%%%%%%%%%%%%%%%%%
%\section{Feedback \& Midterm Results}

%%%%%%%%%%%%%%%%%%%%%%%%%%%%%%%%%%%%%%%%%%%%%%%%%%%%%%%%%%%%%%%%%%%%%%%%%%%%%%%%
\section{Warmup Quiz (No Real Quiz Today)}

%%%%%%%%%%%%%%%%%%%%%%%%%%%%%%%%%%%%%%%%%%%%%%%%%%%%%%%%%%%%%%%%%%%%%%%%%%%%%%%%
\begin{frame}[fragile]
  \frametitle{Mutability Question}
  \Enlarge

  Which of the following sets of \texttt{list} methods \emph{all} change the function in place (have no return value)?

  \begin{enumerate}[label=\Alph*]
  \item  \texttt{split}, \texttt{append}, \texttt{extend}
  \item  \texttt{del}, \texttt{index}, \texttt{upper}
  \item  \texttt{read}, \texttt{readlines}, \texttt{close}
  \item  \texttt{sort}, \texttt{reverse}, \texttt{append}, \texttt{extend}
  \end{enumerate}
\end{frame}

%%%%%%%%%%%%%%%%%%%%%%%%%%%%%%%%%%%%%%%%%%%%%%%%%%%%%%%%%%%%%%%%%%%%%%%%%%%%%%%%
\begin{frame}[fragile]
  \frametitle{Mutability Question}
  \Enlarge

  Which of the following sets of \texttt{list} methods \emph{all} change the function in place (have no return value)?

  \begin{enumerate}[label=\Alph*]
  \item  \texttt{split}, \textcolor{CS101Alt}{\texttt{append}}, \textcolor{CS101Alt}{\texttt{extend}}
  \item  \textcolor{CS101Alt}{\texttt{del}}, \texttt{index}, \texttt{upper}
  \item  \texttt{read}, \texttt{readlines}, \texttt{close}
  \item  \textcolor{CS101Alt}{\texttt{sort}}, \textcolor{CS101Alt}{\texttt{reverse}}, \textcolor{CS101Alt}{\texttt{append}}, \textcolor{CS101Alt}{\texttt{extend}}  $\star$
  \end{enumerate}
\end{frame}

%%%%%%%%%%%%%%%%%%%%%%%%%%%%%%%%%%%%%%%%%%%%%%%%%%%%%%%%%%%%%%%%%%%%%%%%%%%%%%%%
\section{Working with Containers}

%%%%%%%%%%%%%%%%%%%%%%%%%%%%%%%%%%%%%%%%%%%%%%%%%%%%%%%%%%%%%%%%%%%%%%%%%%%%%%%%
\begin{frame}[fragile]
  \frametitle{\texttt{list}s and \texttt{dict}s}
  \Enlarge

  \begin{tabular}{ll}
  \hline\hline \\
  \texttt{list} & \\
  modifies in place \hspace{2cm} & returns value \\
  \hline \\
  \texttt{append} & \texttt{index} \\
  \texttt{extend} & \texttt{count} \\
  \texttt{reverse} & \texttt{upper} \\
  \texttt{sort} & \texttt{isupper} \\
  \texttt{del} (not method) & etc. \\
  \end{tabular}
\end{frame}

%%%%%%%%%%%%%%%%%%%%%%%%%%%%%%%%%%%%%%%%%%%%%%%%%%%%%%%%%%%%%%%%%%%%%%%%%%%%%%%%
\begin{frame}[fragile]
  \frametitle{\texttt{list}s and \texttt{dict}s}
  \Enlarge

  \begin{tabular}{ll}
  \hline\hline \\
  \texttt{dict} & \\
  modifies in place \hspace{2cm} & returns value \\
  \hline \\
  \texttt{del} (not method) & \texttt{values} \\
   & \texttt{keys} \\
  \end{tabular} \pause
  \begin{enumerate}
  \myitem  Note that there isn't a \texttt{sort} method!
  \end{enumerate}
\end{frame}

%%%%%%%%%%%%%%%%%%%%%%%%%%%%%%%%%%%%%%%%%%%%%%%%%%%%%%%%%%%%%%%%%%%%%%%%%%%%%%%%
\begin{frame}[fragile]
  \frametitle{Sorting a \texttt{dict} by value}

  \begin{Verbatim}
# remember me?
def sortDictAsList( d ):
    items = list( d.items() )
    items.sort( key=lambda x:x[1] )
    return items

d = { 'a':2, 'b':1, 'c':-1, 'd':14 }
sortDictAsList( d )
  \end{Verbatim}
\end{frame}

%%%%%%%%%%%%%%%%%%%%%%%%%%%%%%%%%%%%%%%%%%%%%%%%%%%%%%%%%%%%%%%%%%%%%%%%%%%%%%%%
\begin{frame}[fragile]
  \frametitle{Sorting a \texttt{dict} by value}
  \Enlarge

  We want to know which plankton species has the largest population. \pause

  \begin{Verbatim}
from csv import DictReader
reader = DictReader( open( 'plankton.csv' ) )
plankdata = {}
for row in reader:
    plankdata[ row['Species'] ] = \
        max( float( row['Near-shore, May-93']),
             float( row['Near-shore, Aug-93']),
             float( row['Off-shore, May-93']),
             float( row['Off-shore, Aug-93']) )
sortDictAsList( plankdata )
  \end{Verbatim}
\end{frame}

%%%%%%%%%%%%%%%%%%%%%%%%%%%%%%%%%%%%%%%%%%%%%%%%%%%%%%%%%%%%%%%%%%%%%%%%%%%%%%%%
\begin{frame}[fragile]
  \frametitle{Accessing \texttt{list}s}
  %\Enlarge

  \begin{enumerate}
  \myitem  Sometimes we have two \texttt{list}s that correspond to each other.
  \myitem  If we want to loop over both together, we have two approaches open:
  \end{enumerate}
  \begin{Verbatim}
qs = [ 'name', 'quest', 'favourite colour' ]
as = [ 'Lancelot', 'the Holy Grail', 'blue' ]
# method 1:
for i in range(len(qs)):
    print( 'What is your %s?  It is %s.'%(qs[i],as[i]) )
  \end{Verbatim}
  \pause
  \begin{Verbatim}
  # method 2:
  for q,a in zip(qs,as):
      print( 'What is your %s?  It is %s.'%(q,a) )
  \end{Verbatim}
\end{frame}

%%%%%%%%%%%%%%%%%%%%%%%%%%%%%%%%%%%%%%%%%%%%%%%%%%%%%%%%%%%%%%%%%%%%%%%%%%%%%%%%
\begin{frame}[fragile]
  \frametitle{Accessing \texttt{list}s}
  \Enlarge

  \begin{enumerate}
  \myitem  \texttt{zip} makes two \texttt{list}s \emph{jointly iterable}.
  \myitem  Consider a function which compares two \texttt{list}s of measurements and determines for each pair of measurements which is larger:
  \end{enumerate}
  \begin{Verbatim}
def pickLarger( a,b ):
    result = [ ]  # a list of largest values
    for i,j in zip(a,b):
        result.append( max( i,j ) )
    return result
  \end{Verbatim}
\end{frame}

%%%%%%%%%%%%%%%%%%%%%%%%%%%%%%%%%%%%%%%%%%%%%%%%%%%%%%%%%%%%%%%%%%%%%%%%%%%%%%%%
\begin{frame}[fragile]
  \frametitle{Accessing \texttt{list}s}
  \Enlarge

  \begin{enumerate}
  \myitem  What if you need to know both an item and the index of the item?
  \end{enumerate}
  \begin{Verbatim}
my_list = [ 'meter', 'kilogram', 'second' ]
# one way
for i in range( len(my_list) ):
    print( '%s is the %sth item.' % (my_list[i],i) )
  \end{Verbatim}
  \pause
  \begin{Verbatim}
# another way
for i,item in enumerate( my_list ):
    print( '%s is the %sth item.' % (item,i) )
  \end{Verbatim}
\end{frame}

%%%%%%%%%%%%%%%%%%%%%%%%%%%%%%%%%%%%%%%%%%%%%%%%%%%%%%%%%%%%%%%%%%%%%%%%%%%%%%%%
\begin{frame}[fragile]
  \frametitle{Accessing \texttt{list}s}
  \Enlarge

  \begin{enumerate}
  \myitem  Both \texttt{zip} and \texttt{enumerate} are \emph{convenience} functions!
  \myitem  There are multiple approaches!
  \end{enumerate}
\end{frame}

%%%%%%%%%%%%%%%%%%%%%%%%%%%%%%%%%%%%%%%%%%%%%%%%%%%%%%%%%%%%%%%%%%%%%%%%%%%%%%%%
\begin{frame}[fragile]
  \frametitle{Accessing \texttt{list}s}
  %\Enlarge

  \begin{enumerate}
  \myitem  \emph{Permutations} are used in statistics to analyze all possible configurations of a group of things.
  \myitem  In engineering, for instance, you see them used in experimental design.
  \end{enumerate}
  \begin{Verbatim}
# one way
for i in 'ABCD':
    for j in 'ABCD':
        if i == j:
            continue
        print( i, j )
  \end{Verbatim}
  \pause
  \begin{Verbatim}
# another way
from itertools import permutations
for doublet in permutations( 'ABCD',2 ):
    print( doublet )
  \end{Verbatim}
\end{frame}

%%%%%%%%%%%%%%%%%%%%%%%%%%%%%%%%%%%%%%%%%%%%%%%%%%%%%%%%%%%%%%%%%%%%%%%%%%%%%%%%
\begin{frame}[fragile]
  \frametitle{Homework debugging}
  \Enlarge

  \begin{Verbatim}
# how to figure out what directory Python is in
import os
os.getcwd()  # Get Current Working Directory

# how to figure out what's in that directory
os.listdir('.')

# when submitting, use:
open( 'batting.csv' )  #(since in same dir)
  \end{Verbatim}
\end{frame}

%%%%%%%%%%%%%%%%%%%%%%%%%%%%%%%%%%%%%%%%%%%%%%%%%%%%%%%%%%%%%%%%%%%%%%%%%%%%%%%%
\section{Reminders}

%%%%%%%%%%%%%%%%%%%%%%%%%%%%%%%%%%%%%%%%%%%%%%%%%%%%%%%%%%%%%%%%%%%%%%%%%%%%%%%%
\begin{frame}
  \frametitle{Reminders}
  \Enlarge

  \begin{itemize}
  \myitem  Homework \#7 is due Friday, Nov.\ 25.
  \mysubitem  Use the \texttt{split(',')} approach.
  \myitem  Midterm reflection exercise on website for extra credit.
  \end{itemize}
\end{frame}

\end{document}

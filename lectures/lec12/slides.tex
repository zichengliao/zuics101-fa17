%!TEX program = xelatex
\documentclass[11pt]{beamer}

\usepackage{amsfonts}
\usepackage{amsmath}
\usepackage{blindtext}
\usepackage{enumitem}
\usepackage{fancyvrb}

\usetheme{SaoPaulo}

\title{Python Applications}
\subtitle{workflow, data sources, \texttt{requests}}
\author{CS101 Lecture \#12}
\date{2016-11-09}

\setcounter{showSlideNumbers}{1}

\begin{document}
  \setcounter{showProgressBar}{0}
  \setcounter{showSlideNumbers}{0}

%%%%%%%%%%%%%%%%%%%%%%%%%%%%%%%%%%%%%%%%%%%%%%%%%%%%%%%%%%%%%%%%%%%%%%%%%%%%%%%%
\frame{\titlepage}

%%%%%%%%%%%%%%%%%%%%%%%%%%%%%%%%%%%%%%%%%%%%%%%%%%%%%%%%%%%%%%%%%%%%%%%%%%%%%%%%
\setcounter{framenumber}{0}
\setcounter{showProgressBar}{1}
\setcounter{showSlideNumbers}{1}

%%%%%%%%%%%%%%%%%%%%%%%%%%%%%%%%%%%%%%%%%%%%%%%%%%%%%%%%%%%%%%%%%%%%%%%%%%%%%%%%
\section{Administrivia}

%%%%%%%%%%%%%%%%%%%%%%%%%%%%%%%%%%%%%%%%%%%%%%%%%%%%%%%%%%%%%%%%%%%%%%%%%%%%%%%%
\begin{frame}
  \frametitle{Administrivia}
  \Enlarge

  \begin{itemize}
 % \myitem  Midterm \& feedback will be discussed next week.
  \myitem  Homework \#6 is due Weds, Nov.\ 16.
  \myitem  Homework \#7 is due Friday, Nov.\ 25.
  \end{itemize}
\end{frame}


%%%%%%%%%%%%%%%%%%%%%%%%%%%%%%%%%%%%%%%%%%%%%%%%%%%%%%%%%%%%%%%%%%%%%%%%%%%%%%%%
\section{Workflow}

%%%%%%%%%%%%%%%%%%%%%%%%%%%%%%%%%%%%%%%%%%%%%%%%%%%%%%%%%%%%%%%%%%%%%%%%%%%%%%%%
\begin{frame}[fragile]
  \frametitle{Imperative programming}
  \Enlarge

  \begin{itemize}
  \myitem  Every program tells a story. %\pause
    \begin{itemize}
    \mysubitem  Beginning %\pause = Input    = Preprocessing %\pause
    \mysubitem  Middle %\pause    = Analysis = Processing %\pause
    \mysubitem  End %\pause       = Output   = Postprocessing %\pause
    \end{itemize}
  \myitem  A good way to write a program is to make this explicit! %\pause
  \myitem  Everything else we do in this class will follow this pattern.
  \end{itemize}
\end{frame}

%%%%%%%%%%%%%%%%%%%%%%%%%%%%%%%%%%%%%%%%%%%%%%%%%%%%%%%%%%%%%%%%%%%%%%%%%%%%%%%%
\begin{frame}[fragile]
  \frametitle{Imperative programming}
  \Enlarge

  \begin{itemize}
  \myitem  This structure applies at every level.
    \begin{itemize}
    \mysubitem  expressions \\
    \mysubitem  statements \\
    \mysubitem  blocks \\
    \mysubitem  programs
    \end{itemize} %\pause
  \myitem  This is one reason why return type is so critical!
  \end{itemize}
\end{frame}

%%%%%%%%%%%%%%%%%%%%%%%%%%%%%%%%%%%%%%%%%%%%%%%%%%%%%%%%%%%%%%%%%%%%%%%%%%%%%%%%
\section{Input Sources}

%%%%%%%%%%%%%%%%%%%%%%%%%%%%%%%%%%%%%%%%%%%%%%%%%%%%%%%%%%%%%%%%%%%%%%%%%%%%%%%%
\begin{frame}[fragile]
  \frametitle{Input sources}
  \Enlarge

  \begin{itemize}
  \myitem  The user:  %\pause \texttt{input} %\pause
  \myitem  The hard drive:  %\pause \texttt{open} (files) %\pause
    \begin{itemize}
    \mysubitem  plain text files \\
    \mysubitem  comma-separated value files (\texttt{csv}) \\
    \end{itemize} %\pause
  \myitem  The Internet:  %\pause \texttt{requests}
  \end{itemize}
\end{frame}


%%%%%%%%%%%%%%%%%%%%%%%%%%%%%%%%%%%%%%%%%%%%%%%%%%%%%%%%%%%%%%%%%%%%%%%%%%%%%%%%
\begin{frame}[fragile]
  \frametitle{Review:  User \texttt{input}}
  \Enlarge

  \begin{itemize}
  \myitem  \texttt{input}:
    \begin{itemize}
    \mysubitem  accepts as argument a message
    \mysubitem  \emph{blocks} (pauses) for the user
    \mysubitem  \texttt{return}s a \texttt{str}ing
    \end{itemize}
  \end{itemize}
\end{frame}

%%%%%%%%%%%%%%%%%%%%%%%%%%%%%%%%%%%%%%%%%%%%%%%%%%%%%%%%%%%%%%%%%%%%%%%%%%%%%%%%
\begin{frame}[fragile]
  \frametitle{Review:  Files/\texttt{open}}
  \Enlarge

  \begin{itemize}
  \myitem  \texttt{open}:
    \begin{itemize}
    \mysubitem  accepts as argument a file name
    \mysubitem  \texttt{return}s a \texttt{file} data type
    \end{itemize}
    \myitem  \texttt{file} has three useful methods: %\pause
      \begin{itemize}
      \mysubitem  \texttt{read} \texttt{return}s a \texttt{str}ing
      \mysubitem  \texttt{readlines} \texttt{return}s a \texttt{list}
      \mysubitem  \texttt{close}
      \end{itemize}
  \end{itemize}
\end{frame}

%%%%%%%%%%%%%%%%%%%%%%%%%%%%%%%%%%%%%%%%%%%%%%%%%%%%%%%%%%%%%%%%%%%%%%%%%%%%%%%%
\begin{frame}[fragile]
  \frametitle{Files/\texttt{csv}}
  \Enlarge

  \begin{itemize}
  \myitem  \texttt{csv} files look like spreadsheets with columns separated by commas.
  \end{itemize}
  \begin{semiverbatim}
Year,Make,Model,Price
2007,Chevrolet,Camaro,5000.00
2010,Ford,F150,8000.00
  \end{semiverbatim}
\end{frame}

%%%%%%%%%%%%%%%%%%%%%%%%%%%%%%%%%%%%%%%%%%%%%%%%%%%%%%%%%%%%%%%%%%%%%%%%%%%%%%%%
\begin{frame}[fragile]
  \frametitle{Example:  \texttt{plankton.csv}}
  \Enlarge

  \begin{itemize}
  \myitem  Given a field report on plankton populations, determine the largest plankton and the most common (at any location and during any season).
  \end{itemize}
\end{frame}

%%%%%%%%%%%%%%%%%%%%%%%%%%%%%%%%%%%%%%%%%%%%%%%%%%%%%%%%%%%%%%%%%%%%%%%%%%%%%%%%
\begin{frame}[fragile]
  \frametitle{Files/\texttt{csv}}
  \Enlarge

  \begin{itemize}
  \myitem  \texttt{csv} files look like spreadsheets with columns separated by commas.
  \end{itemize}
  \begin{semiverbatim}
Year,Make,Model,Price
2007,Chevrolet,Camaro,5000.00
2010,Ford,F150,8000.00
  \end{semiverbatim}
  \begin{itemize}
  \myitem  There are two ways to read them: %\pause
    \begin{itemize}
    \mysubitem  \emph{tokenize} (\texttt{split}) the line into components %\pause
    \mysubitem  use the \texttt{csv.DictReader} tool to access components
    \end{itemize}
  \end{itemize}
\end{frame}

%%%%%%%%%%%%%%%%%%%%%%%%%%%%%%%%%%%%%%%%%%%%%%%%%%%%%%%%%%%%%%%%%%%%%%%%%%%%%%%%
\begin{frame}[fragile]
  \frametitle{Files/\texttt{csv}}
  \Enlarge

  \begin{semiverbatim}
# assuming that we have a file autos.csv
myfile = open( 'autos.csv' )
rows = myfile.readlines()
for row in rows:
    print( row[ 0 ], row[ 1 ] )
  \end{semiverbatim}
\end{frame}

%%%%%%%%%%%%%%%%%%%%%%%%%%%%%%%%%%%%%%%%%%%%%%%%%%%%%%%%%%%%%%%%%%%%%%%%%%%%%%%%
\begin{frame}[fragile]
  \frametitle{Files/\texttt{csv}}
  \Enlarge

  \begin{semiverbatim}
# assuming that we have a file autos.csv
from csv import DictReader
reader = DictReader( open( 'autos.csv' ) )
for row in reader:
    print( row[ 'Make' ], row[ 'Price' ] )
  \end{semiverbatim} %\pause
  \begin{itemize}
  \myitem  So how would our \texttt{plankton.csv} example look?
  \end{itemize}
\end{frame}

%%%%%%%%%%%%%%%%%%%%%%%%%%%%%%%%%%%%%%%%%%%%%%%%%%%%%%%%%%%%%%%%%%%%%%%%%%%%%%%%
\begin{frame}[fragile]
  \frametitle{Review:  Internet data/\texttt{requests}}
  \Enlarge

  \begin{itemize}
  \myitem  \texttt{requests} is a module to \\ access server-based resources %\pause
    \begin{itemize}
    \mysubitem  This is a complex process!
    \mysubitem  \texttt{get} \texttt{return}s a \texttt{Response} data type \\
    (but you don't need to know this) %\pause
    \mysubitem  The ONLY thing you need is \\ the \texttt{text} attribute (NOT method).
    \end{itemize}
  \end{itemize}
\end{frame}

%%%%%%%%%%%%%%%%%%%%%%%%%%%%%%%%%%%%%%%%%%%%%%%%%%%%%%%%%%%%%%%%%%%%%%%%%%%%%%%%
\begin{frame}[fragile]
  \frametitle{Internet data/\texttt{requests}}
  \Enlarge

  \begin{itemize}
  \myitem  The \texttt{text} attribute is a \texttt{str}ing. %\pause
  \myitem  But websites are HTML! %\pause
    \begin{itemize}
    \mysubitem  We will only access plain-text resources. %\pause
    \mysubitem  HTML requires \emph{parsing}, which we won't cover. %\pause
    \mysubitem  Another possible approach is to inspect the page for structure.
    \end{itemize}
  \end{itemize}
\end{frame}

%%%%%%%%%%%%%%%%%%%%%%%%%%%%%%%%%%%%%%%%%%%%%%%%%%%%%%%%%%%%%%%%%%%%%%%%%%%%%%%%
\begin{frame}[fragile]
  \frametitle{Internet data/\texttt{requests}}

  \begin{semiverbatim}
import requests
url = 'http://www.nws.noaa.gov/mdl/gfslamp/lavlamp.shtml'
website = requests.get( url )
offset = website.text.find( 'KCMI' )+169
temperature_string = website.text[ offset:offset+3 ]
temperature = float( temperature_string )
  \end{semiverbatim}
\end{frame}

%%%%%%%%%%%%%%%%%%%%%%%%%%%%%%%%%%%%%%%%%%%%%%%%%%%%%%%%%%%%%%%%%%%%%%%%%%%%%%%%
\begin{frame}[fragile]
  \frametitle{Question}
  \Enlarge
%[commandchars=\\\{\},commentchar=\%]
  \begin{semiverbatim}
import requests
text = requests.get( 'mydataurl.com/data' )
data = ???
  \end{semiverbatim}
  This code should produce a \texttt{list} containing the comma-separated numbers at the URL.  What should replace the ??? ?
  \begin{enumerate}[label=\Alph*]
  \item  \texttt{text.split(',')}
  \item  \texttt{text.text.split(',')}
  \item  \texttt{text().split(',')}
  \item  \texttt{text.text().split(',')}
  \end{enumerate}
\end{frame}

%%%%%%%%%%%%%%%%%%%%%%%%%%%%%%%%%%%%%%%%%%%%%%%%%%%%%%%%%%%%%%%%%%%%%%%%%%%%%%%%
\begin{frame}[fragile]
  \frametitle{Question}
  \Enlarge
%[commandchars=\\\{\},commentchar=\%]
  \begin{semiverbatim}
import requests
text = requests.get( 'mydataurl.com/data' )
data = \textcolor{CS101Alt}{text.text.split(',')}
  \end{semiverbatim}
  This code should produce a \texttt{list} containing the comma-separated numbers at the URL.  What should replace the ??? ?
  \begin{enumerate}[label=\Alph*]
  \item  \texttt{text.split(',')}
  \item  \texttt{text.text.split(',')}  $\star$
  \item  \texttt{text().split(',')}
  \item  \texttt{text.text().split(',')}
  \end{enumerate}
\end{frame}

%%%%%%%%%%%%%%%%%%%%%%%%%%%%%%%%%%%%%%%%%%%%%%%%%%%%%%%%%%%%%%%%%%%%%%%%%%%%%%%%
\begin{frame}[fragile]
  \frametitle{Sorting a \texttt{dict} by value}

  \begin{semiverbatim}
def sortDictAsList( d ):
    items = list( d.items() )
    items.sort( key=lambda x:x[1] )
    return items
  \end{semiverbatim}
  This is MAGIC.  Don't worry AT ALL about understanding it in 101. \\
  \begin{semiverbatim}
d = \{ 'a':2, 'b':1, 'c':-1, 'd':14 \}
sortDictAsList( d )
  \end{semiverbatim}
\end{frame}

%%%%%%%%%%%%%%%%%%%%%%%%%%%%%%%%%%%%%%%%%%%%%%%%%%%%%%%%%%%%%%%%%%%%%%%%%%%%%%%%
\begin{frame}[fragile]
  \frametitle{Sorting a \texttt{dict} by value}

  Given a dictionary \texttt{d}, create a new dictionary that reverses the keys and values of \texttt{d}.  Thus, the keys of \texttt{d} become the values of the new dictionary and the values of \texttt{d} become the keys of the new dictionary.  You may assume \texttt{d} contains no duplicate values (that is, no two keys map to the same values).  Associate the new dictionary with the variable \texttt{inverse}.
\end{frame}

%%%%%%%%%%%%%%%%%%%%%%%%%%%%%%%%%%%%%%%%%%%%%%%%%%%%%%%%%%%%%%%%%%%%%%%%%%%%%%%%
\section{Reminders}

%%%%%%%%%%%%%%%%%%%%%%%%%%%%%%%%%%%%%%%%%%%%%%%%%%%%%%%%%%%%%%%%%%%%%%%%%%%%%%%%
\begin{frame}
  \frametitle{Reminders}
  \Enlarge

  \begin{itemize}
  \myitem  Homework \#6 is due Weds, Nov.\ 16.
  \myitem  Homework \#7 is due Friday, Nov.\ 25.
  \mysubitem  Use the \texttt{read().split(',')} approach.
  \end{itemize}
\end{frame}

\end{document}

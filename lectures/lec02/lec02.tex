%!TEX program = xelatex
\documentclass[11pt]{beamer}

\usepackage{amsmath}
\usepackage{blindtext}
\usepackage{enumitem}

\usetheme{SaoPaulo}  

\title{Python Basics!}
\subtitle{operators, expressions, computing}
\author{CS101 Lecture \#2}
\date{2016-09-28}

\setcounter{showSlideNumbers}{1}

\begin{document}
  \setcounter{showProgressBar}{0}
  \setcounter{showSlideNumbers}{0}

%%%%%%%%%%%%%%%%%%%%%%%%%%%%%%%%%%%%%%%%%%%%%%%%%%%%%%%%%%%%%%%%%%%%%%%%%%%%%%%%
\frame{\titlepage}

%%%%%%%%%%%%%%%%%%%%%%%%%%%%%%%%%%%%%%%%%%%%%%%%%%%%%%%%%%%%%%%%%%%%%%%%%%%%%%%%
\setcounter{framenumber}{0}
\setcounter{showProgressBar}{1}
\setcounter{showSlideNumbers}{1}

\iffalse
%%%%%%%%%%%%%%%%%%%%%%%%%%%%%%%%%%%%%%%%%%%%%%%%%%%%%%%%%%%%%%%%%%%%%%%%%%%%%%%%
\section{Administrivia}

%%%%%%%%%%%%%%%%%%%%%%%%%%%%%%%%%%%%%%%%%%%%%%%%%%%%%%%%%%%%%%%%%%%%%%%%%%%%%%%%
\begin{frame}
  \frametitle{Administrivia}
  \Enlarge
  \begin{itemize}
%  \myitem  Register your i>clickers on the course Compass page.
  %\myitem  
  \item Complete homework before THIS Friday at 6:00 p.m.
  \end{itemize}
\end{frame}
\fi

%%%%%%%%%%%%%%%%%%%%%%%%%%%%%%%%%%%%%%%%%%%%%%%%%%%%%%%%%%%%%%%%%%%%%%%%%%%%%%%%
\section{Warmup Quiz}

%%%%%%%%%%%%%%%%%%%%%%%%%%%%%%%%%%%%%%%%%%%%%%%%%%%%%%%%%%%%%%%%%%%%%%%%%%%%%%%%
\begin{frame}
  \frametitle{Question \#1}
  \Enlarge

  A set of instructions executed by a computer to achieve a goal is called:
  \begin{enumerate}[label=\Alph*]
  \item  a process
  \item  a program
  \item  a procedure
  \item  an algorithm
  \end{enumerate}
\end{frame}

%%%%%%%%%%%%%%%%%%%%%%%%%%%%%%%%%%%%%%%%%%%%%%%%%%%%%%%%%%%%%%%%%%%%%%%%%%%%%%%%
\begin{frame}
  \frametitle{Question \#2}
  \Enlarge

  A group of eight bits is called:
  \begin{enumerate}[label=\Alph*]
  \item  a nybble
  \item  a gobble
  \item  a byte
  \item  a bite
  \end{enumerate}
\end{frame}

%%%%%%%%%%%%%%%%%%%%%%%%%%%%%%%%%%%%%%%%%%%%%%%%%%%%%%%%%%%%%%%%%%%%%%%%%%%%%%%%
\begin{frame}
  \frametitle{Question \#3}
  \Enlarge

  Python is:
  \begin{enumerate}[label=\Alph*]
  \item  a high-level language
  \item  a low-level language
  \end{enumerate}
\end{frame}

%%%%%%%%%%%%%%%%%%%%%%%%%%%%%%%%%%%%%%%%%%%%%%%%%%%%%%%%%%%%%%%%%%%%%%%%%%%%%%%%
\begin{frame}
  \frametitle{Question \#4}
  \Enlarge

  Python is:
  \begin{enumerate}[label=\Alph*]
  \item  an interpreted language
  \item  a compiled language
  \end{enumerate}
\end{frame}

%%%%%%%%%%%%%%%%%%%%%%%%%%%%%%%%%%%%%%%%%%%%%%%%%%%%%%%%%%%%%%%%%%%%%%%%%%%%%%%%
\begin{frame}
  \frametitle{Question \#4}
  \Enlarge

  Python is:
  \begin{enumerate}[label=\Alph*]
  \item  \emph{an interpreted language}
  \item  a compiled language
  \end{enumerate}
\end{frame}

%%%%%%%%%%%%%%%%%%%%%%%%%%%%%%%%%%%%%%%%%%%%%%%%%%%%%%%%%%%%%%%%%%%%%%%%%%%%%%%%
\begin{frame}
  \frametitle{Interpreted vs Compiled}
  \Enlarge

  \begin{itemize}
    \item A \emph{compiled} language has a compiler to compile its code into machine instructions, e.g., C, C++\\
    
    \item A \emph{interpreted} language executes its program directly -- one line at a time,  e.g., python, matlab
  \end{itemize}

\end{frame}


%%%%%%%%%%%%%%%%%%%%%%%%%%%%%%%%%%%%%%%%%%%%%%%%%%%%%%%%%%%%%%%%%%%%%%%%%%%%%%%%
\section{Elements of Programming}


%%%%%%%%%%%%%%%%%%%%%%%%%%%%%%%%%%%%%%%%%%%%%%%%%%%%%%%%%%%%%%%%%%%%%%%%%%%%%%%%
\begin{frame}
  \frametitle{Literal}
  	{\Enlarge What is a literal? }\\ \pause
	A fixed value, e.g. 5, 'apple'
\end{frame}


\begin{frame}
  \frametitle{Executing a literal?}
  \includegraphics[width=\textwidth]{./img/computer.png}
\end{frame}

%%%%%%%%%%%%%%%%%%%%%%%%%%%%%%%%%%%%%%%%%%%%%%%%%%%%%%%%%%%%%%%%%%%%%%%%%%%%%%%%
\begin{frame}
  \frametitle{Executing a literal?}
  \includegraphics[width=\textwidth]{./img/computer-literal-0.png}
\end{frame}

%%%%%%%%%%%%%%%%%%%%%%%%%%%%%%%%%%%%%%%%%%%%%%%%%%%%%%%%%%%%%%%%%%%%%%%%%%%%%%%%
\begin{frame}
  \frametitle{Executing a literal?}
  \includegraphics[width=\textwidth]{./img/computer-literal-1.png}
\end{frame}

%%%%%%%%%%%%%%%%%%%%%%%%%%%%%%%%%%%%%%%%%%%%%%%%%%%%%%%%%%%%%%%%%%%%%%%%%%%%%%%%
\begin{frame}
  \frametitle{What is an \textbf{operator}?}
  \Enlarge

  \begin{itemize}
  \myitem  Manipulates data (verb)
  \myitem \texttt{+},\texttt{-},\texttt{*},\texttt{/}
  \end{itemize}
\end{frame}

%%%%%%%%%%%%%%%%%%%%%%%%%%%%%%%%%%%%%%%%%%%%%%%%%%%%%%%%%%%%%%%%%%%%%%%%%%%%%%%%
\begin{frame}
  \frametitle{Executing an operator?}
  \includegraphics[width=\textwidth]{./img/computer-operator-0.png}
\end{frame}

%%%%%%%%%%%%%%%%%%%%%%%%%%%%%%%%%%%%%%%%%%%%%%%%%%%%%%%%%%%%%%%%%%%%%%%%%%%%%%%%
\begin{frame}
  \frametitle{It needs operands to make sense!}
  \includegraphics[width=\textwidth]{./img/computer-operator-1.png}
\end{frame}

%%%%%%%%%%%%%%%%%%%%%%%%%%%%%%%%%%%%%%%%%%%%%%%%%%%%%%%%%%%%%%%%%%%%%%%%%%%%%%%%
\begin{frame}
  \frametitle{What is an \textbf{expression}?}
  \Enlarge

  \begin{itemize}
  \myitem  The combination of operands and operators \pause
  \myitem  Produce a new value (another literal)
    \begin{itemize}
    \mysubitem  \texttt{3 * 5}
    \mysubitem  \texttt{100 - 23}
    \end{itemize}
  \end{itemize}
\end{frame}

%%%%%%%%%%%%%%%%%%%%%%%%%%%%%%%%%%%%%%%%%%%%%%%%%%%%%%%%%%%%%%%%%%%%%%%%%%%%%%%%
\begin{frame}
  \frametitle{Executing an expression?}
  \includegraphics[width=\textwidth]{./img/computer-expression-0.png}
\end{frame}

%%%%%%%%%%%%%%%%%%%%%%%%%%%%%%%%%%%%%%%%%%%%%%%%%%%%%%%%%%%%%%%%%%%%%%%%%%%%%%%%
\begin{frame}
  \frametitle{Executing an expression?}
  \includegraphics[width=\textwidth]{./img/computer-expression-1.png}
\end{frame}

%%%%%%%%%%%%%%%%%%%%%%%%%%%%%%%%%%%%%%%%%%%%%%%%%%%%%%%%%%%%%%%%%%%%%%%%%%%%%%%%
\begin{frame}
  \frametitle{What is an \textbf{expression}?}
  \Enlarge

  \begin{itemize}
  \myitem  Can be arbitrarily complicated
    \begin{itemize}
    \mysubitem  \texttt{3 + 8*5 + 4 - 7/100}
    \end{itemize}
  \end{itemize}
\end{frame}

%%%%%%%%%%%%%%%%%%%%%%%%%%%%%%%%%%%%%%%%%%%%%%%%%%%%%%%%%%%%%%%%%%%%%%%%%%%%%%%%
\begin{frame}
  \frametitle{Order of Operator}
  \Enlarge

  (also called: Operator Precedence)\\
  $\mathtt{1 + 1 * 2} \overset{?}{=}$
  \begin{enumerate}[label=\Alph*]
  \item  4
  \item  3
  \item  Something else
  \end{enumerate}
\end{frame}


%%%%%%%%%%%%%%%%%%%%%%%%%%%%%%%%%%%%%%%%%%%%%%%%%%%%%%%%%%%%%%%%%%%%%%%%%%%%%%%%
\begin{frame}
  \frametitle{Change order of operator by parentheses}
  \Enlarge

  $\mathtt{(1 + 1) * 2} \overset{?}{=}$
  \begin{enumerate}[label=\Alph*]
  \item  4
  \item  3
  \item  Something else
  \end{enumerate}
\end{frame}

%%%%%%%%%%%%%%%%%%%%%%%%%%%%%%%%%%%%%%%%%%%%%%%%%%%%%%%%%%%%%%%%%%%%%%%%%%%%%%%%
\begin{frame}
  \frametitle{Question}
  \Enlarge

  $\mathtt{23 + 6 \texttt{/}  2 - 4} \overset{?}{=}$
  \begin{enumerate}[label=\Alph*]
  \item  22
  \item  -1
  \item  35
  \item  Something else
  \end{enumerate}
\end{frame}


%%%%%%%%%%%%%%%%%%%%%%%%%%%%%%%%%%%%%%%%%%%%%%%%%%%%%%%%%%%%%%%%%%%%%%%%%%%%%%%%
\begin{frame}
  \frametitle{Use parentheses!}
  \Enlarge

  $\mathtt{23 + (6 \texttt{/} 2) - 4}$ is always clearer.
\end{frame}


\iffalse
%%%%%%%%%%%%%%%%%%%%%%%%%%%%%%%%%%%%%%%%%%%%%%%%%%%%%%%%%%%%%%%%%%%%%%%%%%%%%%%%
\begin{frame}
  \frametitle{Use parentheses!}
  \Enlarge
  
  $\mathtt{23 + \left(6 / 3\right) - 4}$ is always clearer.
  %\begin{itemize}
  %\myitem $\mathtt{23 + \left(6 / 2\right) - 4}$ is always clearer. \pause
  %\myitem avoid mistake when expression gets complicated \pause
  %\myitem readability!
  %\end{itemize}
\end{frame}
\fi

%%%%%%%%%%%%%%%%%%%%%%%%%%%%%%%%%%%%%%%%%%%%%%%%%%%%%%%%%%%%%%%%%%%%%%%%%%%%%%%%
\begin{frame}
  \frametitle{What are some other operators?}
  \Enlarge

  \begin{itemize}
  \myitem  exponentiation, \texttt{**} \pause
  \myitem  modulus, \texttt{\%}  (important) \pause
  	\begin{itemize} 
	%\mysubitem a\%b: reminder of a\texttt{/}b \pause
	\mysubitem 5\texttt{\%}2 = 1 \pause
	\end{itemize}
  \myitem  floor division, \texttt{//}\pause
        \begin{itemize} 
	\mysubitem 5\texttt{//}2 = 2 \pause
	\mysubitem (-5)\texttt{//}2 = ? 
	%\mysubitem a\texttt{\%}b = a - (a\texttt{//}b)\texttt{*}b
	\end{itemize}
  \end{itemize}
\end{frame}

%%%%%%%%%%%%%%%%%%%%%%%%%%%%%%%%%%%%%%%%%%%%%%%%%%%%%%%%%%%%%%%%%%%%%%%%%%%%%%%%
\begin{frame}
  \frametitle{What are some other operators?}
  \Enlarge

  \begin{itemize}
  \myitem  bitwise \texttt{OR}, \texttt{|}
  \myitem  bitwise \texttt{AND}, \texttt{\&}
  \myitem  bitwise \texttt{XOR}, \texttt{\^{}}
  \end{itemize}
  
  \includegraphics[width=0.8\textwidth]{./img/logictable.png}
\end{frame}

%%%%%%%%%%%%%%%%%%%%%%%%%%%%%%%%%%%%%%%%%%%%%%%%%%%%%%%%%%%%%%%%%%%%%%%%%%%%%%%%
\begin{frame}
  \frametitle{Example}
  \Enlarge

  \texttt{1 \^{} 2} $\overset{?}{=}$
  \\
  \begin{enumerate}[label=\Alph*]
  \item  0
  \item  1
  \item  2
  \item  3
  \end{enumerate}
\end{frame}


%%%%%%%%%%%%%%%%%%%%%%%%%%%%%%%%%%%%%%%%%%%%%%%%%%%%%%%%%%%%%%%%%%%%%%%%%%%%%%%%
\begin{frame}
  \frametitle{What are some other operators?}
  \Enlarge

  \begin{itemize}
  \myitem  bitwise left shift, \texttt{<{}<}
  \myitem  bitwise right shift, \texttt{>{}>}
  \end{itemize}  \pause
  
  \hspace{5mm} 00001111 \texttt{<{}<} 1 = 00011110 \\ \pause
  %\hspace{5mm} a \texttt{<{}<} b = a*(2**b) \\ \pause
  \hspace{5mm} 00011110 \texttt{>{}>} 2 = 00000111
  %\hspace{5mm} what does ``a \texttt{>{}>} b'' equal to? \\ \pause
  %\hspace{5mm} a \texttt{>{}>} b = a//(2**b) 
  
\end{frame}

%TODO: make a graphic
%http://www.tutorialspoint.com/python/bitwise_operators_example.htm


%%%%%%%%%%%%%%%%%%%%%%%%%%%%%%%%%%%%%%%%%%%%%%%%%%%%%%%%%%%%%%%%%%%%%%%%%%%%%%%%
\begin{frame}
  \frametitle{So what?}
  \Enlarge

  \begin{itemize}
  \myitem  The machine state hasn't changed. \pause
  \myitem  Programs are complex, and we need to remember results.
  \end{itemize}
\end{frame}

%%%%%%%%%%%%%%%%%%%%%%%%%%%%%%%%%%%%%%%%%%%%%%%%%%%%%%%%%%%%%%%%%%%%%%%%%%%%%%%%
\begin{frame}
  \frametitle{How do we keep values around?}
  \includegraphics[width=\textwidth]{./img/computer-memory.png}
\end{frame}

%%%%%%%%%%%%%%%%%%%%%%%%%%%%%%%%%%%%%%%%%%%%%%%%%%%%%%%%%%%%%%%%%%%%%%%%%%%%%%%%
\begin{frame}
  \frametitle{How do we keep values around?}
  \includegraphics[width=\textwidth]{./img/computer-memory-0.png}
\end{frame}

\iffalse
%%%%%%%%%%%%%%%%%%%%%%%%%%%%%%%%%%%%%%%%%%%%%%%%%%%%%%%%%%%%%%%%%%%%%%%%%%%%%%%%
\begin{frame}
  \frametitle{How do we reuse values?}
  \Enlarge

  \begin{itemize}
  \myitem  Low-level languages refer directly to memory address:
  
  \begin{tabular}{*{27}{l}}
    \texttt{LOAD DATA AT} & \texttt{00001101 01001110} \\
    \texttt{ADD DATA AT}     & \texttt{10101101 11010100} \\
  \end{tabular}
  \end{itemize}
\end{frame}

\fi

%%%%%%%%%%%%%%%%%%%%%%%%%%%%%%%%%%%%%%%%%%%%%%%%%%%%%%%%%%%%%%%%%%%%%%%%%%%%%%%%
\begin{frame}
  \frametitle{What is a \textbf{variable}?}
  \Enlarge

  \begin{itemize}
  \myitem  The solution:  \textcolor{CS101GradBot}{name memory locations!} \pause
  \myitem  Variables name a memory location \pause
  \myitem  Variables store a value \pause
  \myitem  This value can change over time---it is a placeholder.
  \end{itemize}
\end{frame}

%%%%%%%%%%%%%%%%%%%%%%%%%%%%%%%%%%%%%%%%%%%%%%%%%%%%%%%%%%%%%%%%%%%%%%%%%%%%%%%%
\begin{frame}
  \frametitle{What new operator do we need?}
  \Enlarge

  \begin{itemize}
  \myitem  assignment, \texttt{=} (single equals sign)
  \end{itemize}
\end{frame}

%%%%%%%%%%%%%%%%%%%%%%%%%%%%%%%%%%%%%%%%%%%%%%%%%%%%%%%%%%%%%%%%%%%%%%%%%%%%%%%%
\begin{frame}
  \frametitle{How do we reuse values?}
  \includegraphics[width=\textwidth]{./img/computer-memory-1.png}
\end{frame}

%%%%%%%%%%%%%%%%%%%%%%%%%%%%%%%%%%%%%%%%%%%%%%%%%%%%%%%%%%%%%%%%%%%%%%%%%%%%%%%%
\begin{frame}
  \frametitle{How do we reuse values?}
  \includegraphics[width=\textwidth]{./img/computer-memory-2.png}
\end{frame}

%%%%%%%%%%%%%%%%%%%%%%%%%%%%%%%%%%%%%%%%%%%%%%%%%%%%%%%%%%%%%%%%%%%%%%%%%%%%%%%%
\begin{frame}
  \frametitle{How do we reuse values?}
  \includegraphics[width=\textwidth]{./img/computer-memory-3.png}
\end{frame}

%%%%%%%%%%%%%%%%%%%%%%%%%%%%%%%%%%%%%%%%%%%%%%%%%%%%%%%%%%%%%%%%%%%%%%%%%%%%%%%%
\begin{frame}
  \frametitle{How do we reuse values?}
  \includegraphics[width=\textwidth]{./img/computer-memory-4.png}
\end{frame}

%%%%%%%%%%%%%%%%%%%%%%%%%%%%%%%%%%%%%%%%%%%%%%%%%%%%%%%%%%%%%%%%%%%%%%%%%%%%%%%%
\begin{frame}
  \frametitle{How do we reuse values?}
  \includegraphics[width=\textwidth]{./img/computer-memory-5.png}
\end{frame}

%%%%%%%%%%%%%%%%%%%%%%%%%%%%%%%%%%%%%%%%%%%%%%%%%%%%%%%%%%%%%%%%%%%%%%%%%%%%%%%%
\begin{frame}
  \frametitle{Example}
  \Enlarge

  What value is stored in the variable \texttt{x}? \\
  \texttt{x = 17 + 7*9}
  \begin{enumerate}[label=\Alph*]
  \item  3
  \item  31
  \item  55
  \item  78
  \end{enumerate}
\end{frame}

%%%%%%%%%%%%%%%%%%%%%%%%%%%%%%%%%%%%%%%%%%%%%%%%%%%%%%%%%%%%%%%%%%%%%%%%%%%%%%%%
\begin{frame}
  \frametitle{Example}
  \Enlarge

  What value is stored in the variable \texttt{x}? \\
  \texttt{x = 17 + 7*9} \\
  \texttt{x = 3}
  \begin{enumerate}[label=\Alph*]
  \item  1
  \item  3
  \item  6
  \item Something Else
  \end{enumerate}
\end{frame}

%%%%%%%%%%%%%%%%%%%%%%%%%%%%%%%%%%%%%%%%%%%%%%%%%%%%%%%%%%%%%%%%%%%%%%%%%%%%%%%%
\begin{frame}
  \frametitle{Example}
  \Enlarge

  What value is stored in the variable \texttt{x}? \\
  \texttt{x = 17 + 7*9} \\
  \texttt{x = 3}\\
  \texttt{x + 3}
  \begin{enumerate}[label=\Alph*]
  \item  1
  \item  3
  \item  6
  \item Something Else
  \end{enumerate}
  
\end{frame}


%%%%%%%%%%%%%%%%%%%%%%%%%%%%%%%%%%%%%%%%%%%%%%%%%%%%%%%%%%%%%%%%%%%%%%%%%%%%%%%%
\begin{frame}
  \frametitle{What is a \textbf{statement}?}
  \Enlarge

  \begin{itemize}
  \myitem  A \emph{statement} changes the state of the computer (sentence) \pause
  \myitem  Example:  an assignment
  \end{itemize}
\end{frame}

%%%%%%%%%%%%%%%%%%%%%%%%%%%%%%%%%%%%%%%%%%%%%%%%%%%%%%%%%%%%%%%%%%%%%%%%%%%%%%%%
\begin{frame}
  \frametitle{What is a \textbf{program}?}
  \Enlarge

  \begin{itemize}
  \myitem  Programs consist of series of statements: \pause
    \begin{itemize}
    \mysubitem  A series of python statements is also called a python \emph{script}. \pause
    %\mysubitem  A notebook (as we use in the lab) also collects series of Python statements. \pause
    \mysubitem  A python script is stored in text that you can edit (there’s no magic, just text). \pause
    %\mysubitem There are various IDEs for writing and managing complicated programs (1000+ code files) \pause
    %\mysubitem IDE: integrated development environment for writing programs and manage them \pause
  \end{itemize}
  \myitem  Each statement is executed in order from top to bottom by the python interpreter --��together, these statements  form a program.
  \end{itemize}
 \end{frame}

%%%%%%%%%%%%%%%%%%%%%%%%%%%%%%%%%%%%%%%%%%%%%%%%%%%%%%%%%%%%%%%%%%%%%%%%%%%%%%%%
\begin{frame}[fragile]
  \frametitle{Our first program}
  \Enlarge

  \begin{semiverbatim}
x = 10
y = x ** 2
y = y + y
  \end{semiverbatim} \pause
  
  {\small
  Save the code to file: my\_first\_script.py \\
  Run from your command line window (Linux or iOS, Windows also OK):\\ >> python my\_first\_script.py }
\end{frame}



%%%%%%%%%%%%%%%%%%%%%%%%%%%%%%%%%%%%%%%%%%%%%%%%%%%%%%%%%%%%%%%%%%%%%%%%%%%%%%%%
\section{Reminders}

%%%%%%%%%%%%%%%%%%%%%%%%%%%%%%%%%%%%%%%%%%%%%%%%%%%%%%%%%%%%%%%%%%%%%%%%%%%%%%%%
\begin{frame}
  \frametitle{Reminders}
  \Enlarge

  \begin{itemize}
  %\myitem  Register your i>clicker on Compass.
  %\myitem  
  \item Homework \#1 due Monday, Sep.\ 25, 6:00 p.m.
  \end{itemize}
\end{frame}

\end{document}

\documentclass{article}
\usepackage{amsmath,amssymb,listings,upquote}
\usepackage[margin=3cm]{geometry}
\usepackage{graphicx,color}
\lstset{language=Python}
\usepackage{enumerate}% http://ctan.org/pkg/enumerate
\usepackage{fancyvrb}


\newcounter{zone}
\setcounter{zone}{0}
\newcommand{\zone}{\clearpage\refstepcounter{zone}\section*{Zone \arabic{zone}}}
\newcounter{question}
\setcounter{question}{0}
\newcounter{variant}
\newcounter{questionpoints}
\newcommand{\question}[1]{\newpage \refstepcounter{question} \setcounter{variant}{0} \setcounter{questionpoints}{#1}}
\newcommand{\variant}{\vspace{4em}\refstepcounter{variant}\noindent \arabic{question}/\arabic{variant}. (\arabic{questionpoints} point\ifnum \thequestionpoints > 1 s\fi) }
\newenvironment{answers}{\begin{enumerate}}{\end{enumerate}}
\newcommand{\answer}{\item }
\newcommand{\correctanswer}{\item $\bigstar$ }
\renewcommand{\theenumi}{\Alph{enumi}}
\newenvironment{solution}{{\bf Solution.} }{\vspace*{.3in}\hrule}

\begin{document}

\begin{center}
%\textbf{\Large CS 101 Final Exam}
%
\includegraphics[width=2in]{../img/final-header.png}
\end{center}

\bigskip
\noindent
\begin{itemize}
\item  \textbf{Be sure to enter your \underline{NetID} and \underline{the code below} on your Scantron}.  Do not turn this page until instructed to.
\item  This is a 120-minute exam with 30 questions:
  \begin{itemize}
  \item  12 MATLAB multiple-choice questions worth 5 points each;
  \item  16 Python multiple-choice questions worth 5 points each; and
  \item  2 coding questions worth 30 points each
  \end{itemize}
  for a total of 200 possible points.
\item  Each multiple choice question has only \emph{one} correct answer.
\item  You must not communicate with other students during the exam.
\item  No books, notes, or electronic devices are permitted during the exam.
\end{itemize}

\bigskip\bigskip
\noindent
\textbf{\Large 1. Fill in your information:}

\bigskip
{\Large\bf
\begin{tabular}{ll}
Full Name: & \underbar{\hskip 8cm} \\[0.5em]
UIN (Student Number): & \underbar{\hskip 8cm} \\[0.5em]
NetID: & \underbar{\hskip 8cm} \\[0.5em]
Lab Section: & \underbar{\hskip 8cm}
\end{tabular}
}

\bigskip
\bigskip
\noindent
\textbf{\Large 2. Fill in the following answers on the Scantron form:}

%%%%%%%%%%%%%%%%%%%%%%%%%%%%%%%%%%%%%%%%%%%%%%%%%%%%%%%%%%%%%%%%%%%%%%%%%%%%%%%
%%%%%%%%%%%%%%%%%%%%%%%%%%%%%%%%%%%%%%%%%%%%%%%%%%%%%%%%%%%%%%%%%%%%%%%%%%%%%%%
\zone \pagebreak \noindent
\textbf{The following 12 questions involve MATLAB.}
\\\\

%%%%%%%%%%%%%%%%%%%%%%%%%%%%%%%%%%%%%%%%%%%%%%%%%%%%%%%%%%%%%%%%%%%%%%%%%%%%%%%
\question{5}
\variant %---------------------------------------------------------------------
Consider the following MATLAB program:
\begin{Verbatim}
x = [ 1 2 ];
y = [ 3 4 ];
z = [ y x ; x y ]';
\end{Verbatim}

What is the \textbf{value} of \texttt{z} after this program executes?

\begin{answers}
  \correctanswer $ \left[ \begin{array}{cc} 3 & 1 \\ 4 & 2 \\ 1 & 3 \\ 2 & 4 \\ \end{array} \right] $
  \answer $ \left[ \begin{array}{cccc} 3 & 4 & 1 & 2 \\ 1 & 2 & 3 & 4 \\ \end{array} \right] $
  \answer $ \left[ \begin{array}{cccc} 1 & 2 & 3 & 4 \\ 3 & 4 & 1 & 2 \\ \end{array} \right] $
  \answer $ \left[ \begin{array}{cc} 1 & 3 \\ 2 & 4 \\ 3 & 1 \\ 4 & 2 \\ \end{array} \right] $
  \answer None of the other answers are correct
\end{answers}

\begin{solution}
\end{solution}

\variant %---------------------------------------------------------------------
Consider the following MATLAB program:
\begin{Verbatim}
x = [ 1 2 ];
y = [ 3 4 ];
z = [ x y ; y x ]';
\end{Verbatim}

What is the \textbf{value} of \texttt{z} after this program executes?

\begin{answers}
  \answer $ \left[ \begin{array}{cc} 3 & 1 \\ 4 & 2 \\ 1 & 3 \\ 2 & 4 \\ \end{array} \right] $
  \answer $ \left[ \begin{array}{cccc} 3 & 4 & 1 & 2 \\ 1 & 2 & 3 & 4 \\ \end{array} \right] $
  \answer $ \left[ \begin{array}{cccc} 1 & 2 & 3 & 4 \\ 3 & 4 & 1 & 2 \\ \end{array} \right] $
  \correctanswer $ \left[ \begin{array}{cc} 1 & 3 \\ 2 & 4 \\ 3 & 1 \\ 4 & 2 \\ \end{array} \right] $
  \answer None of the other answers are correct
\end{answers}

\begin{solution}
\end{solution}


%%%%%%%%%%%%%%%%%%%%%%%%%%%%%%%%%%%%%%%%%%%%%%%%%%%%%%%%%%%%%%%%%%%%%%%%%%%%%%%
\question{5}
\variant %---------------------------------------------------------------------
Consider the following MATLAB program:
\begin{Verbatim}
A = ones( 3,3 ) - eye( 3,3 );
A = A * 2;
A( 1:2,: ) += 3;
\end{Verbatim}

What is the \textbf{value} of \texttt{A} after this program executes?

\begin{answers}
  \correctanswer $ \left[ \begin{array}{ccc} 3 & 5 & 5 \\ 5 & 3 & 5 \\ 2 & 2 & 0 \\ \end{array} \right] $
  \answer $ \left[ \begin{array}{ccc} 3 & 5 & 2 \\ 5 & 3 & 2 \\ 5 & 5 & 0 \\ \end{array} \right] $
  \answer $ \left[ \begin{array}{ccc} 0 & 2 & 2 \\ 5 & 3 & 5 \\ 5 & 5 & 3 \\ \end{array} \right] $
  \answer $ \left[ \begin{array}{ccc} 0 & 5 & 5 \\ 2 & 3 & 5 \\ 2 & 5 & 3 \\ \end{array} \right] $
  \answer None of the other answers are correct
\end{answers}

\begin{solution}
\end{solution}

\variant %---------------------------------------------------------------------
Consider the following MATLAB program:
\begin{Verbatim}
A = ones( 3,3 ) - eye( 3,3 );
A = A * 3;
A( :,1:2 ) = A( :,1:2 ) + 3;
\end{Verbatim}

What is the \textbf{value} of \texttt{A} after this program executes?

\begin{answers}
  \answer $ \left[ \begin{array}{ccc} 3 & 6 & 6 \\ 6 & 3 & 6 \\ 3 & 3 & 0 \\ \end{array} \right] $
  \correctanswer $ \left[ \begin{array}{ccc} 3 & 6 & 3 \\ 6 & 3 & 3 \\ 6 & 6 & 0 \\ \end{array} \right] $
  \answer $ \left[ \begin{array}{ccc} 0 & 3 & 3 \\ 6 & 3 & 6 \\ 6 & 6 & 3 \\ \end{array} \right] $
  \answer $ \left[ \begin{array}{ccc} 0 & 6 & 6 \\ 3 & 3 & 6 \\ 3 & 6 & 3 \\ \end{array} \right] $
  \answer None of the other answers are correct
\end{answers}

\begin{solution}
\end{solution}


%%%%%%%%%%%%%%%%%%%%%%%%%%%%%%%%%%%%%%%%%%%%%%%%%%%%%%%%%%%%%%%%%%%%%%%%%%%%%%%
\question{5}
\variant %---------------------------------------------------------------------
Consider the following MATLAB function stored in \texttt{squrge.m}:
\begin{Verbatim}
function [ a b ] = squrge( x,y )
  a = x .^ 2;
  b = a .* 3 + y;
end
\end{Verbatim}

Which of the following correctly assigns the results of a call to \texttt{squrge} \texttt{a} to \texttt{A} and \texttt{b} to \texttt{B}, respectively?

\begin{answers}
  \answer  \texttt{A,B = squrge( 5,4 );}
  \answer  \texttt{[ A B ] = squrge( [ 5 4 ] );}
  \answer  \texttt{[ A B ] = squrge( 5 4 );}
  \answer  \texttt{[ A B ] = squrge [ 5 4 ];}
  \correctanswer  \texttt{[ A B ] = squrge( 5,4 );}
\end{answers}

\begin{solution}
\end{solution}

\variant %---------------------------------------------------------------------
Consider the following MATLAB function stored in \texttt{splink.m}:
\begin{Verbatim}
function [ a b ] = splink( x,y )
  a = x .^ 3 - y .^ 2;
  b = y / 2 + a;
end
\end{Verbatim}

Which of the following correctly assigns the results of a call to \texttt{splink} \texttt{a} to \texttt{A} and \texttt{b} to \texttt{B}, respectively?

\begin{answers}
  \answer  \texttt{A,B = splink( 5,4 );}
  \answer  \texttt{[ A B ] = splink( [ 5 4 ] );}
  \answer  \texttt{[ A B ] = splink( 5 4 );}
  \answer  \texttt{[ A B ] = splink [ 5 4 ];}
  \correctanswer  \texttt{[ A B ] = splink( 5,4 );}
\end{answers}

\begin{solution}
\end{solution}


%%%%%%%%%%%%%%%%%%%%%%%%%%%%%%%%%%%%%%%%%%%%%%%%%%%%%%%%%%%%%%%%%%%%%%%%%%%%%%%
\question{5}
\variant %---------------------------------------------------------------------
Recollect that MATLAB represents polynomials as an array of coefficients from the highest-order coefficient to the lowest.  For instance,
$$
3 x^{2} + 2 x + 1
$$
is written as the array \texttt{[ 3 2 1 ]}.

How would we represent the summation of the two polynomials
$$
-x^{2} + 3 x + 1
$$
and
$$
2 x^{3} + 4 x - 1
$$
as a MATLAB polynomial array?

\begin{answers}
  \answer  \texttt{[ -1 3 1 ] + [ 2 4 -1 ]}
  \answer  \texttt{[ -1 3 1 ] + [ 2 0 4 -1 ]}
  \correctanswer  \texttt{[ 0 -1 3 1 ] + [ 2 0 4 -1 ]}
  \answer  \texttt{[ 1 3 -1 0 ] + [ -1 4 0 2 ]}
  \answer  \texttt{[ 1 3 -1 ] + [ -1 4 2 ]}
\end{answers}

\begin{solution}
\end{solution}

\variant %---------------------------------------------------------------------
Recollect that MATLAB represents polynomials as an array of coefficients from the highest-order coefficient to the lowest.  For instance,
$$
3 x^{2} + 2 x + 1
$$
is written as the array \texttt{[ 3 2 1 ]}.

How would we represent the difference of the two polynomials
$$
-x^{2} + 3 x + 1
$$
and
$$
x^{3} + 4 x^{2} + 2
$$
as a MATLAB polynomial array?

\begin{answers}
  \correctanswer  \texttt{[ 0 -1 3 1 ] - [ 1 4 0 2 ]}
  \answer  \texttt{[ -1 3 1 ] - [ 2 0 4 1 ]}
  \answer  \texttt{[ 0 -1 3 1 ] - [ 2 0 4 1 ]}
  \answer  \texttt{[ 1 3 -1 0 ] - [ 1 4 0 2 ]}
  \answer  \texttt{[ 1 3 -1 ] - [ 1 4 2 ]}
\end{answers}

\begin{solution}
\end{solution}


%%%%%%%%%%%%%%%%%%%%%%%%%%%%%%%%%%%%%%%%%%%%%%%%%%%%%%%%%%%%%%%%%%%%%%%%%%%%%%%
\question{5}
\variant %---------------------------------------------------------------------
Consider the following two-dimensional MATLAB array, stored in the variable \texttt{A}:
$$
\left[ \begin{array}{ccc}
1 &  16 &  256 \\
2 &  32 &  512 \\
4 &  64 & 1024 \\
8 & 128 & 2048 \\
\end{array} \right]
$$

How can we index and retrieve the value 128 from this array?

\begin{answers}
  \answer  \texttt{A( 2,4 )}
  \correctanswer  \texttt{A( 4,2 )}
  \answer  \texttt{A( 1,3 )}
  \answer  \texttt{A[ 2,4 ]}
  \answer  \texttt{A[ 3,1 ]}
\end{answers}

\begin{solution}
\end{solution}

\variant %---------------------------------------------------------------------
Consider the following two-dimensional MATLAB array, stored in the variable \texttt{A}:
$$
\left[ \begin{array}{ccc}
1 &  16 &  256 \\
2 &  32 &  512 \\
4 &  64 & 1024 \\
8 & 128 & 2048 \\
\end{array} \right]
$$

How can we index and retrieve the value 512 from this array?

\begin{answers}
  \answer  \texttt{A( 3,2 )}
  \correctanswer  \texttt{A( 2,3 )}
  \answer  \texttt{A( 6 )}
  \answer  \texttt{A[ 2,3 ]}
  \answer  \texttt{A[ 1,2 ]}
\end{answers}

\begin{solution}
\end{solution}


%%%%%%%%%%%%%%%%%%%%%%%%%%%%%%%%%%%%%%%%%%%%%%%%%%%%%%%%%%%%%%%%%%%%%%%%%%%%%%%
\question{5}
\variant %---------------------------------------------------------------------
\emph{For this problem, you should compose a function which accomplishes a given task using the available code blocks arranged in the correct functional order.}

Compose a function \texttt{cross\_prod} which accepts two column vectors \texttt{a} and \texttt{b} and returns the value of the cross product,
$$
\vec{c} =
\vec{a} \times \vec{b} =
\begin{bmatrix}
    a_2 b_3 - a_3 b_2
    a_3 b_1 - a_1 b_3
    a_1 b_2 - a_2 b_1
\end{bmatrix}
\text{.}
$$

\begin{enumerate}[1]
\item \texttt{end}
\item \texttt{c(1) = a(2)*b(3) - a(3)*b(2);}
\item \texttt{function [ c ] = cross\_prod( a,b )}
\item \texttt{c(2) = a(3)*b(1) - a(1)*b(3);}
\item \texttt{c = zeros( 3,1 );}
\item \texttt{c(3) = a(1)*b(2) - a(2)*b(1);}
\item \texttt{c = zeros( 1,3 );}
\item \texttt{c = a .* b - b .* a;}
\item \texttt{function cross\_prod( a,b )}
\end{enumerate}

\begin{answers}
  \answer  3, 7, 2, 4, 6, 1
  \correctanswer  3, 5, 2, 4, 6, 1
  \answer  9, 5, 8, 1
  \answer  9, 7, 2, 4, 6, 1
  \answer  3, 7, 8, 1
\end{answers}

\begin{solution}
\end{solution}


%%%%%%%%%%%%%%%%%%%%%%%%%%%%%%%%%%%%%%%%%%%%%%%%%%%%%%%%%%%%%%%%%%%%%%%%%%%%%%%
\question{5}
\variant %---------------------------------------------------------------------
Consider the following MATLAB program:

\begin{Verbatim}
s = (3 < 5) | ((2 > 3) & (1 ~= 0))
\end{Verbatim}

What is the final value of \texttt{s}?

\begin{answers}
  \answer  \texttt{True}
  \correctanswer  \texttt{1}
  \answer  \texttt{0}
  \answer  \texttt{false}
\end{answers}

\begin{solution}
\end{solution}

\variant %---------------------------------------------------------------------
Consider the following MATLAB program:

\begin{Verbatim}
s = (5 < 3) | ((2 > 3) & (1 ~= 0))
\end{Verbatim}

What is the final value of \texttt{s}?

\begin{answers}
  \answer  \texttt{true}
  \answer  \texttt{1}
  \correctanswer  \texttt{0}
  \answer  \texttt{False}
\end{answers}

\begin{solution}
\end{solution}


%%%%%%%%%%%%%%%%%%%%%%%%%%%%%%%%%%%%%%%%%%%%%%%%%%%%%%%%%%%%%%%%%%%%%%%%%%%%%%%
\question{5}
\variant %---------------------------------------------------------------------
\begin{Verbatim}
x = eye( 2,2 );
y = [ x(2,:) ; x(1,:) ];
A = [ x y ; y x ];
\end{Verbatim}

What is the final value of \texttt{A( 2:3,2:3 )}?

\begin{answers}
  \answer  \texttt{[ 0 1 ; 1 0 ]}
  \correctanswer  \texttt{[ 1 1 ; 1 1 ]}
  \answer  \texttt{[ 0 0 ; 0 0 ]}
  \answer  \texttt{[ 1 0 ; 0 1 ]}
\end{answers}

\begin{solution}
\end{solution}

\variant %---------------------------------------------------------------------
\begin{Verbatim}
x = eye( 2,2 );
y = [ x(2,:) ; x(1,:) ];
A = [ y x ; x y ];
\end{Verbatim}

What is the final value of \texttt{A( 2:3,2:3 )}?

\begin{answers}
  \answer  \texttt{[ 0 1 ; 1 0 ]}
  \answer  \texttt{[ 1 1 ; 1 1 ]}
  \correctanswer  \texttt{[ 0 0 ; 0 0 ]}
  \answer  \texttt{[ 1 0 ; 0 1 ]}
\end{answers}

\begin{solution}
\end{solution}


%%%%%%%%%%%%%%%%%%%%%%%%%%%%%%%%%%%%%%%%%%%%%%%%%%%%%%%%%%%%%%%%%%%%%%%%%%%%%%%
\question{5}
\variant %---------------------------------------------------------------------
\begin{Verbatim}
x  = linspace( -10,10,201 );
y1 = sin( x );
y2 = cos( x );
y3 = randn( 1,numel(x) );
\end{Verbatim}

How would you successfully plot all three of these data series as points?  (Assume any given plot format strings are valid.)

\begin{answers}
  \answer
    \begin{Verbatim}
plot( x, y1,'r.', y2,'g.', y3,'b.' );
    \end{Verbatim}
  \answer
    \begin{Verbatim}
plot( x, y1, 'r.' );
plot( x, y2, 'g.' );
plot( x, y3, 'b.' );
    \end{Verbatim}
  \correctanswer
    \begin{Verbatim}
hold on;
plot( x, y1, 'r.' );
plot( x, y2, 'g.' );
plot( x, y3, 'b.' );
    \end{Verbatim}
  \answer
    \begin{Verbatim}
plot( x,y1, x,y2, x,y3 );
    \end{Verbatim}
\end{answers}

\begin{solution}
\end{solution}


%%%%%%%%%%%%%%%%%%%%%%%%%%%%%%%%%%%%%%%%%%%%%%%%%%%%%%%%%%%%%%%%%%%%%%%%%%%%%%%
\question{5}
\variant %---------------------------------------------------------------------
Consider the following plot, produced from 10,000 random numbers selected from an as-yet-undetermined distribution.

\includegraphics[width=0.4\textwidth]{./hist-normal.png}

Which of the following MATLAB programs could produce this plot?  Assume that all programs work as written.

\begin{answers}
  \answer
    \begin{Verbatim}
x = rand( 10000,1 );
plot( x );
    \end{Verbatim}
  \correctanswer
    \begin{Verbatim}
x = randn( 10000,1 );
hist( x );
    \end{Verbatim}
  \answer
    \begin{Verbatim}
x = rand( 10000,1 );
hist( x );
    \end{Verbatim}
  \answer
    \begin{Verbatim}
x = randn( 10000,1 );
plot( x );
    \end{Verbatim}
\end{answers}

\begin{solution}
\end{solution}

\variant %---------------------------------------------------------------------
Consider the following program, which produces 10,000 random numbers selected from a certain distribution and plots them:
\begin{Verbatim}
x = rand( 10000,1 );
plot( x,'k.' );
\end{Verbatim}

Which of the following plots could result from executing this program?

\begin{answers}
  \answer
    \includegraphics[width=0.4\textwidth]{./hist-normal.png}
  \answer
    \includegraphics[width=0.4\textwidth]{./plot-normal.png}
  \correctanswer
    \includegraphics[width=0.4\textwidth]{./plot-uniform.png}
  \answer
    \includegraphics[width=0.4\textwidth]{./plot-integer.png}
\end{answers}

\begin{solution}
\end{solution}


%%%%%%%%%%%%%%%%%%%%%%%%%%%%%%%%%%%%%%%%%%%%%%%%%%%%%%%%%%%%%%%%%%%%%%%%%%%%%%%
\question{5}
\variant %---------------------------------------------------------------------
\begin{Verbatim}
A = eye( 3,3 );
for x = 1:2:3
    A( x,x ) = 0;
end
\end{Verbatim}

What is the final value of \texttt{A}?

\begin{answers}
  \answer $ \left[ \begin{array}{ccc} 1 & 0 & 0 \\ 0 & 1 & 0 \\ 0 & 0 & 1 \\ \end{array} \right] $
  \correctanswer $ \left[ \begin{array}{ccc} 0 & 0 & 0 \\ 0 & 1 & 0 \\ 0 & 0 & 0 \\ \end{array} \right] $
  \answer $ \left[ \begin{array}{ccc} 1 & 0 & 1 \\ 0 & 1 & 0 \\ 1 & 0 & 1 \\ \end{array} \right] $
  \answer $ \left[ \begin{array}{ccc} 0 & 0 & 0 \\ 0 & 0 & 0 \\ 0 & 0 & 0 \\ \end{array} \right] $
  \answer $ \left[ \begin{array}{ccc} 1 & 0 & 0 \\ 0 & 0 & 0 \\ 0 & 0 & 1 \\ \end{array} \right] $
\end{answers}

\begin{solution}
\end{solution}

\variant %---------------------------------------------------------------------
\begin{Verbatim}
A = eye( 3,3 );
for x = 2:1:3
    A( x,x ) = 0;
end
\end{Verbatim}

What is the final value of \texttt{A}?

\begin{answers}
  \answer $ \left[ \begin{array}{ccc} 1 & 0 & 0 \\ 0 & 1 & 0 \\ 0 & 0 & 1 \\ \end{array} \right] $
  \answer $ \left[ \begin{array}{ccc} 0 & 0 & 0 \\ 0 & 1 & 0 \\ 0 & 0 & 0 \\ \end{array} \right] $
  \answer $ \left[ \begin{array}{ccc} 1 & 0 & 1 \\ 0 & 1 & 0 \\ 1 & 0 & 1 \\ \end{array} \right] $
  \answer $ \left[ \begin{array}{ccc} 0 & 0 & 0 \\ 0 & 0 & 0 \\ 0 & 0 & 0 \\ \end{array} \right] $
  \correctanswer $ \left[ \begin{array}{ccc} 1 & 0 & 0 \\ 0 & 0 & 0 \\ 0 & 0 & 0 \\ \end{array} \right] $
\end{answers}

\begin{solution}
\end{solution}


%%%%%%%%%%%%%%%%%%%%%%%%%%%%%%%%%%%%%%%%%%%%%%%%%%%%%%%%%%%%%%%%%%%%%%%%%%%%%%%
\question{5}
\variant %---------------------------------------------------------------------
\begin{Verbatim}
A = eye( 3,3 ) - ones( 3,3 );
for x = 1:3
    for y = 1:3
        if x <= y
            A( x,y ) = x + y;
        end
    end
end
\end{Verbatim}

What is the final value of \texttt{A}?

\begin{answers}
  \answer $ \left[ \begin{array}{ccc} 2 & 3 & 4 \\ -1 & 2 & 5 \\ -1 & -1 & 2 \\ \end{array} \right] $
  \correctanswer $ \left[ \begin{array}{ccc} 2 & 3 & 4 \\ -1 & 4 & 5 \\ -1 & -1 & 6 \\ \end{array} \right] $
  \answer $ \left[ \begin{array}{ccc} 2 & -1 & -1 \\ 3 & 2 & -1 \\ 4 & 5 & 2 \\ \end{array} \right] $
  \answer $ \left[ \begin{array}{ccc} -1 & -1 & -1 \\ 2 & -1 & -1 \\ 3 & 4 & -1 \\ \end{array} \right] $
  \answer $ \left[ \begin{array}{ccc} -1 & -1 & -1 \\ 3 & -1 & -1 \\ 4 & 5 & -1 \\ \end{array} \right] $
\end{answers}

\begin{solution}
\end{solution}

\variant %---------------------------------------------------------------------
\begin{Verbatim}
A = eye( 3,3 ) + ones( 3,3 );
for x = 1:3
    for y = 1:3
        if x <= y
            A( x,y ) = x - y;
        end
    end
end
\end{Verbatim}

What is the final value of \texttt{A}?

\begin{answers}
  \answer $ \left[ \begin{array}{ccc} 2 & 1 & 1 \\ 1 & 2 & 1 \\ 2 & 1 & 2 \\ \end{array} \right] $
  \answer $ \left[ \begin{array}{ccc} 0 & 1 & 1 \\ -1 & 0 & 1 \\ -2 & -1 & 0 \\ \end{array} \right] $
  \correctanswer $ \left[ \begin{array}{ccc} 0 & -1 & -2 \\ 1 & 0 & -1 \\ 1 & 1 & 0 \\ \end{array} \right] $
  \answer $ \left[ \begin{array}{ccc} -1 & -1 & -1 \\ 2 & -1 & -1 \\ 3 & 4 & -1 \\ \end{array} \right] $
  \answer $ \left[ \begin{array}{ccc} -1 & -1 & -1 \\ 0 & -1 & -1 \\ 0 & 0 & -1 \\ \end{array} \right] $
\end{answers}

\begin{solution}
\end{solution}

\variant %---------------------------------------------------------------------
\begin{Verbatim}
A = eye( 3,3 ) + ones( 3,3 );
for x = 1:3
    for y = 1:3
        if x <= y
            A( y,x ) = x - y;
        end
    end
end
\end{Verbatim}

What is the final value of \texttt{A}?

\begin{answers}
  \answer $ \left[ \begin{array}{ccc} 2 & 1 & 1 \\ 1 & 2 & 1 \\ 2 & 1 & 2 \\ \end{array} \right] $
  \correctanswer $ \left[ \begin{array}{ccc} 0 & 1 & 1 \\ -1 & 0 & 1 \\ -2 & -1 & 0 \\ \end{array} \right] $
  \answer $ \left[ \begin{array}{ccc} 0 & -1 & -2 \\ 1 & 0 & -1 \\ 1 & 1 & 0 \\ \end{array} \right] $
  \answer $ \left[ \begin{array}{ccc} -1 & -1 & -1 \\ 2 & -1 & -1 \\ 3 & 4 & -1 \\ \end{array} \right] $
  \answer $ \left[ \begin{array}{ccc} -1 & -1 & -1 \\ 0 & -1 & -1 \\ 0 & 0 & -1 \\ \end{array} \right] $

\end{answers}

\begin{solution}
\end{solution}


%%%%%%%%%%%%%%%%%%%%%%%%%%%%%%%%%%%%%%%%%%%%%%%%%%%%%%%%%%%%%%%%%%%%%%%%%%%%%%%
%%%%%%%%%%%%%%%%%%%%%%%%%%%%%%%%%%%%%%%%%%%%%%%%%%%%%%%%%%%%%%%%%%%%%%%%%%%%%%%
\zone \pagebreak \noindent
\textbf{The following 16 questions involve Python.}
\\\\

%%%%%%%%%%%%%%%%%%%%%%%%%%%%%%%%%%%%%%%%%%%%%%%%%%%%%%%%%%%%%%%%%%%%%%%%%%%%%%%
\question{5}
\variant %---------------------------------------------------------------------
Consider the following incomplete Python program:
\begin{Verbatim}
a = 'DWALIN'
b = 'THORIN'
d = { }
for x,y in zip( a,b ):
    ???
s = ''
for c in a:
    s += d[ c ]
\end{Verbatim}

What should replace the three question marks to cause this program to yield a final value for \texttt{s} of \texttt{'THORIN'}?

\begin{answers}
  \correctanswer  \texttt{d[ x ] = y}
  \answer  \texttt{d[ y ] = x}
  \answer  \texttt{d[ a ] = b}
  \answer  \texttt{d[ b ] = a}
  \answer  \texttt{d[ a ] = x}
\end{answers}

\begin{solution}
\end{solution}

\variant %---------------------------------------------------------------------
Consider the following incomplete Python program:
\begin{Verbatim}
a = 'DWALIN'
b = 'THORIN'
d = { }
for x,y in zip( a,b ):
    ???
s = ''
for c in b:
    s += d[ c ]
\end{Verbatim}

What should replace the three question marks to cause this program to yield a final value for \texttt{s} of \texttt{'DWALIN'}?

\begin{answers}
  \answer  \texttt{d[ x ] = y}
  \correctanswer  \texttt{d[ y ] = x}
  \answer  \texttt{d[ a ] = b}
  \answer  \texttt{d[ b ] = a}
  \answer  \texttt{d[ a ] = x}
\end{answers}

\begin{solution}
\end{solution}


%%%%%%%%%%%%%%%%%%%%%%%%%%%%%%%%%%%%%%%%%%%%%%%%%%%%%%%%%%%%%%%%%%%%%%%%%%%%%%%
\question{5}
\variant %---------------------------------------------------------------------
Consider the following Python program:
\begin{Verbatim}
d = { 0:0,1:0,2:0 }
for i in range( 10,15 ):
    d[ i%3 ] += i
x = d[ 1 ]
\end{Verbatim}

What is the final \emph{value} of \texttt{x}?

\begin{answers}
  \answer  \texttt{12}
  \correctanswer  \texttt{23}
  \answer  \texttt{11}
  \answer  \texttt{25}
  \answer  \texttt{1}
\end{answers}

\begin{solution}
\end{solution}

\variant %---------------------------------------------------------------------
Consider the following Python program:
\begin{Verbatim}
d = { 0:0,1:0,2:0 }
for i in range( 11,18 ):
    d[ i%3 ] += i
x = d[ 1 ]
\end{Verbatim}

What is the final \emph{value} of \texttt{x}?

\begin{answers}
  \answer  \texttt{40}
  \answer  \texttt{42}
  \answer  \texttt{45}
  \answer  \texttt{29}
  \correctanswer  \texttt{27}
\end{answers}

\begin{solution}
\end{solution}


%%%%%%%%%%%%%%%%%%%%%%%%%%%%%%%%%%%%%%%%%%%%%%%%%%%%%%%%%%%%%%%%%%%%%%%%%%%%%%%
\question{5}
\variant %---------------------------------------------------------------------
Consider the following Python program:
\begin{Verbatim}
d = { "B":1,"A":1,"G":2,"I":1,"N":1,"S":1 }
for c in "BILBO":
    print( d[ c ] + '-' )
\end{Verbatim}

What kind of exception will this program throw?

\begin{answers}
  \answer \texttt{KeyError: 'L'}
  \answer \texttt{TypeError: list indices must be integers, not str}
  \answer \texttt{SyntaxError: invalid syntax}
  \correctanswer \texttt{TypeError: unsupported operand type(s) for +: 'int' and 'str'}
\end{answers}

\begin{solution}
\end{solution}

\variant %---------------------------------------------------------------------
Consider the following Python program:
\begin{Verbatim}
d = { "B":1,"A":1,"G":2,"I":1,"N" 1,"S":1 }
for c in "BILBO":
    print( d[ c ] + '-' )
\end{Verbatim}

What kind of exception will this program throw?

\begin{answers}
  \answer \texttt{KeyError: 'L'}
  \answer \texttt{TypeError: list indices must be integers, not str}
  \correctanswer \texttt{SyntaxError: invalid syntax}
  \answer \texttt{TypeError: unsupported operand type(s) for +: 'int' and 'str'}
\end{answers}

\begin{solution}
\end{solution}


%%%%%%%%%%%%%%%%%%%%%%%%%%%%%%%%%%%%%%%%%%%%%%%%%%%%%%%%%%%%%%%%%%%%%%%%%%%%%%%
\question{5}
\variant %---------------------------------------------------------------------
Consider the following Python program:
\begin{Verbatim}
e = list( range( 0,10,2 ) )
d = [ 0,0,0,0 ]
for i in range( 0,len(e) ):
    d[ i%4 ] += e[ i ]
x = d[ 1 ]
\end{Verbatim}

What is the final \emph{value} of \texttt{x}?

\begin{answers}
  \answer  \texttt{0}
  \answer  \texttt{8}
  \answer  \texttt{10}
  \correctanswer  \texttt{2}
  \answer  \texttt{14}
\end{answers}

\begin{solution}
\end{solution}

\variant %---------------------------------------------------------------------
Consider the following Python program:
\begin{Verbatim}
e = list( range( 0,10,2 ) )
d = [ 0,0,0,0 ]
for i in range( 0,len(e) ):
    d[ i%4 ] += e[ i ]
x = d[ 2 ]
\end{Verbatim}

What is the final \emph{value} of \texttt{x}?

\begin{answers}
  \answer  \texttt{0}
  \answer  \texttt{8}
  \answer  \texttt{10}
  \answer  \texttt{2}
  \correctanswer  \texttt{4}
\end{answers}

\begin{solution}
\end{solution}


%%%%%%%%%%%%%%%%%%%%%%%%%%%%%%%%%%%%%%%%%%%%%%%%%%%%%%%%%%%%%%%%%%%%%%%%%%%%%%%
\question{5}
\variant %---------------------------------------------------------------------
Consider the following incomplete Python program:
\begin{Verbatim}
sum = 0
???:
    sum += i
\end{Verbatim}

The program is intended to sum all of the integers between 1 and 100 (inclusive). What should replace the three question marks to complete the program?

\begin{answers}
  \answer  \texttt{for i in range( 0,100 )}
  \answer  \texttt{while i <= 100}
  \correctanswer  \texttt{for i in range( 1, 101 )}
  \answer  \texttt{while i in range( 100 )}
\end{answers}

\begin{solution}
\end{solution}

\variant %---------------------------------------------------------------------
Consider the following incomplete Python program:
\begin{Verbatim}
sum = 0
for i in range( 0,100 ):
    ???
\end{Verbatim}

The program is intended to sum all of the integers between 1 and 100 (inclusive). What should replace the three question marks to complete the program?

\begin{answers}
  \answer  \texttt{sum += 1}
  \answer  \texttt{sum + 1 = sum}
  \correctanswer  \texttt{sum += i + 1}
  \answer  \texttt{sum += i}
\end{answers}

\begin{solution}
\end{solution}


%%%%%%%%%%%%%%%%%%%%%%%%%%%%%%%%%%%%%%%%%%%%%%%%%%%%%%%%%%%%%%%%%%%%%%%%%%%%%%%
\question{5}
\variant %---------------------------------------------------------------------
\begin{Verbatim}
x = np.array( [ [ 2 ] , [ 3 ] ] * 2 )
\end{Verbatim}

What is the final \emph{value} of \texttt{x}?

\begin{answers}
  \answer  $ \left[ \begin{array}{cc} 2 & 2 \\ 3 & 3 \end{array} \right] $
  \correctanswer  $ \left[ \begin{array}{c} 2 \\ 3 \\ 2 \\ 3 \end{array} \right] $
  \answer  $ \left[ \begin{array}{cccc} 2 & 3 & 2 & 3 \end{array} \right] $
  \answer  $ \left[ \begin{array}{cc} 2 & 3 \\ 2 & 3 \end{array} \right] $
\end{answers}

\begin{solution}
\end{solution}

\variant %---------------------------------------------------------------------
\begin{Verbatim}
x = np.array( [ [ 2 ] , [ 3 ] ] * 3 )
\end{Verbatim}

What is the final \emph{value} of \texttt{x}?

\begin{answers}
  \answer  $ \left[ \begin{array}{ccc} 2 & 2 & 2 \\ 3 & 3 & 3 \end{array} \right] $
  \correctanswer  $ \left[ \begin{array}{c} 2 \\ 3 \\ 2 \\ 3 \\ 2 \\ 3 \end{array} \right] $
  \answer  $ \left[ \begin{array}{cccccc} 2 & 3 & 2 & 3 & 2 & 3 \end{array} \right] $
  \answer  $ \left[ \begin{array}{ccc} 2 & 3 \\ 2 & 3 \\ 2 & 3 \end{array} \right] $
\end{answers}

\begin{solution}
\end{solution}


%%%%%%%%%%%%%%%%%%%%%%%%%%%%%%%%%%%%%%%%%%%%%%%%%%%%%%%%%%%%%%%%%%%%%%%%%%%%%%%
\question{5}
\variant %---------------------------------------------------------------------
\begin{Verbatim}
import itertools
x = 'beorn'
???
    print( x )
\end{Verbatim}

Replacing the three question marks with which of the following will result in \texttt{'beorn'} being printed exactly five times?

\begin{answers}
  \answer  \texttt{for a in itertools.combinations(x,5):}
  \answer  \texttt{for a in itertools.combinations(x,2):}
  \answer  \texttt{for a in itertools.combinations(x,3):}
  \correctanswer  \texttt{for a in itertools.combinations(x,4):}
\end{answers}

\begin{solution}
\end{solution}

\variant %---------------------------------------------------------------------
\begin{Verbatim}
import itertools
x = 'smaug'
???
    print( x )
\end{Verbatim}

Replacing the three question marks with which of the following will result in \texttt{'smaug'} being printed exactly one time?

\begin{answers}
  \correctanswer  \texttt{for a in itertools.combinations(x,5):}
  \answer  \texttt{for a in itertools.combinations(x,2):}
  \answer  \texttt{for a in itertools.combinations(x,3):}
  \answer  \texttt{for a in itertools.combinations(x,4):}
\end{answers}

\begin{solution}
\end{solution}


%%%%%%%%%%%%%%%%%%%%%%%%%%%%%%%%%%%%%%%%%%%%%%%%%%%%%%%%%%%%%%%%%%%%%%%%%%%%%%%
\question{5}
\variant %---------------------------------------------------------------------
Consider the following incomplete Python program:
\begin{Verbatim}
y = 1.0   # initial position, m
v = 0.0   # initial velocity, m/s
g = -9.8  # acceleration due to gravity, m/s^2
t = ???   # initial time, s
nt = ???  # number of time intervals, -
dt = t/nt # time increment, s

while y > 0.0:
    t += dt
    v += g * dt
    y += v * dt
\end{Verbatim}

Which of the following values for \texttt{t} and \texttt{nt} will yield the most accurate solution?

\begin{answers}
  \correctanswer  \texttt{t,nt = 1.0,1e5}
  \answer  \texttt{t,nt = 10.0,1e3}
  \answer  \texttt{t,nt = 10.0,1e4}
  \answer  \texttt{t,nt = 1.0,10}
\end{answers}

\begin{solution}
\end{solution}


%%%%%%%%%%%%%%%%%%%%%%%%%%%%%%%%%%%%%%%%%%%%%%%%%%%%%%%%%%%%%%%%%%%%%%%%%%%%%%%
\question{5}
\variant %---------------------------------------------------------------------
\begin{Verbatim}
s = 'THRANDUIL'
x = ''
for i in range( 0,len( s ) ):
    if ( i>3 ) and ( i<6 ):
        x += s[ i:i+2 ]
\end{Verbatim}

What is the \emph{value} of \texttt{x} after this program is executed?

\begin{answers}
  \correctanswer  \texttt{'NDDU'}
  \answer  \texttt{'ANND'}
  \answer  \texttt{'AN'}
  \answer  \texttt{'ND'}
  \answer  None of the other answers are correct.
\end{answers}

\begin{solution}
\end{solution}

\variant %---------------------------------------------------------------------
\begin{Verbatim}
s = 'ELROND'
x = ''
for i in range( 0,len( s ) ):
    if ( i>2 ) and ( i<5 ):
        x += s[ i:i+2 ]
\end{Verbatim}

What is the \emph{value} of \texttt{x} after this program is executed?

\begin{answers}
  \correctanswer  \texttt{'ONND'}
  \answer  \texttt{'ROON'}
  \answer  \texttt{'RO'}
  \answer  \texttt{'ND'}
  \answer  None of the other answers are correct.
\end{answers}

\begin{solution}
\end{solution}


%%%%%%%%%%%%%%%%%%%%%%%%%%%%%%%%%%%%%%%%%%%%%%%%%%%%%%%%%%%%%%%%%%%%%%%%%%%%%%%
\question{5}
\variant %---------------------------------------------------------------------
\begin{Verbatim}
def sum_pairs( A ):
    total = 0
    ???
    return total
\end{Verbatim}

The function \texttt{sum\_pairs} accepts a list of floats named \texttt{A}.  \texttt{sum\_pairs} should return the sum of all pairs of values in the list (without repeats).  For example, given the list \texttt{[ 1,2,3 ]}, \texttt{sum\_pairs} should return \texttt{12} from $(1+2) + (1+3) + (2+3) = 12$.  What should replace the three question marks to complete the function?  (Assume any necessary \texttt{import}s to have taken place already.)

\begin{answers}
  \correctanswer
    \begin{Verbatim}
for i in range( len( A ) ):
    for j in range( i+1,len( A ) ):
        total += A[ i ] + A[ j ]
    \end{Verbatim}
  \answer
    \begin{Verbatim}
for i in range( len( A ) ):
    for j in range( len( A ) ):
        total += A[ i ] + A[ j ]
    \end{Verbatim}
  \answer
    \begin{Verbatim}
for i,j in enumerate( A ):
        total += A[ i ] + A[ j ]
    \end{Verbatim}
  \answer
    \begin{Verbatim}
for i in itertools.permutations( A ):
        total += i[ 0 ] + i[ 1 ]
    \end{Verbatim}
\end{answers}

\begin{solution}
\end{solution}


%%%%%%%%%%%%%%%%%%%%%%%%%%%%%%%%%%%%%%%%%%%%%%%%%%%%%%%%%%%%%%%%%%%%%%%%%%%%%%%
\question{5}
\variant %---------------------------------------------------------------------
What do we call the optimization heuristic that involves iteratively checking to see if neighboring solutions improve upon the current solution?

\begin{answers}
  \answer  Conjugate gradient
  \answer  Local optimum
  \correctanswer  Hill-climbing
  \answer  Random search
\end{answers}

\begin{solution}
\end{solution}

\variant %---------------------------------------------------------------------
What do we call the optimization heuristic that involves choosing the best from a stochastically sampled subset of the domain?

\begin{answers}
  \answer  Brute-force search
  \answer  Local optimum
  \answer  Gradient descent
  \correctanswer  Random search
\end{answers}

\begin{solution}
\end{solution}

\variant %---------------------------------------------------------------------
What do we call the optimization heuristic that involves taking \emph{any} available improvement on the current solution?

\begin{answers}
  \correctanswer  Brute-force search
  \answer  Conjugate gradient
  \answer  Gradient descent
  \answer  Random search
\end{answers}

\begin{solution}
\end{solution}


%%%%%%%%%%%%%%%%%%%%%%%%%%%%%%%%%%%%%%%%%%%%%%%%%%%%%%%%%%%%%%%%%%%%%%%%%%%%%%%
\question{5}
\variant %---------------------------------------------------------------------
\begin{Verbatim}
def total_sales( sales_file ):
    d = { }
    for line in open( sales_file ):
        ???
    return d
\end{Verbatim}

The function \texttt{total\_sales} should compute the total sales of each employee working for a company by reading a comma-separated value input file of employee sale data.  The result should be returned from the function as a dictionary.  The first column of each line in the input file is expected to contain the employee's name represented as a string.  The second column is expected to contain a floating point number representing the total for that sale.  Here is an example input file:
\begin{Verbatim}
Tom,10.0
Bill,10.55
Bill,115.50
Your program should ignore a non-conforming line like this one.
Bert,30.25
\end{Verbatim}

The resulting return value for this file should be the following dictionary:
\begin{Verbatim}
{ 'Bert':30.25, 'Bill':126.05, 'Tom':10.0 }
\end{Verbatim}

What should replace the three question marks to complete the function?

\begin{answers}
  \correctanswer
    \begin{Verbatim}
try:
    s,f = line.split( "," )
    if s not in d:
        d[ s ] = 0.0
    d[ s ] += float( f )
except:
    continue
    \end{Verbatim}
  \answer
    \begin{Verbatim}
if line not in d:
    d[ line ] = 0.0
try:
    s,f = line.split( "," )
except:
    d[ s ] += float( f )
    continue
    \end{Verbatim}
  \answer
    \begin{Verbatim}
try:
    s,f = line.split( "," )
except:
    continue
if f not in d:
    d[ f ] = 0.0
d[ f ] += float( s )
    \end{Verbatim}
  \answer
    \begin{Verbatim}
  try:
      s,f = line.split()
      d[ s ] += float( f )
  except:
      break
    \end{Verbatim}
\end{answers}

\begin{solution}
\end{solution}


%%%%%%%%%%%%%%%%%%%%%%%%%%%%%%%%%%%%%%%%%%%%%%%%%%%%%%%%%%%%%%%%%%%%%%%%%%%%%%%
\question{5}
\variant %---------------------------------------------------------------------
\begin{Verbatim}
s = ''.join( [ "0","1","2","1" ] )
x = 0
for i in range( len( s )-1 ):
    x += int( ??? )
\end{Verbatim}

What should replace the three question marks so the resulting value of x is 34?

\begin{answers}
  \answer \texttt{s[ i:i+2:i ]}
  \answer \texttt{s[ i:i+1 ]}
  \correctanswer \texttt{s[ i+2:i:-1 ]}
  \answer \texttt{s[ i+1:i+2 ]}
\end{answers}

\begin{solution}
\end{solution}

\variant %---------------------------------------------------------------------
\begin{Verbatim}
s = ''.join( [ "2","2","0","1" ] )
x = 0
for i in range( len( s )-1 ):
    x += int( ??? )
\end{Verbatim}

What should replace the three question marks so the resulting value of x is 13?

\begin{answers}
  \answer \texttt{s[ i:i+2:i ]}
  \answer \texttt{s[ i:i+1 ]}
  \correctanswer \texttt{s[ i+2:i:-1 ]}
  \answer \texttt{s[ i+1:i+2 ]}
\end{answers}

\begin{solution}
\end{solution}

\variant %---------------------------------------------------------------------
\begin{Verbatim}
s = ''.join( [ "1","1","0","2" ] )
x = 0
for i in range( len( s )-1 ):
    x += int( ??? )
\end{Verbatim}

What should replace the three question marks so the resulting value of x is 23?

\begin{answers}
  \answer \texttt{s[ i:i+2:i ]}
  \answer \texttt{s[ i:i+1 ]}
  \correctanswer \texttt{s[ i+2:i:-1 ]}
  \answer \texttt{s[ i+1:i+2 ]}
\end{answers}

\begin{solution}
\end{solution}


%%%%%%%%%%%%%%%%%%%%%%%%%%%%%%%%%%%%%%%%%%%%%%%%%%%%%%%%%%%%%%%%%%%%%%%%%%%%%%%
\question{5}
\variant %---------------------------------------------------------------------
\begin{Verbatim}
x = [ ]
for i in range( 1,101 ):
    for j in range( i+1,101 ):
        t = i,j
        x.append( t )
\end{Verbatim}

After the program runs, which of the following is an element of \texttt{x}?

\begin{answers}
  \correctanswer \texttt{(10,52)}
  \answer \texttt{(0,33)}
  \answer \texttt{(42,15)}
  \answer \texttt{(78,78)}
  \answer \texttt{(11,4)}
\end{answers}

\begin{solution}
\end{solution}

\variant %---------------------------------------------------------------------
\begin{Verbatim}
x = [ ]
for i in range( 1,101 ):
    for j in range( i+1,101 ):
        t = i,j
        x.append( t )
\end{Verbatim}

After the program runs, which of the following is \emph{not} an element of \texttt{x}?

\begin{answers}
  \correctanswer \texttt{(55,55)}
  \answer \texttt{(4,33)}
  \answer \texttt{(19,32)}
  \answer \texttt{(78,100)}
  \answer \texttt{(1,20)}
\end{answers}

\begin{solution}
\end{solution}


%%%%%%%%%%%%%%%%%%%%%%%%%%%%%%%%%%%%%%%%%%%%%%%%%%%%%%%%%%%%%%%%%%%%%%%%%%%%%%%
\question{5}
\variant %---------------------------------------------------------------------
\begin{Verbatim}
e = [ 1,1,2,2,3,3,4,4,5,5 ]
d = { 0:0,1:0,2:0 }
for a,b in enumerate( e ):
    d[ a%3 ] += b
x = d[ 1 ]
\end{Verbatim}

After it is run, what is the final \emph{value} of \texttt{x}?

\begin{answers}
  \answer \texttt{3}
  \answer \texttt{10}
  \answer \texttt{12}
  \answer \texttt{22}
  \correctanswer \texttt{8}
\end{answers}

\begin{solution}
\end{solution}

\variant %---------------------------------------------------------------------
\begin{Verbatim}
e = [ 5,5,4,4,3,3,2,2,1,1 ]
d = { 0:0,1:0,2:0 }
for a,b in enumerate( e ):
    d[ a%3 ] += b
x = d[ 2 ]
\end{Verbatim}

After it is run, what is the final \emph{value} of \texttt{x}?

\begin{answers}
  \answer \texttt{3}
  \answer \texttt{10}
  \answer \texttt{12}
  \answer \texttt{22}
  \correctanswer \texttt{8}
\end{answers}

\begin{solution}
\end{solution}


%%%%%%%%%%%%%%%%%%%%%%%%%%%%%%%%%%%%%%%%%%%%%%%%%%%%%%%%%%%%%%%%%%%%%%%%%%%%%%%
\question{5}
\variant %---------------------------------------------------------------------
\begin{Verbatim}
x = "5 4 1".split()
x = x.sort()
try:
    print( len( x ) )
except:
    print( type( x ) )
\end{Verbatim}

After it is run, what is printed by this program?

\begin{answers}
  \answer \texttt{TypeError}
  \answer \texttt{3}
  \answer \texttt{list}
  \correctanswer \texttt{NoneType}
\end{answers}

\begin{solution}
\end{solution}

\variant %---------------------------------------------------------------------
\begin{Verbatim}
x = "1 2 3".split()
x = ','.join( x )
try:
    print( x.append( 4 ) )
except:
    print( type( x ) )
\end{Verbatim}

After it is run, what is printed by this program?

\begin{answers}
  \answer \texttt{TypeError}
  \answer \texttt{[1,2,3,4]}
  \answer \texttt{list}
  \correctanswer \texttt{str}
\end{answers}

\begin{solution}
\end{solution}


%%%%%%%%%%%%%%%%%%%%%%%%%%%%%%%%%%%%%%%%%%%%%%%%%%%%%%%%%%%%%%%%%%%%%%%%%%%%%%%
%%%%%%%%%%%%%%%%%%%%%%%%%%%%%%%%%%%%%%%%%%%%%%%%%%%%%%%%%%%%%%%%%%%%%%%%%%%%%%%
\zone \pagebreak \noindent

\question{25}
\variant %---------------------------------------------------------------------

You have been hired by a private investigation firm to crack an smartphone of indeterminate provenance (and a process of questionable legality).  The default password is exactly five characters long, with possible characters selected from the upper- and lower-case alphabets and the ten digits \texttt{0} to \texttt{9}.  Assume that you have available a function \texttt{test\_password} which returns \texttt{True} if the password is correct and \texttt{False} otherwise.

Compose a Python function \texttt{crack\_phone} which accepts no arguments and returns a string representing the correct password which unlocks the smartphone.  You may \texttt{import itertools} in your solution if you prefer, but no other libraries are allowed.

\begin{Verbatim}
alphabet = 'ABCDEFGHIJKLMNOPQRSTUVWXYZabcdefghijklmnopqrstuvwxyz0123456789'
\end{Verbatim}

\begin{solution}
\end{solution}


%%%%%%%%%%%%%%%%%%%%%%%%%%%%%%%%%%%%%%%%%%%%%%%%%%%%%%%%%%%%%%%%%%%%%%%%%%%%%%%
%%%%%%%%%%%%%%%%%%%%%%%%%%%%%%%%%%%%%%%%%%%%%%%%%%%%%%%%%%%%%%%%%%%%%%%%%%%%%%%
\zone \pagebreak \noindent

\question{25}
\variant %---------------------------------------------------------------------
Consider the Taylor series definition of the sine function:
$$
\sin (x) = x
+ \frac{x^3}{3!}
+ \frac{x^5}{5!}
+ \frac{x^7}{7!}
+ ...
$$
The series converges for all real $x$, so to calculate $\sin(x)$ to within a few decimal places of accuracy one just needs to include sufficient terms in the calculation.

The following MATLAB function \texttt{sine} was written in order to calculate the value of $\sin(x)$ for all $x$ to three decimal places of accuracy (\texttt{atol} in the code).  Translate this function into a Python function---also called \texttt{sine}---which yields identical output from the function as the MATLAB function for given input.  You may \texttt{import numpy as np} in your solution if you prefer, but no other libraries are allowed.  (Assume a valid NumPy-compatible function \texttt{factorial} is also available.)

\begin{Verbatim}
function [ y ] = sine( x )
    y = 0;
    yold = 1;
    n = 0;
    atol = 1e-3;  % tolerance
    while ( abs( y-yold ) > atol )
        yold = y;
        term = ( x .^ ( 2*n+1 ) ) / factorial( ( 2*n+1 ) );
        if (mod(n,2) == 1)
            term = -term;
        end
        y = y + term;
        n = n + 1;
    end
end
\end{Verbatim}

\begin{solution}
\end{solution}
% practice:  exp

\end{document}
